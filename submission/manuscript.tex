\documentclass[12pt,]{article}
\usepackage{lmodern}
\usepackage{amssymb,amsmath}
\usepackage{ifxetex,ifluatex}
\usepackage{fixltx2e} % provides \textsubscript
\ifnum 0\ifxetex 1\fi\ifluatex 1\fi=0 % if pdftex
  \usepackage[T1]{fontenc}
  \usepackage[utf8]{inputenc}
\else % if luatex or xelatex
  \ifxetex
    \usepackage{mathspec}
  \else
    \usepackage{fontspec}
  \fi
  \defaultfontfeatures{Ligatures=TeX,Scale=MatchLowercase}
\fi
% use upquote if available, for straight quotes in verbatim environments
\IfFileExists{upquote.sty}{\usepackage{upquote}}{}
% use microtype if available
\IfFileExists{microtype.sty}{%
\usepackage{microtype}
\UseMicrotypeSet[protrusion]{basicmath} % disable protrusion for tt fonts
}{}
\usepackage[margin=1.0in]{geometry}
\usepackage{hyperref}
\hypersetup{unicode=true,
            pdfborder={0 0 0},
            breaklinks=true}
\urlstyle{same}  % don't use monospace font for urls
\usepackage{longtable,booktabs}
\usepackage{graphicx,grffile}
\makeatletter
\def\maxwidth{\ifdim\Gin@nat@width>\linewidth\linewidth\else\Gin@nat@width\fi}
\def\maxheight{\ifdim\Gin@nat@height>\textheight\textheight\else\Gin@nat@height\fi}
\makeatother
% Scale images if necessary, so that they will not overflow the page
% margins by default, and it is still possible to overwrite the defaults
% using explicit options in \includegraphics[width, height, ...]{}
\setkeys{Gin}{width=\maxwidth,height=\maxheight,keepaspectratio}
\IfFileExists{parskip.sty}{%
\usepackage{parskip}
}{% else
\setlength{\parindent}{0pt}
\setlength{\parskip}{6pt plus 2pt minus 1pt}
}
\setlength{\emergencystretch}{3em}  % prevent overfull lines
\providecommand{\tightlist}{%
  \setlength{\itemsep}{0pt}\setlength{\parskip}{0pt}}
\setcounter{secnumdepth}{0}
% Redefines (sub)paragraphs to behave more like sections
\ifx\paragraph\undefined\else
\let\oldparagraph\paragraph
\renewcommand{\paragraph}[1]{\oldparagraph{#1}\mbox{}}
\fi
\ifx\subparagraph\undefined\else
\let\oldsubparagraph\subparagraph
\renewcommand{\subparagraph}[1]{\oldsubparagraph{#1}\mbox{}}
\fi

%%% Use protect on footnotes to avoid problems with footnotes in titles
\let\rmarkdownfootnote\footnote%
\def\footnote{\protect\rmarkdownfootnote}

%%% Change title format to be more compact
\usepackage{titling}

% Create subtitle command for use in maketitle
\newcommand{\subtitle}[1]{
  \posttitle{
    \begin{center}\large#1\end{center}
    }
}

\setlength{\droptitle}{-2em}
  \title{}
  \pretitle{\vspace{\droptitle}}
  \posttitle{}
  \author{}
  \preauthor{}\postauthor{}
  \date{}
  \predate{}\postdate{}

\usepackage{helvet} % Helvetica font
\renewcommand*\familydefault{\sfdefault} % Use the sans serif version of the font
\usepackage[T1]{fontenc}

\usepackage[none]{hyphenat}

\usepackage{setspace}
\doublespacing
\setlength{\parskip}{1em}

\usepackage{lineno}

\usepackage{pdfpages}

\begin{document}

\section{The Microbiota and Individual Community Members in Colorectal
Cancer: Is There a Common
Theme?}\label{the-microbiota-and-individual-community-members-in-colorectal-cancer-is-there-a-common-theme}

\begin{center}
\vspace{25mm}

Marc A Sze${^1}$ and Patrick D Schloss${^1}$${^\dagger}$

\vspace{20mm}

$\dagger$ To whom correspondence should be addressed: pschloss@umich.edu

$1$ Department of Microbiology and Immunology, University of Michigan, Ann Arbor, MI




\end{center}

Co-author e-mails:

\begin{itemize}
\tightlist
\item
  \href{mailto:marcsze@med.umich.edu}{\nolinkurl{marcsze@med.umich.edu}}
\end{itemize}

\newpage

\linenumbers

\subsection{Abstract}\label{abstract}

\textbf{Background.} An increasing body of literature suggests that
there is a role for the microbiota in colorectal cancer. Important
drivers within this axis have ranged from individual microbes to the
whole community. A recent meta-analysis investigated whether consistent
biomarkers could be identified across studies. This report expands on
this previous research and tests the hypothesis that the bacterial
community is an important driver of disease from control to adenoma to
carcinoma. To test this hypothesis we examined both feces (total
individuals = 1737) and tissue (total samples = 492) across 14 different
studies.

\textbf{Results.} Fecal samples had a significant decrease from control
to adenoma to carcinoma for both Shannon diversity and evenness (P-value
\textless{} 0.05) after correcting for study effect and variable region
sequenced. Lower Shannon diversity and evenness in fecal samples
resulted in a significant increase in relative risk for carcinoma
(P-value \textless{} 0.05) while only evenness for adenoma (P-value
\textless{} 0.05) resulted in a slightly increased relative risk.
Previously associated colorectal cancer genera (\emph{Fusobacterium},
\emph{Parvimonas}, \emph{Peptostreptococcus}, or \emph{Porphyromonas})
followed a similar pattern with a significantly increased relative risk
by their presence for carcinoma (P-value \textless{} 0.05) but not
adenoma (P-value \textgreater{} 0.05) with the exception of
\emph{Porphyromonas} (P-value \textless{} 0.05). Using the whole
community versus only CRC associated genera to build a prediction model
resulted in a higher classification AUC for both adenoma and carcinoma
for fecal and tissue samples. For the studies that were analyzed, most
were adequately powered for large effect size differences which may be
more amenable for carcinoma than for adenoma microbiota research.

\textbf{Conclusions.} This data provides support for the importance of
the bacteral community to both adenoma and carcinoma genesis. The
evidence that has been collated within this study is much stronger for
carcinoma and this may be in part due to the low power to detect more
subtle changes in the majority of studies that have been performed to
date.

\subsubsection{Keywords}\label{keywords}

microbiota; colorectal cancer; polyps; adenoma; meta-analysis.

\newpage

\subsection{Background}\label{background}

Colorectal cancer (CRC) is a growing world wide health problem {[}1{]}
in which the microbiota has been purported to play an active role in
disease pathogenesis {[}2{]}. Numerous studies have shown the importance
of both individual microbes {[}3--7{]} and the overall community
{[}8--10{]} in polyp formation in mouse models. There has also been
numerous case/control studies investigating the microbiota in both
adenoma and carcinoma. Recently, a meta-analysis was published
investigating whether specific biomarkers could be consistently
identified using multiple data sets {[}11{]}. Many of the studies along
with the current meta-analysis focus on identifying biomarkers or
individual microbes but do not critically investigate the community role
in the disease.

Using both fecal (total individuals = 1737) and tissue samples (total
samples = 492) totalling over 2229 total samples across 14 studies
{[}12--25{]} we expand both the breadth and scope of the previous
meta-analysis; to investigate whether the bacterial community is an
important risk factor for both adenoma and carcinoma. To accomplish this
we first assessed whether the diversity changes across disease (control
to adenoma to carcinoma) and if it results in an increased relative risk
for adenoma or carcinoma. Next, we assessed how typical CRC associated
genera (\emph{Fusobacterium}, \emph{Parvimonas},
\emph{Peptostreptococcus}, or \emph{Porphyromonas}) affect the relative
risk of adenoma or carcinoma. Third, using Random Forest models we
analyzed whether the full community or only the CRC associated genera
resulted in better model classification area under the curve (AUC).
Finally, we examined at what effect size and sample n the studies that
were used were adequately powered for.

Our analysis found a continuous decrease in Shannon diversity from
control to adenoma to carcinoma and a significantly increased relative
risk for carcinoma with this lower diversity. Using only the CRC
associated genera this relative risk was higher than what was observed
for Shannon diversity. However, by using the full community instead of
only the CRC associated genera the AUC of the classification models
increased. Although we analyze a data set with a large number of total
individuals each individual study was underpowered for effect size
differences of 10\% or below between the case and control.

\newpage

\subsection{Results}\label{results}

\textbf{\emph{Diversity is Lower in Those with Carcinoma and Increases
Relative Risk:}} Using power transformed and Z-score normalized alpha
diversity metrics, both evenness and Shannon diversity in feces and not
tissue were lower in those with carcinoma {[}Figure 1{]}. Using linear
mixed-effect models to control for study and variable region there was a
significant decrease from control to adenoma to carcinoma for both
evenness (P-value = 0.025) and Shannon diversity (P-value = 0.043). This
effect was not observed in tissue when additionally controlling for
whether the sample came from the same individual (P-value \textgreater{}
0.05). For fecal samples a decrease in Shannon diversity and evenness
resulted in a significantly increased relative risk for carcinoma
(P-value = 0.01 and P-value = 0.0011, respectively) {[}Figure 2{]}.
Although these values were significant, the effect size was relatively
small for both metrics (Shannon RR = 1.31 and evenness RR = 1.34)
{[}Figure 2{]}. Only a decrease in evenness had an increased relative
risk for adenoma (P-value = 0.032) {[}Figure 2A \& S1{]} but this effect
size was even smaller than what was observed for carcinoma (RR = 1.16).

Using the Bray-Curtis distance metric, there was a significant
difference across studies in the bacterial community of fecal samples
between carcinoma and controls but not between adenoma and controls
{[}Table S1 \& S2{]}. Many studies with unmatched tissue samples had a
significant difference for both adenoma and carcinoma versus controls
{[}Table S3 \& S4{]} while studies with matched tissue samples had no
differences {[}Table S3 \& S4{]}.

\textbf{\emph{Genera Previously Associated with Carcinoma Predominately
Increases Relative Risk for Carcinoma but not Adenoma:}} The majority of
CRC associated genera for both feces and tissue had a significantly
increased relative risk for carcinoma but not for adenoma {[}Figure
3{]}. In fecal samples the relative risk due to CRC associated genera
was greater than either evenness or Shannon diversity {[}Figure 2 \&
3{]}. Additionally, the relative risk of carcinoma continuously
increased as individuals tested positive for more CRC associated genera
{[}Figure 3B \& 3D{]}. The relative risk effect size was greater for
stool (RR range = 1.61 - 2.74) then for tissue (RR range = 1.21 - 1.81).
This decrease may be explained by the fact that the tissue analysis
include matched samples.

Two measures in stool for adenoma were significant when investigating
these CRC associated genera. The first was \emph{Porphyromonas} (P-value
= 0.023) and the second was being positive for three CRC associated
genera (P-value = 0.022) {[}Figure 3A{]}. For tissue three measures for
adenoma were significant; being positive for one CRC associated genera
(P-value = 0.032), being positive for two CRC associated genera (P-value
= 0.008), and being positive for four CRC associated genera (P-value =
0.039) {[}Figure 3C{]}.

\textbf{\emph{Using the Whole Community Increases Model AUC over CRC
Associated Genera:}} For both fecal and tissue samples (matched and
unmatched) the AUC decreases when only OTUs from the CRC associated
genera are used {[}Figure 4 \& 5{]}. This decrease is observed in both
adenoma and carcinoma groups {[}Figure 4 \& 5{]}. The genus models
generally had similar trends as observed for the OTU based models with
the full genera models performing better then the CRC associated genera
models {[}Figure S2-S3{]}. Both genus models perform similarily in their
ability to be able to predict lesion (adenoma or carcinoma) with
carcinoma having a higher AUC then adenoma {[}Figure S4-S5{]}. Matched
tissue samples for those with carcinoma had an AUC that was more similar
to the adenoma models {[}Figure S4A, S5B, \& S6{]} then carcinoma models
{[}Figure S4B \& S5A{]}.

\textbf{\emph{Majority of Studies are Underpowered for Detecting Small
Effect Size Differences:}} When assessing the power of each study at
different effect sizes the majority of studies for both adenoma and
carcinoma have an 80\% power to detect a 30\% difference {[}Figure 6A \&
B{]}. No single study that was analyzed had the standard 80\% power to
detect an effect size difference that was eqaul to or below 10\%
{[}Figure 6A \& B{]}. In order to achieve adequate power for small
effect sizes it would be necessary to recruit over 1000 individuals for
each arm of the study {[}Figure 6C{]}.

\newpage

\subsection{Discussion}\label{discussion}

Our study identifies clear diversity changes both at the community level
and within individual genera that are present in individuals with
carcinoma versus those without the disease. Although there was a step
wise decrease in diversity from control to adenoma to carcinoma; this
did not translate into large effect sizes for the relative risk of
either of these two conditions. Even though CRC associated genera
increases the relative risk of carcinoma it does not consistently
increase the relative risk of adenoma. This information suggests that
these specific genera are important in carcinoma genesis but may not be
the primary members of the microbial community that contribute to the
formation of an adenoma. Additionally, our data shows that by using the
whole community our models perform better then when they only use the
CRC associated genera. CRC associated genera are clearly important to
carcinoma but the context or community in which these microbes are a
part of can drastically increase the ability of models to make
predictions. This data supports the concept that small localized changes
within the community may be occuring that are important in the disease
progression of colorectal cancer and that they may not directly involve
CRC associated genera.

The driver-passenger model of the microbial role in CRC, as summarized
by Flynn {[}2{]}, can be supported with this data for carcinoma but not
necessarily for adenoma. The drasitically increased relative risk of
disease, when considering the CRC associated genera, is highly
supportive of this type of process, especially in the context of
increasing relative risk with more CRC associated genera positivity. It
is also possible that in a driver-passenger scenario that simply having
the driver present or only identifying the passenger is a good enough
proxy that the event is occuring. This would account for the observation
that there is no constant additive effect on relative risk for
increasing positivity. Additionally, the initial establishment of the
driver within the system is also dependent on the community that is
present and this is supported by the observation that when adding the
community context to our models along with the CRC associated genera the
model AUC increases.

Our carcinoma observations fit the driver-passenger model and support
this concept within the framework of the transition from adenoma to
carcinoma. In contrast, with the present data we can only suggest that
the adenoma observations might fit with this model but the changes that
occur at this timepoint are small and possibly focal to the adenoma. The
stepwise decrease in diversity suggests that the adenoma community is
not normal but this change is subtle. Although there may be localized
changes that do depend on the driver-passenger model, supported by an
increased relative risk for one, two, and four positive CRC associated
genera in tissue {[}Figure 3C{]}, there may be other processes involved
that ultimately exacerbates the condition from a subtle localized change
to a global community one. The poor performance of the Random Forest
models for classifying adenoma based only on the microbiota would
suggest that this is the case. It is possible to hypothesize that at
early stages of the diease, how the host interacts to these subtle
changes could be the catlyst that ultimately leads to this larger global
dysfunctional community.

Although there are still questions that need to be answered for the
microbiota and carcinoma, a clearer framework is beginning to develop as
to how this occurs. The role of the microbiota in adenoma is still not
clear and part of the reason this may be is because many studies are not
powered effectively to observe the small changes reported here. It is
realistic to suspect that many changes in carcinoma could easily result
in effect sizes that are 30\% or more between the case and control. Most
of the studies analyzed have sufficient power to detect these type of
changes. In contrast, our data suggests that the adenoma effect size is
relatively small. None of the studies analyzed were properly powered to
detect a 10\% or lower change between case and controls and this may
well be the range in which differences consistently occur in adenoma.
Future studies investigating adenoma and the microbiota need to take
these factors into consideration if we are to work out the role of the
microbiota in adenoma formation.

\subsection{Conclusion}\label{conclusion}

By aggregating together a large collection of studies from both feces
and tissue we are able to provide information in support of the
importance of the bacterial community in both adenoma and carcinoma. We
are also able to provide support for the driver-passenger model in the
context of carcinoma. However, within the context of adenoma it is less
clear that this relationship exists. These observations highlight the
importance of power and sample number considerations when considering
investigations into the microbiota and adenoma due to the subtle changes
in the community.

\newpage

\subsection{Methods}\label{methods}

\textbf{\emph{Obtaining Data Sets:}} Studies used for this meta-analysis
were identified through the review articles written by Keku, et al. and
Vogtmann, et al. {[}26,27{]}. Additional studies not mentioned in the
reviews were obtained from the authors knowledge of the literature. All
studies were included that used tissue or feces as their sample source
for 16S rRNA gene sequencing analysis. Studies using either 454 or
Illumina sequencing technology were included. Only data sets that had
sequences available for analysis were included. Some studies did not
have publically available sequences or did not have meta data in which
the authors were able to share. After these filtering steps the
following studies remained: Ahn {[}21{]}, Baxter {[}24{]}, Brim
{[}17{]}, Burns {[}22{]}, Chen {[}14{]}, Dejea {[}19{]}, Flemer
{[}13{]}, Geng {[}25{]}, Hale {[}12{]}, Kostic {[}28{]}, Lu {[}16{]},
Sanapareddy {[}20{]}, Wang {[}15{]}, Weir {[}18{]}, and Zeller {[}23{]}.
The Zackular {[}29{]} study was not included becasue the 90 individuals
analyzed within the study are contained within the larger Baxter study.
The Kostic study was not used since after sequence processing all the
case samples did not have more than 100 sequences remaining. This left a
total of 14 studies in which complete analysis could be completed.

\textbf{\emph{Data Set Breakdown:}} In total there were 7 studies with
only fecal samples (Ahn, Baxter, Brim, Hale, Wang, Weir, and Zeller), 5
studies with only tissue samples (Burns, Dejea, Geng, Lu, Sanapareddy),
and 2 studies with both fecal and tissue samples (Chen and Flemer). The
total number of individuals that were analyzed after sequence processing
for feces was 1737 {[}Table 1{]}. The total number of matched and
unmatched tissue samples that were analyzed after sequence processing
was 492 {[}Table 2{]}.

\textbf{\emph{Sequence Processing:}} For the majority of studies raw
sequences were downloaded from the SRA
(\url{ftp://ftp-trace.ncbi.nih.gov/sra/sra-instant/reads/ByStudy/sra/SRP/})
and metadata was obtained from the following website:
\url{http://www.ncbi.nlm.nih.gov/Traces/study/} by searching the
respective accession number of the study. Of the studies that did not
have sequences and meta data on the SRA one study had the data stored on
DBGap {[}21{]} and for four studies the data was obtained directly from
the authors {[}12,13,18,20{]}. Each study was processed using the mothur
(v1.39.3) software program {[}30{]}. Where possible quality filtering
utilized the default methods used in mothur for either 454 or Illumina
based sequencing. If it was not possible to use these defaults the
author stated quality cut-offs were used instead. Chimeras were
identifed and removed using the VSEARCH {[}31{]} program and \emph{de
novo} OTU clustering at 97\% similarity using the OptiClust algorithm
{[}32{]} was utilized.

\textbf{\emph{Statistical Analysis:}} All statistical analysis after
sequence processing utilized the R software package (v3.4.2). For the
alpha diversity analysis values were power transformed using the
rcompanion (v1.10.1) package and then Z-score normalized using the car
(v2.1.5) package. Testing for alpha diversity differences utilized
linear mixed-effect models created using the lme4 (v1.1.14) package to
correct for study and variable region effects in feces and study,
variable region, and individual effects in tissue. Relative Risk was
analyzed using both the epiR (v0.9.87) and metafor (v2.0.0) packages.
Relative risk significance testing utilized the chi-squred test.
Beta-diversity differences utilized a Bray-Curtis distance matrix and
PERMANOVA executed with the vegan (v2.4.4) package. Random Forest models
were built using both the caret (v6.0.77) and randomForest (v4.6.12)
packages. Random Forest testing of the obtained AUC versus a random
model AUC utilized T-tests. Power analysis and estimations were made
using the pwr (v1.2.1) and statmod (v1.4.30) packages. All figures were
created using both ggplot2 (v2.2.1) and gridExtra (v2.3) packages.

\textbf{\emph{Study Analysis Overview:}} Alpha diversity was first
assessed for differences between controls, adenoma, and carcinoma. We
analyzed the data using linear mixed-effect models and relative risk.
Beta-diversity was then assessed for each inidividual study. Next, four
specific CRC-associated genera (\emph{Fusobacterium}, \emph{Parvimonas},
\emph{Peptostreptococcus}, and \emph{Porphyromonas}) were assessed for
differences in relative risk. We then built Random Forest models based
on all genera or the select CRC-associated genera. The models were
trained on one study then tested on the remaining studies for every
study. The data was split between feces and tissue samples. Within the
tissue groups the data was further divided between matched and unmatched
tissue samples. Where applicable for each study predictions for adenoma
and carcinoma were tested. This same approach was then applied at the
OTU level with the exception that instead of testing on the other
studies a 10-fold cross validation was utilized and 100 different models
were created based on random 80/20 splitting of the data to generate a
range of expected AUCs. For OTU based models the CRC associated genera
included all OTUs that had a taxonomic classification to
\emph{Fusobacterium}, \emph{Parvimonas}, \emph{Peptostreptococcus}, or
\emph{Porphyromonas}. The power of each study was assessed for an effect
size ranging from 1\% to 30\%. An estimated sample n for these effect
sizes was also generated based on 80\% power.

\textbf{\emph{Reproducible Methods:}} The code and analysis can be found
here
\url{https://github.com/SchlossLab/Sze_CRCMetaAnalysis_Microbiome_2017}.
Unless mentioned otherwise the accession number for the raw sequences
for the studies used in this analysis can be found directly in the
respective batch file, on the GitHub repository or in the original
manuscript.

\newpage

\subsection{Declarations}\label{declarations}

\subsubsection{Ethics approval and consent to
participate}\label{ethics-approval-and-consent-to-participate}

Ethics approval and informed consent for each of the studies used is
mentioned in the respective manuscript used in this meta-analysis.

\subsubsection{Consent for publication}\label{consent-for-publication}

Not applicable.

\subsubsection{Availability of data and
material}\label{availability-of-data-and-material}

A detailed and reporducible description of how the data were processed
and analyzed for each study can be found at
\url{https://github.com/SchlossLab/Sze_CRCMetaAnalysis_Microbiome_2017}.
Raw sequences can be downloaded from the SRA in most cases and can be
found in the respective studies batch file in the GitHub repo or within
the original publication. When sequences were not publicly available
contacting the corresponding author for raw sequences needs to be
undertaken.

\subsubsection{Competing Interests}\label{competing-interests}

All authors declare that they do not have any relevant competing
interests to report.

\subsubsection{Funding}\label{funding}

MAS is supported by a CIHR fellowship and a University of Michigan PTSP
fellowship grant.

\subsubsection{Authors' contributions}\label{authors-contributions}

All authors helped to design and conceptualize the study. MAS identified
and analyzed the data. MAS and PDS interpreted the data. MAS wrote the
first draft of the manuscript and both he and PDS reviewed and revised
updated versions. All authors approved the final manuscript.

\subsubsection{Acknowledgements}\label{acknowledgements}

The authors would like to thank all the study participants who were
apart of each of the individual studies uitlized. We would also like to
thank each of the study authors for making their data available for use.
Finally we would like to thank the members of the Schloss lab for
valuable feed back and proof reading during the formulation of this
manuscript.

\newpage

\subsection{References}\label{references}

\hypertarget{refs}{}
\hypertarget{ref-siegel_cancer_2016}{}
1. Siegel RL, Miller KD, Jemal A. Cancer statistics, 2016. CA: a cancer
journal for clinicians. 2016;66:7--30.

\hypertarget{ref-flynn_metabolic_2016}{}
2. Flynn KJ, Baxter NT, Schloss PD. Metabolic and Community Synergy of
Oral Bacteria in Colorectal Cancer. mSphere. 2016;1.

\hypertarget{ref-goodwin_polyamine_2011}{}
3. Goodwin AC, Destefano Shields CE, Wu S, Huso DL, Wu X, Murray-Stewart
TR, et al. Polyamine catabolism contributes to enterotoxigenic
Bacteroides fragilis-induced colon tumorigenesis. Proceedings of the
National Academy of Sciences of the United States of America.
2011;108:15354--9.

\hypertarget{ref-abed_fap2_2016}{}
4. Abed J, Emgård JEM, Zamir G, Faroja M, Almogy G, Grenov A, et al.
Fap2 Mediates Fusobacterium nucleatum Colorectal Adenocarcinoma
Enrichment by Binding to Tumor-Expressed Gal-GalNAc. Cell Host \&
Microbe. 2016;20:215--25.

\hypertarget{ref-arthur_intestinal_2012}{}
5. Arthur JC, Perez-Chanona E, Mühlbauer M, Tomkovich S, Uronis JM, Fan
T-J, et al. Intestinal inflammation targets cancer-inducing activity of
the microbiota. Science (New York, N.Y.). 2012;338:120--3.

\hypertarget{ref-kostic_fusobacterium_2013}{}
6. Kostic AD, Chun E, Robertson L, Glickman JN, Gallini CA, Michaud M,
et al. Fusobacterium nucleatum potentiates intestinal tumorigenesis and
modulates the tumor-immune microenvironment. Cell Host \& Microbe.
2013;14:207--15.

\hypertarget{ref-wu_human_2009}{}
7. Wu S, Rhee K-J, Albesiano E, Rabizadeh S, Wu X, Yen H-R, et al. A
human colonic commensal promotes colon tumorigenesis via activation of T
helper type 17 T cell responses. Nature Medicine. 2009;15:1016--22.

\hypertarget{ref-zackular_manipulation_2016}{}
8. Zackular JP, Baxter NT, Chen GY, Schloss PD. Manipulation of the Gut
Microbiota Reveals Role in Colon Tumorigenesis. mSphere. 2016;1.

\hypertarget{ref-zackular_gut_2013}{}
9. Zackular JP, Baxter NT, Iverson KD, Sadler WD, Petrosino JF, Chen GY,
et al. The gut microbiome modulates colon tumorigenesis. mBio.
2013;4:e00692--00613.

\hypertarget{ref-baxter_structure_2014}{}
10. Baxter NT, Zackular JP, Chen GY, Schloss PD. Structure of the gut
microbiome following colonization with human feces determines colonic
tumor burden. Microbiome. 2014;2:20.

\hypertarget{ref-shah_leveraging_2017}{}
11. Shah MS, DeSantis TZ, Weinmaier T, McMurdie PJ, Cope JL, Altrichter
A, et al. Leveraging sequence-based faecal microbial community survey
data to identify a composite biomarker for colorectal cancer. Gut. 2017;

\hypertarget{ref-hale_shifts_2017}{}
12. Hale VL, Chen J, Johnson S, Harrington SC, Yab TC, Smyrk TC, et al.
Shifts in the Fecal Microbiota Associated with Adenomatous Polyps.
Cancer Epidemiology, Biomarkers \& Prevention: A Publication of the
American Association for Cancer Research, Cosponsored by the American
Society of Preventive Oncology. 2017;26:85--94.

\hypertarget{ref-flemer_tumour-associated_2017}{}
13. Flemer B, Lynch DB, Brown JMR, Jeffery IB, Ryan FJ, Claesson MJ, et
al. Tumour-associated and non-tumour-associated microbiota in colorectal
cancer. Gut. 2017;66:633--43.

\hypertarget{ref-chen_human_2012}{}
14. Chen W, Liu F, Ling Z, Tong X, Xiang C. Human intestinal lumen and
mucosa-associated microbiota in patients with colorectal cancer. PloS
One. 2012;7:e39743.

\hypertarget{ref-wang_structural_2012}{}
15. Wang T, Cai G, Qiu Y, Fei N, Zhang M, Pang X, et al. Structural
segregation of gut microbiota between colorectal cancer patients and
healthy volunteers. The ISME journal. 2012;6:320--9.

\hypertarget{ref-lu_mucosal_2016}{}
16. Lu Y, Chen J, Zheng J, Hu G, Wang J, Huang C, et al. Mucosal
adherent bacterial dysbiosis in patients with colorectal adenomas.
Scientific Reports. 2016;6:26337.

\hypertarget{ref-brim_microbiome_2013}{}
17. Brim H, Yooseph S, Zoetendal EG, Lee E, Torralbo M, Laiyemo AO, et
al. Microbiome analysis of stool samples from African Americans with
colon polyps. PloS One. 2013;8:e81352.

\hypertarget{ref-weir_stool_2013}{}
18. Weir TL, Manter DK, Sheflin AM, Barnett BA, Heuberger AL, Ryan EP.
Stool microbiome and metabolome differences between colorectal cancer
patients and healthy adults. PloS One. 2013;8:e70803.

\hypertarget{ref-dejea_microbiota_2014}{}
19. Dejea CM, Wick EC, Hechenbleikner EM, White JR, Mark Welch JL,
Rossetti BJ, et al. Microbiota organization is a distinct feature of
proximal colorectal cancers. Proceedings of the National Academy of
Sciences of the United States of America. 2014;111:18321--6.

\hypertarget{ref-sanapareddy_increased_2012}{}
20. Sanapareddy N, Legge RM, Jovov B, McCoy A, Burcal L, Araujo-Perez F,
et al. Increased rectal microbial richness is associated with the
presence of colorectal adenomas in humans. The ISME journal.
2012;6:1858--68.

\hypertarget{ref-ahn_human_2013}{}
21. Ahn J, Sinha R, Pei Z, Dominianni C, Wu J, Shi J, et al. Human gut
microbiome and risk for colorectal cancer. Journal of the National
Cancer Institute. 2013;105:1907--11.

\hypertarget{ref-burns_virulence_2015}{}
22. Burns MB, Lynch J, Starr TK, Knights D, Blekhman R. Virulence genes
are a signature of the microbiome in the colorectal tumor
microenvironment. Genome Medicine. 2015;7:55.

\hypertarget{ref-zeller_potential_2014}{}
23. Zeller G, Tap J, Voigt AY, Sunagawa S, Kultima JR, Costea PI, et al.
Potential of fecal microbiota for early-stage detection of colorectal
cancer. Molecular Systems Biology. 2014;10:766.

\hypertarget{ref-baxter_microbiota-based_2016}{}
24. Baxter NT, Ruffin MT, Rogers MAM, Schloss PD. Microbiota-based model
improves the sensitivity of fecal immunochemical test for detecting
colonic lesions. Genome Medicine. 2016;8:37.

\hypertarget{ref-geng_diversified_2013}{}
25. Geng J, Fan H, Tang X, Zhai H, Zhang Z. Diversified pattern of the
human colorectal cancer microbiome. Gut Pathogens. 2013;5:2.

\hypertarget{ref-keku_gastrointestinal_2015}{}
26. Keku TO, Dulal S, Deveaux A, Jovov B, Han X. The gastrointestinal
microbiota and colorectal cancer. American Journal of Physiology -
Gastrointestinal and Liver Physiology {[}Internet{]}. 2015 {[}cited 2017
Oct 30{]};308:G351--63. Available from:
\url{http://ajpgi.physiology.org/lookup/doi/10.1152/ajpgi.00360.2012}

\hypertarget{ref-vogtmann_epidemiologic_2016}{}
27. Vogtmann E, Goedert JJ. Epidemiologic studies of the human
microbiome and cancer. British Journal of Cancer {[}Internet{]}. 2016
{[}cited 2017 Oct 30{]};114:237--42. Available from:
\url{http://www.nature.com/doifinder/10.1038/bjc.2015.465}

\hypertarget{ref-kostic_genomic_2012}{}
28. Kostic AD, Gevers D, Pedamallu CS, Michaud M, Duke F, Earl AM, et
al. Genomic analysis identifies association of Fusobacterium with
colorectal carcinoma. Genome Research. 2012;22:292--8.

\hypertarget{ref-zackular_human_2014}{}
29. Zackular JP, Rogers MAM, Ruffin MT, Schloss PD. The human gut
microbiome as a screening tool for colorectal cancer. Cancer Prevention
Research (Philadelphia, Pa.). 2014;7:1112--21.

\hypertarget{ref-schloss_introducing_2009}{}
30. Schloss PD, Westcott SL, Ryabin T, Hall JR, Hartmann M, Hollister
EB, et al. Introducing mothur: Open-Source, Platform-Independent,
Community-Supported Software for Describing and Comparing Microbial
Communities. Appl.Environ.Microbiol. {[}Internet{]}. 2009 {[}cited 12AD
Jan 1{]};75:7537--41. Available from:
\url{http://aem.asm.org/cgi/content/abstract/75/23/7537}

\hypertarget{ref-rognes_vsearch_2016}{}
31. Rognes T, Flouri T, Nichols B, Quince C, Mahé F. VSEARCH: A
versatile open source tool for metagenomics. PeerJ. 2016;4:e2584.

\hypertarget{ref-westcott_opticlust_2017}{}
32. Westcott SL, Schloss PD. OptiClust, an Improved Method for Assigning
Amplicon-Based Sequence Data to Operational Taxonomic Units. mSphere.
2017;2.

\newpage

\textbf{Table 1: Studies with Stool Samples Included in the Analysis}

\footnotesize

\begin{longtable}[]{@{}cccccc@{}}
\toprule
Study & Data Stored & 16S Region & Controls & Adenoma &
Carcinoma\tabularnewline
\midrule
\endhead
Ahn & DBGap & V3-4 & 148 & 0 & 62\tabularnewline
Baxter & SRA & V4 & 172 & 198 & 120\tabularnewline
Brim & SRA & V1-3 & 6 & 6 & 0\tabularnewline
Flemer & Author & V3-4 & 37 & 0 & 43\tabularnewline
Hale & Author & V3-5 & 473 & 214 & 17\tabularnewline
Wang & SRA & V3 & 56 & 0 & 46\tabularnewline
Weir & Author & V4 & 4 & 0 & 7\tabularnewline
Zeller & SRA & V4 & 50 & 37 & 41\tabularnewline
\bottomrule
\end{longtable}

\normalsize
\newpage

\textbf{Table 2: Studies with Tissue Samples Included in the Analysis}

\footnotesize

\begin{longtable}[]{@{}cccccc@{}}
\toprule
Study & Data Stored & 16S Region & Controls & Adenoma &
Carcinoma\tabularnewline
\midrule
\endhead
Burns & SRA & V5-6 & 18 & 0 & 16\tabularnewline
Chen & SRA & V1-V3 & 9 & 0 & 9\tabularnewline
Dejea & SRA & V3-5 & 31 & 0 & 32\tabularnewline
Flemer & Author & V3-4 & 103 & 37 & 94\tabularnewline
Geng & SRA & V1-2 & 16 & 0 & 16\tabularnewline
Lu & SRA & V3-4 & 20 & 20 & 0\tabularnewline
Sanapareddy & Author & V1-2 & 38 & 0 & 33\tabularnewline
\bottomrule
\end{longtable}

\normalsize
\newpage

\textbf{Figure 1: Alpha Diversity Differences between Control, Adenoma,
and Carcinoma Across Sampling Site.} A) Alpha diversity metric
differences by group in stool samples. B) Alpha diversity metric
differences by group in unmatched tissue samples. C) Alpha diversity
metric differences by group in matched tissue samples. The dashed line
represents a Z-score of 0 or no difference from the median.

\textbf{Figure 2: Relative Risk for Adenoma or Carcinoma based on Alpha
Diversity Metrics in Stool.} A) Alpha metric relative risk for adenoma.
B) Alpha metric relative risk for carcinoma. Colors represent the
different variable regions used within the respective study.

\textbf{Figure 3: Colorectal Cancer Associated Genera Relative Risk for
Adenoma and Carcinoma in Stool and Tissue.} A) Adenoma relative risk in
stool. B) Carinoma relative risk in stool. C) Adenoma relative risk in
tissue. D) Carcinoma relative risk in tissue. For all panels the
relative risk was also compared to whether one, two, three, or four of
the CRC associated genera were present.

\textbf{Figure 4: OTU Random Forest Model of Stool Across Studies.} A)
Adenoma random forest model between the full community and CRC
associated genera OTUs only. B) Carcinoma random forest model between
the full community and CRC associated genera OTUs only. The dotted line
represents an AUC of 0.5 and the lines represent the range in which the
AUC for the 100 different 80/20 runs fell between.

\textbf{Figure 5: OTU Random Forest Model of Tissue Across Studies.} A)
Adenoma random forest model between the full community and CRC
associated genera OTUs only. B) Carcinoma random forest model between
the full community and CRC associated genera OTUs only. The dotted line
represents an AUC of 0.5 and the lines represent the range in which the
AUC for the 100 different 80/20 runs fell between.

\textbf{Figure 6: Power and Effect Size Analysis of Studies Included.}
A) Power based on effect size for studies with Adenoma individuals. B)
Power based on effect size for studies with Carcinoma individuals. C)
The estimated sample number needed for each arm of each study to detect
aneffect size of 1-30\%. The dotted red lines in A) and B) represent the
generally used power of 0.8.

\newpage

\textbf{Figure S1: Relative Risk for Adenoma or Carcinoma based on Alpha
Diversity Metrics in Tissue.} A) Alpha metric relative risk for adenoma.
B) Alpha metric relative risk for carcinoma. Colors represent the
different variable regions used within the respective study.

\textbf{Figure S2: Random Forest Genus Model AUC for each Stool Study.}
A) AUC of Adenoma models using all genera or CRC associated genera only.
B) AUC of Carcinoma models using all genera or CRC associated genera
only. The black line represents the median within each group.

\textbf{Figure S3: Random Forest Genus Model AUC for each Tissue Study.}
A) AUC of Adenoma models using all genera or CRC associated genera only
divided between matched and unmatched tissue. B) AUC of Carcinoma models
using all genera or CRC associated genera only. The black line
represents the median within each group divided between matched and
unmatched tissue.

\textbf{Figure S4: Random Forest Prediction Success Using Genera for
each Stool Study.} A) AUC for prediction in Adenoma using all genera or
CRC associated genera only. B) AUC for prediction in Carcinoma using all
genera or CRC associated genera only. The dotted line represents an AUC
of 0.5. The x-axis is the data set in which the model was initially
trained on.

\textbf{Figure S5: Random Forest Prediction Success of Carcinoma Using
Genera for each Tissue Study.} A) AUC for prediction in unmatched tissue
for all genera or CRC associated genera only. B) AUC for prediction in
matched tissue using all genera or CRC associated genera only. The
dotted line represents an AUC of 0.5. The x-axis is the data set in
which the model was initially trained on.

\textbf{Figure S6: Random Forest Prediction Success of Adenoma Using
Genera for each Tissue Study.}

\newpage


\end{document}
