\documentclass[12pt,]{article}
\usepackage{lmodern}
\usepackage{amssymb,amsmath}
\usepackage{ifxetex,ifluatex}
\usepackage{fixltx2e} % provides \textsubscript
\ifnum 0\ifxetex 1\fi\ifluatex 1\fi=0 % if pdftex
  \usepackage[T1]{fontenc}
  \usepackage[utf8]{inputenc}
\else % if luatex or xelatex
  \ifxetex
    \usepackage{mathspec}
  \else
    \usepackage{fontspec}
  \fi
  \defaultfontfeatures{Ligatures=TeX,Scale=MatchLowercase}
\fi
% use upquote if available, for straight quotes in verbatim environments
\IfFileExists{upquote.sty}{\usepackage{upquote}}{}
% use microtype if available
\IfFileExists{microtype.sty}{%
\usepackage{microtype}
\UseMicrotypeSet[protrusion]{basicmath} % disable protrusion for tt fonts
}{}
\usepackage[margin=1.0in]{geometry}
\usepackage{hyperref}
\hypersetup{unicode=true,
            pdfborder={0 0 0},
            breaklinks=true}
\urlstyle{same}  % don't use monospace font for urls
\usepackage{longtable,booktabs}
\usepackage{graphicx,grffile}
\makeatletter
\def\maxwidth{\ifdim\Gin@nat@width>\linewidth\linewidth\else\Gin@nat@width\fi}
\def\maxheight{\ifdim\Gin@nat@height>\textheight\textheight\else\Gin@nat@height\fi}
\makeatother
% Scale images if necessary, so that they will not overflow the page
% margins by default, and it is still possible to overwrite the defaults
% using explicit options in \includegraphics[width, height, ...]{}
\setkeys{Gin}{width=\maxwidth,height=\maxheight,keepaspectratio}
\IfFileExists{parskip.sty}{%
\usepackage{parskip}
}{% else
\setlength{\parindent}{0pt}
\setlength{\parskip}{6pt plus 2pt minus 1pt}
}
\setlength{\emergencystretch}{3em}  % prevent overfull lines
\providecommand{\tightlist}{%
  \setlength{\itemsep}{0pt}\setlength{\parskip}{0pt}}
\setcounter{secnumdepth}{0}
% Redefines (sub)paragraphs to behave more like sections
\ifx\paragraph\undefined\else
\let\oldparagraph\paragraph
\renewcommand{\paragraph}[1]{\oldparagraph{#1}\mbox{}}
\fi
\ifx\subparagraph\undefined\else
\let\oldsubparagraph\subparagraph
\renewcommand{\subparagraph}[1]{\oldsubparagraph{#1}\mbox{}}
\fi

%%% Use protect on footnotes to avoid problems with footnotes in titles
\let\rmarkdownfootnote\footnote%
\def\footnote{\protect\rmarkdownfootnote}

%%% Change title format to be more compact
\usepackage{titling}

% Create subtitle command for use in maketitle
\newcommand{\subtitle}[1]{
  \posttitle{
    \begin{center}\large#1\end{center}
    }
}

\setlength{\droptitle}{-2em}
  \title{}
  \pretitle{\vspace{\droptitle}}
  \posttitle{}
  \author{}
  \preauthor{}\postauthor{}
  \date{}
  \predate{}\postdate{}

\usepackage{helvet} % Helvetica font
\renewcommand*\familydefault{\sfdefault} % Use the sans serif version of the font
\usepackage[T1]{fontenc}

\usepackage[none]{hyphenat}

\usepackage{setspace}
\doublespacing
\setlength{\parskip}{1em}

\usepackage{lineno}

\usepackage{pdfpages}

\usepackage{amsmath}

\usepackage{mathtools}

\begin{document}

\section{Making Sense of the Noise: Leveraging Existing 16S rRNA Gene
Surveys to Identify Key Community Members in Colorectal
Tumors}\label{making-sense-of-the-noise-leveraging-existing-16s-rrna-gene-surveys-to-identify-key-community-members-in-colorectal-tumors}

\begin{center}
\vspace{25mm}

Marc A Sze${^1}$ and Patrick D Schloss${^1}$${^\dagger}$

\vspace{20mm}

$\dagger$ To whom correspondence should be addressed: pschloss@umich.edu

$1$ Department of Microbiology and Immunology, University of Michigan, Ann Arbor, MI




\end{center}

Co-author e-mails:

\begin{itemize}
\tightlist
\item
  \href{mailto:marcsze@med.umich.edu}{\nolinkurl{marcsze@med.umich.edu}}
\end{itemize}

\newpage

\linenumbers

\subsection{Abstract}\label{abstract}

\textbf{Background.} An increasing body of literature suggests that both
individual and collections of bacteria are associated with the
progression of colorectal cancer. As the number of studies investigating
these associations increases and the number of subjects in each study
increases, a meta-analysis to identify the associations that are the
most predictive of disease progression is warranted. For our
meta-analysis, we analyzed previously published 16S rRNA gene sequencing
data collected from feces (1737 individuals from 8 studies) and colon
tissue (492 total samples from 350 individuals from 7 studies).

\textbf{Results.} We quantified the odds ratios for individual bacterial
genera that were associated with an individual having tumors relative to
a normal colon. Among the stool samples, there were no genera that had a
significant odds ratio associated with adenoma and there were 8 genera
with significant odds ratios associated with carcinoma. Similarly, among
the tissue samples, there were no genera that had a significant odds
ratio associated with adenoma and there were 3 genera with significant
odds ratios associated with carcinoma. Among the significant odds
ratios, the association between individual taxa and tumor diagnosis was
equal or below 7.11. Because individual taxa had limited association
with tumor diagnosis, we trained Random Forest classification models
using the genera with the five highest and lowest odds ratios, using the
entire collection of genera found in each study, and using operational
taxonomic units defined based on a 97\% similarity threshold. All
training approaches yielded similar classification success as measured
using the Area Under the Curve. The ability to correctly classify
individuals with adenomas was poor and the ability to classify
individuals with carcinomas was considerably better using sequences from
stool or tissue.

\textbf{Conclusions.} This meta-analysis confirms previous results
indicating that individuals with adenomas cannot be readily classified
based on their bacterial community, but that those with carcinomas can.
Regardless of the dataset, we found a subset of the fecal community that
was associated with carcinomas was as predictive as the full community.

\subsubsection{Keywords}\label{keywords}

microbiota; colorectal cancer; polyps; adenoma; tumor; meta-analysis.

\newpage

\subsection{Background}\label{background}

Colorectal cancer (CRC) is a growing world-wide health problem in which
the microbiota has been hypothesized to have a role in disease
progression {[}1,2{]}. Numerous studies using murine models of CRC have
shown the importance of both individual microbes {[}3--7{]} and the
overall community {[}8--10{]} in tumorigenesis. Numerous case-control
studies have characterized the microbiota of individuals with colonic
adenomas and carcinomas in an attempt to identify biomarkers of disease
progression {[}6,11--17{]}. Because current CRC screening
recommendations are poorly adhered to due to socioeconomic status, test
invasiveness, and frequency of tests, development and validation of
microbiome-associated biomarkers for CRC progression could further
attempts to develop non-invasive diagnostics {[}18{]}.

Recently, there has been an intense focus on identifying
microbiota-based biomarker yielding a seemingly endless number of
candidate taxa. Some studies point towards mouth-associated genera such
as Fusobacterium, Peptostreptococcus, Parvimonas, and Porphyromonas that
are enriched in people with carcinomas {[}6,11--17{]}. Other studies
have identified members of \emph{Akkermansia}, \emph{Bacteroides},
\emph{Enterococcus}, \emph{Escherichia}, \emph{Klebsiella},
\emph{Mogibacterium}, \emph{Streptococcus}, and \emph{Providencia} are
also associated with carciomas {[}13--15{]}. Additionally,
\emph{Roseburia} has been found in some studies to be more abundant in
people with tumors but in other studies it has been found to be either
less abundant or no different than what is found in subjects with normal
colons {[}14,17,19,20{]}. There are strong results from tissue culture
and murine models that Fusobacterium nucleatum, pks-positive strains of
Escherichia coli, Streptococcus gallolyticus, and an
entertoxin-producing strain of Bacteroides fragilis are important in the
pathogenesis of CRC {[}5,14,21--24{]}. These results point to a
causative role for the microbiota in CRC pathogenesis as well as their
potential as diagnostic biomarkers.

Most studies have focused on identifying biomarkers in patients with
carcinomas but there is a greater clinical need to identify biomarkers
associated with adenomas. Studies focusing on broad scale community
metrics have found that measures such as the total number of Operational
Taxonomic Units (OTUs) are decreased in those with adenomas versus
controls {[}25{]}. Other studies have identified \emph{Acidovorax},
\emph{Bilophila}, \emph{Cloacibacterium}, \emph{Desulfovibrio},
\emph{Helicobacter}, \emph{Lactobacillus}, \emph{Lactococcus},
\emph{Mogibacterium}, and \emph{Pseudomonas} to be enriched in those
with adenomas {[}25--27{]}. There are few genera that are enriched in
patients with adenoma or carcinoma tumors.

Confirming some of these previous findings, a recent meta-analysis found
that 16S rRNA gene sequences from members of the \emph{Akkermansia},
\emph{Fusobacterium}, and \emph{Parvimonas} were fecal biomarkers for
the presence of carcinomas {[}28{]}. Contrary to previous studies they
found sequences similar to members of \emph{Lactobacillus} and
\emph{Ruminococcus} to be enriched in patients with adenoma or carcinoma
relative to those with normal colons {[}12,15,16{]}. In addition, they
found 16S rRNA gene sequences from members of \emph{Haemophilus},
\emph{Methanosphaera}, \emph{Prevotella}, \emph{Succinovibrio} were
enriched in patients with adenoma and \emph{Pantoea} were enriched in
patients with carcinomas. Although this meta-analysis was helpful for
distilling a large number of possible biomarkers, the aggregate number
of samples included in the analysis (n = 509) was smaller than several
larger case-control studies that have since been published {[}12,27{]}

Here we provide an updated meta-analysis using 16S rRNA gene sequence
data from both feces (n = 1737) and colon tissue (492 samples from 350
individuals) from 14 studies {[}11--17,19,20,23,25--27,29{]} {[}Table 1
\& 2{]}. We expand both the breadth and scope of the previous
meta-analysis to investigate whether biomarkers describing the bacterial
community or specific members of the community can more accurately
classify patients as having adenoma or carcinoma. Our results suggest
that the bacterial community changes as disease severity worsens and
that that a subset of the microbial community can be used to diagnose
the presence of carcinoma.

\newpage

\subsection{Results}\label{results}

\textbf{\emph{Lower Bacterial Diversity is Associated with Increased
Odds Ratio (OR) of Tumors:}} We first assessed whether variation in
broad community metrics like total number of operational taxonomic units
(OTUs) (i.e.~richness), the evenness of their abundance, and the overall
diversity was associated with disease stage after controlling for study
and variable region differences. In stool, there was a significant
decrease in both evenness and diversity as disease severity progressed
from normal to adenoma to carcinoma (P-value = 0.025 and 0.043,
respectively) {[}Figure 1{]}; there was not a significant difference for
richness (P-value = 0.21). We next tested whether the decrease in these
community metrics translated into significant ORs for having an adenoma
or carcinoma. For fecal samples, the ORs for richness were not
significantly greater than 1.0 for adenoma or carcinoma (P-value = 0.40)
{[}Figure 2A{]}. The ORs for evenness were significantly higher than 1.0
for adenoma (OR = 1.3 (1.02 - 1.65), P-value = 0.035) and carcinoma (OR
= 1.66 (1.2 - 2.3), P-value = 0.0021) {[}Figure 2B{]}. The ORs for
diversity were only significantly greater than 1.0 for carcinoma (OR =
1.61 (1.14 - 2.28), P-value = 0.0069), but not for adenoma (P-value =
0.11) {[}Figure 2C{]}. Although these OR are significantly greater than
1.0, it is doubtful that these are clinically meaningful values.

Similar to our analysis of sequences obtained from stool samples, we
repeated the analysis using sequences obtained from colon tissue. There
were no significant changes in richness, evenness, or diversity as
disease severity progressed from control to adenoma to carcinoma
(P-value \textgreater{} 0.05). We next analyzed the OR, for matched
(i.e.~where unaffected tissue and tumors were obtained from the same
individual) and unmatched (i.e.~where unaffected tissue and tumor tissue
were not obtained from the same individual) tissue samples. The ORs for
adenoma and carcinoma by any measure were not significantly different
from 1.0 (P-value \textgreater{} 0.05) {[}Figure S1 \& Table S1{]}. This
is likely due to the combination of a small effect size, as suggested
from the results using stool, and the relatively small number of studies
and size of studies used in the analysis.

\textbf{\emph{Disease Progression is Associated with Community-Wide
Changes in Composition and Abundance:}} Based on the differences in
evenness and diversity, we next asked whether there were community-wide
differences in the structure of the communities associated with
different disease stages. We identified significant bacterial community
differences in the stool of patients with adenomas relative to those
with normal colons in 1 of 4 studies and in patients with carcinomas
relative to those with normal colons in 6 of 7 studies (PERMANOVA;
P-value \textless{} 0.05) {[}Table S2{]}. Similar to the analyses using
stool samples, there were significant differences in bacterial community
structure between subjects with normal colons and those with adenoma (1
of 2 studies) and carcinoma (1 of 3 studies) {[}Table S2{]}. For studies
that used matched samples no differences in bacterial community
structures were observed {[}Table S2{]}. Combined, these results
indicate that there consistent and significant community-wide changes in
the fecal community structure of subjects with carcinomas. However, the
signal observed in subjects with adenomas or when using tissue samples
was not as consistent. This is likely due to a smaller effect size or
the relatively small sample sizes among the studies that characterized
the tissue microbiota.

\textbf{\emph{Individual Taxa are Associated with Significant ORs for
Carcinomas:}} Next we identified those taxa were associated with ORs
that were significantly associated with having a normal colon or the
presence of adenomas or carcinomas. No taxa had a significant OR for the
presence of adenomas when we used data collected from stool or tissue
samples (Table S3 \& S4). In contrast, 8 taxa had significant ORs for
the presence of carcinomas using data from stool samples. Of these, 4
are commonly associated with the oral cavity: Fusobacterium (OR = 2.74
(1.95 - 3.85)), Parvimonas (OR = 3.07 (2.11 - 4.46)), Porphyromonas (OR
= 3.2 (2.26 - 4.54)), and Peptostreptococcus (OR = 7.11 (3.84 - 13.17))
{[}Table S3{]}. The other 4 were Clostridium XI (OR = 0.65 (0.49 -
0.86)), Enterobacteriaceae (OR = 1.79 (1.33 - 2.41)), Escherichia (OR =
2.15 (1.57 - 2.95)), and Ruminococcus (OR = 0.63 (0.48 - 0.83)). Among
the data collected from tissue samples, only unmatched carcinoma samples
had taxa with a significant OR. Those included Dorea (OR = 0.35 (0.22 -
0.55)), Blautia (OR = 0.47 (0.3 - 0.73)), and Weissella (OR = 5.15 (2.02
- 13.14)). Mouth-associated genera were not significantly associated
with an increased OR for carcinoma in tissue samples {[}Table S4{]}. For
example, Fusobacterium had an OR of 3.98 (1.19 - 13.24; however, due to
the small number of studies and considerable variation in the data, the
Benjimani-Hochberg-corrected P-value was 0.93 {[}Table S4{]}. It is
interesting to note that Ruminococcus and members of Clostridium group
XI in stool and Dorea and Blautia in tissue had ORs that were
significantly less than 1.0, which suggests that these populations are
protective against the development of carcinomas. Overall, there was no
overlap in the taxa with significant OR between stool and tissue
samples.

\textbf{\emph{Individual Significant OR Taxa Classify Carcinoma
Poorly:}} Since specific taxa were associated with a significant OR for
carcinoma we tested whether they would also be good classifiers of
carcinoma. For stool, the 8 significant taxa did no better at
classifying those with normal colons versus those with carcinomas than
chance {[}Figure 3A{]}. Likewise in unmathced tissue samples the 3
significant taxa were no better than an AUC of 0.5 {[}Figure 3B{]}.
These results suggest that although these taxa are significantly
associated with a decreased or increased OR of carcinoma, individually
they are poor classifiers of disease.

\textbf{\emph{Select Community Models can Recapitulate Whole Community
Models:}} Since specific taxa increased or decreased the OR for
carcinoma but performed poorly as an individual classifier of carcinoma
and assessing ORs for adenoma we assessed whether the overall bacterial
community was better at classifying disease. Three models were tested
and included a full taxa model, full OTU model, and significant OR taxa
model. If no taxa were significant after multiple comparison correction
then no model for that specific grouping (i.e.~adenoma stool) was
analyzed. We first tested three model AUCs. Next, the all genera models
and any significant OR taxa models were tested across all studies that
were not used to train the model.

For stool, all models used had similar AUCs for both adenoma and
carcinoma {[}Figure 4{]}. However, the adenoma AUCs were barely better
than chance {[}Figure 4A{]}. Similarily, when analyzing the tests sets
that were comprised of genera data from other studies the adenoma models
performed poorly {[}Figure 5A{]}. The carcinoma models performed much
better, with both the all genera and singificant OR taxa only models
having a similar ability to detect individuals with carcinomas {[}Figure
5B{]}. The most common genera in the top 10 most important variables, in
the full genera models used to classify carcinomas, were
\emph{Bacteroides} and \emph{Lachnospiraceae} {[}Figure 6A{]}.
Regardless of sample type, mouth-associated genera were present in
models for carcinomas but not consistently across studies {[}Figure
6{]}. For the full community OTU-based models, both \emph{Bacteroides},
\emph{Blautia}, \emph{Lachnospiraceae}, and \emph{Ruminococcaceae} were
present in the top 10 consistently across studies {[}Figure 6B{]}.
Overall, these results suggest that multiple microbes could act as the
inflammatory stimulus needed to exacerbate mutations leading each
individual microbe as a poor individual classifier but much better in
aggregate with others.

The tissue-based models had results that were dependent on genera or OTU
level data and matched or unmatched samples {[}Figure S2{]}. Similarily
to stool, the significant OR taxa performed as well as both the full
taxa and OTU models {[}Figure S3B{]}. When analyzing the tests sets that
were comprised of genera data from other studies, all models performed
poorly with one exception being carcinoma classification in unmatched
tissue {[}Figure S4{]}. Unlike stool, the test sets for the significant
OR taxa did not perform as well as the all genera-based models {[}Figure
S4C{]}. For the full genera models built using either matched or
unmatched samples, showed no consistent taxa representation in the top
10 most important model variables across study {[}Figure S4A \& S4B{]}.
Conversely, the OTU-based models built using either matched or unmatched
samples showed more commonalities {[}Figure S4C \& S4D{]}. In models
built using matched samples \emph{Lachnospiraceae},
\emph{Fusobacteriaceae}, \emph{Comamonadaceae}, and
\emph{Bacteroidaceae} appeared in the top 10 percent in the majority of
studies {[}Figure S4C{]}. Although models built with unmatched samples
also had \emph{Lachnospiraceae} and \emph{Bacteroidaceae}, a major
difference was the presence of \emph{Ruminococcaceae} in the top 10 of
every study {[}Figure S4D{]}. This results suggests that either the
colon tissue microbiota is study or person dependent or that that kit
and/or other types of contamination associated with low biomass samples
may be skewing the results.

\newpage

\subsection{Discussion}\label{discussion}

\emph{Targeting the identification of tumor microbial biomarkers within
stool seems logical since it offers an easy and cost-effective way to
stratify risk of disease. The current gold standard for diagnosis, a
colonoscopy, can be time-consuming and is not without risk of
complications. Although stool represents an easy and less invasive way
to assess risk, it is not clear how well this sample reflects adenoma-
and carcinoma- associated microbial communities. Some studies have tried
to assess this in health and disease but are limited by their sample
size {[}17,30{]}. Sampling the microbiota directly associated with colon
tissue may provide clearer answers but is not without their own
limitations. After the colonoscopy bowel prep the bacterial community
sampled may reflect the better adhered microbiota versus the resident
community. Additionally, these samples contain more host DNA,
potentially limiting the types of analysis that can be done. It is well
known that low biomass samples can be very difficult to work with and
results can be study dependent due to the randomness of contamination
{[}31{]}.}

Our study identifies clear but small differences in diversity at the
community level and larger differences for individual genera, present in
patients with tumors versus controls {[}Figure 1-3{]}. Although there
was a step-wise decrease in diversity as disease progressed from control
to adenoma to carcinoma, this did not translate into large effect size
increases in OR for either adenoma or carcinoma tumors. Even though
mouth-associated genera increased individaul's OR of having a carcinoma
for certain sample types, they did not consistently increase the OR of
having an adenoma. By using these taxa that had significant ORs after
multiple comparison correction we found that we could classify
indviduals with either adenoma or carcinoma as well as models that use
either all genera or all OTUs. Finally, many studies were individually
under powered to be able to reject the null hypothesis and this could
one reason only the comparison between control and carcinoma individuals
for stool samples had relible detectable differences.

The data presented herein support the importance of specific taxa for
carcinoma, but not necessarily adenoma, tumor formation. The results
that we have presented show that the significant OR taxa model and both
the full genera and OTU models, for indviduals with carcinoma, had
similar AUCs {[}Figure 2 \& 3{]}. This suggests that an interplay
between a select number of potentially protective and exacerbating
microbes within the GI community could be crucial for carcinoma
formation. Importantly, it suggests that there may be key members of the
GI community that should be studied further to potentially help reduce
the risk of carcinoma tumor formation. Conversely, using the present
data, it is clear that new approaches may be needed to identify members
of the community associated with adenoma tumors. Regardless of sample
type and whether a full genera- or OTU-based model was used, our Random
Forest models consistently performed poorly. Yet, the step-wise decrease
in diversity suggests that the adenoma-associated community is not
normal but has changed subtly. This change in diversity, at this early
stage of disease, could be focal to the adenoma itself. How the host
interacts with these subtle changes at early stages of the disease could
be what leads to a thoroughly dysfunctional community that is supportive
of tumorgenesis.

For the full genera- and OTU-based models within stool, common GI
microbes were most consistently present in the top 10 genera or OTUs
across studies {[}Figure 4{]}. Changes in \emph{Bacteroides},
\emph{Ruminococcaceae}, \emph{Ruminococcus}, and \emph{Roseburia} were
consistently found to be in the top 10 most important variables across
the different studies for both individuals with adenoma and carcinoma
{[}Figure 4{]}. These data suggest that whether the non-resident
bacterium is \emph{Fusobacteria} or \emph{Peptostreptococcus} may not be
as important as how these bacteria interact with the changing resident
community. Based on these observations, it is possible to hypothesize
that small changes in community structure lead to new niches in which
any one of the mouth-associated or general inflammatory genera can gain
a foothold, exacerbating the initial changes in community and
facilitating the transition from adenoma to carcinoma stage of disease.

The colon tissue-based studies did not provide a clearer understanding
of how the microbiota may be associated with tumors. Generally, the full
OTU-based models of unmatched and matched colon tissue samples were
concordant with stool samples showing that GI resident microbes were the
most prevalent in the top 10 most important variables across study
{[}Figure S4E \& F{]}. Unlike in stool, \emph{Fusobacterium} was the
only mouth-associated bacteria consistently present in the top 10 most
important variables of the full carcinoma stage models {[}Figure S4B-C
\& E-F{]}. The majority of the colon tissue-based results seem to be
study specific with many of the top 10 taxa being present only in a
single study. Additionally, the presence of genera associated with
contamination, within the top 10 most important variables for the genera
and OTU models is worrying. The low bacterial biomass of tissue samples
coupled with potential contamination could explain why these results
seem to be more sporadic than the stool results.

One important caveat to this study is that even though genera associated
with certain species such as \emph{Bacteroides fragilis} and
\emph{Streptococcus gallolyticus} subsp. \emph{gallolyticus} were not
identified, it does not necessarily mean that these specific species are
not important in human CRC {[}22,24{]}. Since we are limited in our
aggregation of the data to the genus level, it is not possible to
clearly delineate which species are contributing to overall disease
progression. Our observations are not inconsistent with the previous
literature on either \emph{Bacteroides fragilis} or \emph{Streptococcus
gallolyticus} subsp. \emph{gallolyticus}. As an example, the stool-based
full community models consistently identified the genus
\emph{Bacteroides}, as well as OTUs that classified as
\emph{Bacteroides}, to be important model components across studies.
This suggests that even though \emph{Bacteroides} may not increase the
OR of individuals having an adenoma or carcinoma and may not vary in
relative abundance, like \emph{Fusobacterium}, it is still important in
CRC. Additionally, \emph{Streptococcus gallolyticus} subsp.
\emph{gallolyticus} is a mouth-associated microbe, and the results from
this study suggest that regardless of sample type, mouth-associated
genera are commonly associated with an increased OR for individuals to
have a carcinoma tumor.

The associations between the microbiota and individuals with adenoma
tumors are inconclusive, in part, because many studies may not be
powered effectively to observe small effect sizes. None of the studies
analyzed were properly powered to detect a 10\% or lower change between
cases and controls. The results within our meta-analysis suggest that a
small effect size may well be the scope in which differences
consistently occur between controls and those with adenomas. Future
studies investigating adenoma tumors and the microbiota need to take
power into consideration to reproducibly study whether the microbiota
contributes to adenoma formation. In contrast to adenoma stage of
disease, our observations suggest that most studies analyzed have
sufficient power to detect many changes in the carcinoma-associated
microbiota because of large effect size differences between cases and
controls {[}Figure 5{]}.

\newpage

\subsection{Conclusion}\label{conclusion}

By aggregating together a large collection of studies analyzing both
fecal and colon tissue samples, we are able to provide evidence
supporting the importance of the bacterial community in colorectal
tumors. The data presented here suggests that mouth-associated microbes
can gain a foothold within the colon and are commonly associated with
the greatest OR of individuals having a carcinoma. Conversely, no
conclusive signal with these mouth-associated microbes could be detected
for individuals with an adenoma. Our observations also highlight the
importance of power and sample number considerations when investigating
the microbiota and adenoma tumors due to possible subtle changes in the
community. Overall, the associations between the microbiota and
individuals with carcinomas were much stronger than with those with
adenomas.

\newpage

\subsection{Methods}\label{methods}

\textbf{\emph{Obtaining Data Sets:}} The studies used for this
meta-analysis were identified through the review articles written by
Keku, \emph{et al.} and Vogtmann, \emph{et al.} {[}32,33{]} and
additional studies not mentioned in the reviews were obtained based on
the authors' knowledge of the literature. Studies that used tissue or
feces as their sample source for 454 or Illumina 16S rRNA gene
sequencing analysis and had data sets with sequences available for
analysis were included. Some studies were excluded because they did not
have publicly available sequences or did not have metadata in which the
authors were able to share. After these filtering steps, the following
studies remained: Ahn, \emph{et al.} {[}11{]}, Baxter, \emph{et al.}
{[}12{]}, Brim, \emph{et al.} {[}29{]}, Burns, \emph{et al.} {[}15{]},
Chen, \emph{et al.} {[}13{]}, Dejea, \emph{et al.} {[}20{]}, Flemer,
\emph{et al.} {[}17{]}, Geng, \emph{et al.} {[}19{]}, Hale, \emph{et
al.} {[}27{]}, Kostic, \emph{et al.} {[}34{]}, Lu, \emph{et al.}
{[}26{]}, Sanapareddy, \emph{et al.} {[}25{]}, Wang, \emph{et al.}
{[}14{]}, Weir, \emph{et al.} {[}23{]}, and Zeller, \emph{et al.}
{[}16{]}. The Zackular {[}35{]} study was not included because the 90
individuals analyzed within the study are contained within the larger
Baxter study {[}12{]}. After sequence processing, all the case samples
for the Kostic study had 100 or less sequences remaining and was
excluded, leaving a total of 14 studies that analysis could be completed
on.

\textbf{\emph{Data Set Breakdown:}} In total, there were seven studies
with only fecal samples (Ahn, Baxter, Brim, Hale, Wang, Weir, and
Zeller), five studies with only tissue samples (Burns, Dejea, Geng, Lu,
Sanapareddy), and two studies with both fecal and tissue samples (Chen
and Flemer). The total number of individuals analyzed after sequence
processing for feces was 1737 {[}Table 1{]}. The total number of matched
and unmatched tissue samples that were analyzed after sequence
processing was 492 {[}Table 2{]}.

\textbf{\emph{Sequence Processing:}} For the majority of studies, raw
sequences were downloaded from the Sequence Read Archive (SRA)
(\url{ftp://ftp-trace.ncbi.nih.gov/sra/sra-instant/reads/ByStudy/sra/SRP/})
and metadata were obtained by searching the respective accession number
of the study at the following website:
\url{http://www.ncbi.nlm.nih.gov/Traces/study/}. Of the studies that did
not have sequences and metadata on the SRA, data was obtained from DBGap
(n = 1, {[}11{]}) and directly from the authors (n = 4,
{[}17,23,25,27{]}). Each study was processed using the mothur (v1.39.3)
software program {[}36{]} and quality filtering utilized the default
methods for both 454 and Illumina based sequencing. If it was not
possible to use the defaults, the stated quality cut-offs, from the
study itself, were used instead. Sequences that were made up of an
artificial combination of two or more different sequences and commonly
known as chimeras were identified and removed using VSEARCH {[}37{]}
before \emph{de novo} OTU clustering at 97\% similarity was completed
using the OptiClust algorithm {[}38{]}.

\textbf{\emph{Study Analysis Overview:}} OTU richness, evenness, and
Shannon diversity were first assessed for differences between controls,
adenoma tumors, and carcinoma turmors using both linear mixed-effect
models and ORs. For each individual study the Bray-Curtis index was used
to assess differences between control-adenoma and control-carcinoma
individuals. Next, all common genera were assessed for differences in
ORs for individuals having an adenoma or carcinoma and corrected for
multiple comparisons using the Benjamini-Hochberg method {[}39{]}. We
then built Random Forest models based on all genera, all OTUs, or
significant OR taxa (only using taxa still significant after multiple
comparison correction). For both the full genera and significant OR
taxa, models were trained on one study then tested on the remaining
studies using genera-based relative abundances. The OTU-based models
were built using OTU level data and a 10-fold CV over 100 different
iterations, based on random 80/20 splitting of the data, was used to
generate a range of expected AUCs. This process was repeated for every
study in the meta-analysis. Comparisons of the initial trained model
AUCs for the full genera and significant OR taxa were made to the mean
AUC generated from the 100 different 10-fold CV runs of the respective
OTU-based model. For comparisons in which only control versus adenoma
individuals were made, the carcinoma individuals were excluded from each
respective study. Similarly, for comparisons in which control versus
carcinoma individuals were made the adenoma individuals were excluded
from each respective study. For all analysis completed fecal and tissue
samples were kept separate. Within the tissue groups the data were
further divided between samples from the same individual (matched) and
those from different individuals (unmatched).

\textbf{\emph{Obtaining Genera Relative Abundance and Significant OR
Taxa Models:}} For the genera analysis of the ORs, OTUs were added
together based on the genus or lowest available taxonomic classification
level and the total average counts, for 100 different subsamplings was
obtained. The significant OR taxa models for the Random Forest models
utilized all taxa that had significant ORs after multiple comparison
correction. This meant only models for the carcinoma stool (8 variables)
and carcinoma unmatched (3 variables) samples were possible to be
created and analyzed.

\textbf{\emph{Matched versus Unmatched Tissue Samples:}} In general,
tissue samples with control and tumor samples from different individuals
were classified as unmatched while samples that belonged to the same
individual were classified as matched. Studies with matched data
included Burns, Dejea, Geng, and Lu while those with unmatched data were
from Burns, Flemer, Chen, and Sanapareddy. For some studies samples
became unmatched when a corresponding matched sample did not make it
through sequence processing. All samples, from both matched and
unmatched tissue samples, were analyzed together for the linear
mixed-effect models with samples from the same individual being
corrected for. All other analysis, where it is not specified explicitly,
matched and unmatched samples were analyzed separately using the
statistical approaches mentioned in the Statistical Analysis section.

\textbf{\emph{Assessing Important Random Forest Model Variables:}} Using
Mean Decrease in Accuracy (MDA) the top 10 most important variables to
the Random Forest model were obtained for the full models of the two
different approaches used. For the first approach utilizing genus-based
models, the number of times that a specific taxa showed up in the top 10
of the training set across each study was counted. For the second
approach, that utilized the OTU-based models, the medians for each OTU
across 100 different 80/20 splits of the data was generated and the top
10 OTUs then counted for each study. Common taxa were then identified by
using the lowest classification for each of the specific OTUs obtained
from these counts and the number of times this classification occurred
across the top 10 of each study was recorded. Finally, the two studies
that had adenoma tumor tissue (Lu and Flemer) were equally divided
between matched and unmatched studies and were grouped together for the
counting of the top 10 genera and OTUs for both Random Forest
approaches.

\textbf{\emph{Statistical Analysis:}} All statistical analysis after
sequence processing utilized the R (v3.4.3) software package {[}40{]}.
For OTU richness, evenness, and Shannon diversity analysis, values were
power transformed using the rcompanion (v1.11.1) package {[}41{]} and
then Z-score normalized using the car (v2.1.6) package {[}42{]}. Testing
for OTU richness, evenness, and Shannon diversity differences utilized
linear mixed-effect models created using the lme4 (v1.1.15) package
{[}43{]} to correct for study, repeat sampling of individuals (tissue
only), and 16S hyper-variable region used. Odds ratios (OR) were
analyzed using both the epiR (v0.9.93) and metafor (v2.0.0) packages
{[}44,45{]} by assessing how many individuals with and without disease
were above and below the overall median value within each specific
study. OR significance testing utilized the chi-squared test. Diversity
differences measured by the Bray-Curtis index utilized the creation of
distance matrix and testing with PERMANOVA executed with the vegan
(v2.4.5) package {[}46{]}. Random Forest models were built using both
the caret (v6.0.78) and randomForest (v4.6.12) packages {[}47,48{]}. All
figures were created using both ggplot2 (v2.2.1) and gridExtra (v2.3)
packages {[}49,50{]}.

\textbf{\emph{Reproducible Methods:}} The code and analysis can be found
at
\url{https://github.com/SchlossLab/Sze_CRCMetaAnalysis_Microbiome_2017}.
Unless otherwise mentioned, the accession number of raw sequences from
the studies used in this analysis can be found directly in the
respective batch file in the GitHub repository or in the original
manuscript.

\newpage

\subsection{Declarations}\label{declarations}

\subsubsection{Ethics approval and consent to
participate}\label{ethics-approval-and-consent-to-participate}

Ethics approval and informed consent for each of the studies used is
mentioned in the respective manuscripts used in this meta-analysis.

\subsubsection{Consent for publication}\label{consent-for-publication}

Not applicable.

\subsubsection{Availability of data and
material}\label{availability-of-data-and-material}

A detailed and reproducible description of how the data were processed
and analyzed for each study can be found at
\url{https://github.com/SchlossLab/Sze_CRCMetaAnalysis_Microbiome_2017}.
Raw sequences can be downloaded from the SRA in most cases and can be
found in the respective study batch file in the GitHub repository or
within the original publication. For instances when sequences are not
publicly available, they may be accessed by contacting the corresponding
authors from whence the data came.

\subsubsection{Competing Interests}\label{competing-interests}

All authors declare that they do not have any relevant competing
interests to report.

\subsubsection{Funding}\label{funding}

MAS is supported by a Canadian Institute of Health Research fellowship
and a University of Michigan Postdoctoral Translational Scholar Program
grant.

\subsubsection{Authors' contributions}\label{authors-contributions}

All authors helped to design and conceptualize the study. MAS identified
and analyzed the data. MAS and PDS interpreted the data. MAS wrote the
first draft of the manuscript and both he and PDS reviewed and revised
updated versions. All authors approved the final manuscript.

\subsubsection{Acknowledgements}\label{acknowledgements}

The authors would like to thank all the study participants who were a
part of each of the individual studies utilized. We would also like to
thank each of the study authors for making their data available for use.
Finally, we would like to thank the members of the Schloss lab for
valuable feed back and proof reading during the formulation of this
manuscript.

\newpage

\subsection{References}\label{references}

\hypertarget{refs}{}
\hypertarget{ref-siegel_cancer_2016}{}
1. Siegel RL, Miller KD, Jemal A. Cancer statistics, 2016. CA: a cancer
journal for clinicians. 2016;66:7--30.

\hypertarget{ref-flynn_metabolic_2016}{}
2. Flynn KJ, Baxter NT, Schloss PD. Metabolic and Community Synergy of
Oral Bacteria in Colorectal Cancer. mSphere. 2016;1.

\hypertarget{ref-goodwin_polyamine_2011}{}
3. Goodwin AC, Destefano Shields CE, Wu S, Huso DL, Wu X, Murray-Stewart
TR, et al. Polyamine catabolism contributes to enterotoxigenic
Bacteroides fragilis-induced colon tumorigenesis. Proceedings of the
National Academy of Sciences of the United States of America.
2011;108:15354--9.

\hypertarget{ref-abed_fap2_2016}{}
4. Abed J, Emgård JEM, Zamir G, Faroja M, Almogy G, Grenov A, et al.
Fap2 Mediates Fusobacterium nucleatum Colorectal Adenocarcinoma
Enrichment by Binding to Tumor-Expressed Gal-GalNAc. Cell Host \&
Microbe. 2016;20:215--25.

\hypertarget{ref-arthur_intestinal_2012}{}
5. Arthur JC, Perez-Chanona E, Mühlbauer M, Tomkovich S, Uronis JM, Fan
T-J, et al. Intestinal inflammation targets cancer-inducing activity of
the microbiota. Science (New York, NY). 2012;338:120--3.

\hypertarget{ref-kostic_fusobacterium_2013}{}
6. Kostic AD, Chun E, Robertson L, Glickman JN, Gallini CA, Michaud M,
et al. Fusobacterium nucleatum potentiates intestinal tumorigenesis and
modulates the tumor-immune microenvironment. Cell Host \& Microbe.
2013;14:207--15.

\hypertarget{ref-wu_human_2009}{}
7. Wu S, Rhee K-J, Albesiano E, Rabizadeh S, Wu X, Yen H-R, et al. A
human colonic commensal promotes colon tumorigenesis via activation of T
helper type 17 T cell responses. Nature Medicine. 2009;15:1016--22.

\hypertarget{ref-zackular_manipulation_2016}{}
8. Zackular JP, Baxter NT, Chen GY, Schloss PD. Manipulation of the Gut
Microbiota Reveals Role in Colon Tumorigenesis. mSphere. 2016;1.

\hypertarget{ref-zackular_gut_2013}{}
9. Zackular JP, Baxter NT, Iverson KD, Sadler WD, Petrosino JF, Chen GY,
et al. The gut microbiome modulates colon tumorigenesis. mBio.
2013;4:e00692--00613.

\hypertarget{ref-baxter_structure_2014}{}
10. Baxter NT, Zackular JP, Chen GY, Schloss PD. Structure of the gut
microbiome following colonization with human feces determines colonic
tumor burden. Microbiome. 2014;2:20.

\hypertarget{ref-ahn_human_2013}{}
11. Ahn J, Sinha R, Pei Z, Dominianni C, Wu J, Shi J, et al. Human gut
microbiome and risk for colorectal cancer. Journal of the National
Cancer Institute. 2013;105:1907--11.

\hypertarget{ref-baxter_microbiota-based_2016}{}
12. Baxter NT, Ruffin MT, Rogers MAM, Schloss PD. Microbiota-based model
improves the sensitivity of fecal immunochemical test for detecting
colonic lesions. Genome Medicine. 2016;8:37.

\hypertarget{ref-chen_human_2012}{}
13. Chen W, Liu F, Ling Z, Tong X, Xiang C. Human intestinal lumen and
mucosa-associated microbiota in patients with colorectal cancer. PloS
One. 2012;7:e39743.

\hypertarget{ref-wang_structural_2012}{}
14. Wang T, Cai G, Qiu Y, Fei N, Zhang M, Pang X, et al. Structural
segregation of gut microbiota between colorectal cancer patients and
healthy volunteers. The ISME journal. 2012;6:320--9.

\hypertarget{ref-burns_virulence_2015}{}
15. Burns MB, Lynch J, Starr TK, Knights D, Blekhman R. Virulence genes
are a signature of the microbiome in the colorectal tumor
microenvironment. Genome Medicine. 2015;7:55.

\hypertarget{ref-zeller_potential_2014}{}
16. Zeller G, Tap J, Voigt AY, Sunagawa S, Kultima JR, Costea PI, et al.
Potential of fecal microbiota for early-stage detection of colorectal
cancer. Molecular Systems Biology. 2014;10:766.

\hypertarget{ref-flemer_tumour-associated_2017}{}
17. Flemer B, Lynch DB, Brown JMR, Jeffery IB, Ryan FJ, Claesson MJ, et
al. Tumour-associated and non-tumour-associated microbiota in colorectal
cancer. Gut. 2017;66:633--43.

\hypertarget{ref-GimenoGarca2012}{}
18. García AZG. Factors influencing colorectal cancer screening
participation. Gastroenterology Research and Practice {[}Internet{]}.
Hindawi Limited; 2012;2012:1--8. Available from:
\url{https://doi.org/10.1155/2012/483417}

\hypertarget{ref-geng_diversified_2013}{}
19. Geng J, Fan H, Tang X, Zhai H, Zhang Z. Diversified pattern of the
human colorectal cancer microbiome. Gut Pathogens. 2013;5:2.

\hypertarget{ref-dejea_microbiota_2014}{}
20. Dejea CM, Wick EC, Hechenbleikner EM, White JR, Mark Welch JL,
Rossetti BJ, et al. Microbiota organization is a distinct feature of
proximal colorectal cancers. Proceedings of the National Academy of
Sciences of the United States of America. 2014;111:18321--6.

\hypertarget{ref-ecoli_Arthur_2014}{}
21. Arthur JC, Gharaibeh RZ, Mühlbauer M, Perez-Chanona E, Uronis JM,
McCafferty J, et al. Microbial genomic analysis reveals the essential
role of inflammation in bacteria-induced colorectal cancer. Nature
Communications {[}Internet{]}. Springer Nature; 2014;5:4724. Available
from: \url{https://doi.org/10.1038/ncomms5724}

\hypertarget{ref-strep_Aymeric_2017}{}
22. Aymeric L, Donnadieu F, Mulet C, Merle L du, Nigro G, Saffarian A,
et al. Colorectal cancer specific conditions promoteStreptococcus
gallolyticusgut colonization. Proceedings of the National Academy of
Sciences {[}Internet{]}. Proceedings of the National Academy of
Sciences; 2017;115:E283--91. Available from:
\url{https://doi.org/10.1073/pnas.1715112115}

\hypertarget{ref-weir_stool_2013}{}
23. Weir TL, Manter DK, Sheflin AM, Barnett BA, Heuberger AL, Ryan EP.
Stool microbiome and metabolome differences between colorectal cancer
patients and healthy adults. PloS One. 2013;8:e70803.

\hypertarget{ref-bfrag_Boleij_2014}{}
24. Boleij A, Hechenbleikner EM, Goodwin AC, Badani R, Stein EM, Lazarev
MG, et al. The bacteroides fragilis toxin gene is prevalent in the colon
mucosa of colorectal cancer patients. Clinical Infectious Diseases
{[}Internet{]}. Oxford University Press (OUP); 2014;60:208--15.
Available from: \url{https://doi.org/10.1093/cid/ciu787}

\hypertarget{ref-sanapareddy_increased_2012}{}
25. Sanapareddy N, Legge RM, Jovov B, McCoy A, Burcal L, Araujo-Perez F,
et al. Increased rectal microbial richness is associated with the
presence of colorectal adenomas in humans. The ISME journal.
2012;6:1858--68.

\hypertarget{ref-lu_mucosal_2016}{}
26. Lu Y, Chen J, Zheng J, Hu G, Wang J, Huang C, et al. Mucosal
adherent bacterial dysbiosis in patients with colorectal adenomas.
Scientific Reports. 2016;6:26337.

\hypertarget{ref-hale_shifts_2017}{}
27. Hale VL, Chen J, Johnson S, Harrington SC, Yab TC, Smyrk TC, et al.
Shifts in the Fecal Microbiota Associated with Adenomatous Polyps.
Cancer Epidemiology, Biomarkers \& Prevention: A Publication of the
American Association for Cancer Research, Cosponsored by the American
Society of Preventive Oncology. 2017;26:85--94.

\hypertarget{ref-shah_leveraging_2017}{}
28. Shah MS, DeSantis TZ, Weinmaier T, McMurdie PJ, Cope JL, Altrichter
A, et al. Leveraging sequence-based faecal microbial community survey
data to identify a composite biomarker for colorectal cancer. Gut. 2017;

\hypertarget{ref-brim_microbiome_2013}{}
29. Brim H, Yooseph S, Zoetendal EG, Lee E, Torralbo M, Laiyemo AO, et
al. Microbiome analysis of stool samples from African Americans with
colon polyps. PloS One. 2013;8:e81352.

\hypertarget{ref-Flynn_preprint_2017}{}
30. Flynn KJ, Ruffin MT, Turgeon DK, Schloss PD. Spatial variation of
the native colon microbiota in healthy adults. Cold Spring Harbor
Laboratory; 2017; Available from: \url{https://doi.org/10.1101/189886}

\hypertarget{ref-Salter_contamination_2014}{}
31. Salter SJ, Cox MJ, Turek EM, Calus ST, Cookson WO, Moffatt MF, et
al. Reagent and laboratory contamination can critically impact
sequence-based microbiome analyses. BMC Biology {[}Internet{]}. Springer
Nature; 2014;12. Available from:
\url{https://doi.org/10.1186/s12915-014-0087-z}

\hypertarget{ref-keku_gastrointestinal_2015}{}
32. Keku TO, Dulal S, Deveaux A, Jovov B, Han X. The gastrointestinal
microbiota and colorectal cancer. American Journal of Physiology -
Gastrointestinal and Liver Physiology {[}Internet{]}. 2015 {[}cited 2017
Oct 30{]};308:G351--63. Available from:
\url{http://ajpgi.physiology.org/lookup/doi/10.1152/ajpgi.00360.2012}

\hypertarget{ref-vogtmann_epidemiologic_2016}{}
33. Vogtmann E, Goedert JJ. Epidemiologic studies of the human
microbiome and cancer. British Journal of Cancer {[}Internet{]}. 2016
{[}cited 2017 Oct 30{]};114:237--42. Available from:
\url{http://www.nature.com/doifinder/10.1038/bjc.2015.465}

\hypertarget{ref-kostic_genomic_2012}{}
34. Kostic AD, Gevers D, Pedamallu CS, Michaud M, Duke F, Earl AM, et
al. Genomic analysis identifies association of Fusobacterium with
colorectal carcinoma. Genome Research. 2012;22:292--8.

\hypertarget{ref-zackular_human_2014}{}
35. Zackular JP, Rogers MAM, Ruffin MT, Schloss PD. The human gut
microbiome as a screening tool for colorectal cancer. Cancer Prevention
Research (Philadelphia, Pa). 2014;7:1112--21.

\hypertarget{ref-schloss_introducing_2009}{}
36. Schloss PD, Westcott SL, Ryabin T, Hall JR, Hartmann M, Hollister
EB, et al. Introducing mothur: Open-Source, Platform-Independent,
Community-Supported Software for Describing and Comparing Microbial
Communities. ApplEnvironMicrobiol {[}Internet{]}. 2009 {[}cited 12AD Jan
1{]};75:7537--41. Available from:
\url{http://aem.asm.org/cgi/content/abstract/75/23/7537}

\hypertarget{ref-rognes_vsearch_2016}{}
37. Rognes T, Flouri T, Nichols B, Quince C, Mahé F. VSEARCH: A
versatile open source tool for metagenomics. PeerJ. 2016;4:e2584.

\hypertarget{ref-westcott_opticlust_2017}{}
38. Westcott SL, Schloss PD. OptiClust, an Improved Method for Assigning
Amplicon-Based Sequence Data to Operational Taxonomic Units. mSphere.
2017;2.

\hypertarget{ref-benjamini_controlling_1995}{}
39. Benjamini Y, Hochberg Y. Controlling the false discovery rate: A
practical and powerful approach to multiple testing. Journal of the
Royal Statistical Society Series B (Methodological). 1995;57:289--300.

\hypertarget{ref-r_citation_2017}{}
40. R Core Team. R: A language and environment for statistical computing
{[}Internet{]}. Vienna, Austria: R Foundation for Statistical Computing;
2017. Available from: \url{https://www.R-project.org/}

\hypertarget{ref-rcompanion_citation_2017}{}
41. Mangiafico S. Rcompanion: Functions to support extension education
program evaluation {[}Internet{]}. 2017. Available from:
\url{https://CRAN.R-project.org/package=rcompanion}

\hypertarget{ref-car_citation_2011}{}
42. Fox J, Weisberg S. An R companion to applied regression
{[}Internet{]}. Second. Thousand Oaks CA: Sage; 2011. Available from:
\url{http://socserv.socsci.mcmaster.ca/jfox/Books/Companion}

\hypertarget{ref-lme4_citation_2015}{}
43. Bates D, Mächler M, Bolker B, Walker S. Fitting linear mixed-effects
models using lme4. Journal of Statistical Software. 2015;67:1--48.

\hypertarget{ref-epir_citation_2017}{}
44. Telmo Nunes MS with contributions from, Heuer C, Marshall J, Sanchez
J, Thornton R, Reiczigel J, et al. EpiR: Tools for the analysis of
epidemiological data {[}Internet{]}. 2017. Available from:
\url{https://CRAN.R-project.org/package=epiR}

\hypertarget{ref-metafor_citation_2010}{}
45. Viechtbauer W. Conducting meta-analyses in R with the metafor
package. Journal of Statistical Software {[}Internet{]}. 2010;36:1--48.
Available from: \url{http://www.jstatsoft.org/v36/i03/}

\hypertarget{ref-vegan_citation_2017}{}
46. Oksanen J, Blanchet FG, Friendly M, Kindt R, Legendre P, McGlinn D,
et al. Vegan: Community ecology package {[}Internet{]}. 2017. Available
from: \url{https://CRAN.R-project.org/package=vegan}

\hypertarget{ref-caret_citation_2017}{}
47. Jed Wing MKC from, Weston S, Williams A, Keefer C, Engelhardt A,
Cooper T, et al. Caret: Classification and regression training
{[}Internet{]}. 2017. Available from:
\url{https://CRAN.R-project.org/package=caret}

\hypertarget{ref-randomforest_citation_2002}{}
48. Liaw A, Wiener M. Classification and regression by randomForest. R
News {[}Internet{]}. 2002;2:18--22. Available from:
\url{http://CRAN.R-project.org/doc/Rnews/}

\hypertarget{ref-ggplot2_citation_2009}{}
49. Wickham H. Ggplot2: Elegant graphics for data analysis
{[}Internet{]}. Springer-Verlag New York; 2009. Available from:
\url{http://ggplot2.org}

\hypertarget{ref-gridextra_citation_2017}{}
50. Auguie B. GridExtra: Miscellaneous functions for ``grid'' graphics
{[}Internet{]}. 2017. Available from:
\url{https://CRAN.R-project.org/package=gridExtra}

\newpage

\textbf{Table 1: Total Individuals in each Study Included in the Stool
Analysis}

\footnotesize

\begin{longtable}[]{@{}cccccc@{}}
\toprule
Study & Data Stored & Region & Control (n) & Adenoma (n) & Carcinoma
(n)\tabularnewline
\midrule
\endhead
Ahn & DBGap & V3-4 & 148 & 0 & 62\tabularnewline
Baxter & SRA & V4 & 172 & 198 & 120\tabularnewline
Brim & SRA & V1-3 & 6 & 6 & 0\tabularnewline
Flemer & Author & V3-4 & 37 & 0 & 43\tabularnewline
Hale & Author & V3-5 & 473 & 214 & 17\tabularnewline
Wang & SRA & V3 & 56 & 0 & 46\tabularnewline
Weir & Author & V4 & 4 & 0 & 7\tabularnewline
Zeller & SRA & V4 & 50 & 37 & 41\tabularnewline
\bottomrule
\end{longtable}

\normalsize
\newpage

\textbf{Table 2: Studies with Tissue Samples Included in the Analysis}

\footnotesize

\begin{longtable}[]{@{}cccccc@{}}
\toprule
Study & Data Stored & Region & Control (n) & Adenoma (n) & Carcinoma
(n)\tabularnewline
\midrule
\endhead
Burns & SRA & V5-6 & 18 & 0 & 16\tabularnewline
Chen & SRA & V1-3 & 9 & 0 & 9\tabularnewline
Dejea & SRA & V3-5 & 31 & 0 & 32\tabularnewline
Flemer & Author & V3-4 & 103 & 37 & 94\tabularnewline
Geng & SRA & V1-2 & 16 & 0 & 16\tabularnewline
Lu & SRA & V3-4 & 20 & 20 & 0\tabularnewline
Sanapareddy & Author & V1-2 & 38 & 0 & 33\tabularnewline
\bottomrule
\end{longtable}

\normalsize
\newpage

\textbf{Figure 1: Significant Bacterial Community Metrics for Adenoma or
Carcinoma in Stool.} A) Adenoma evenness. B) Carcinoma evenness. C)
Carcinoma Shannon diversity. Blue represents controls and red represents
either adenoma (panel A) or carcinoma (panel B and C). The black lines
represent the median value for each repsective group.

\textbf{Figure 2: Odds Ratio for Adenoma or Carcinoma based on Bacterial
Community Metrics in Stool.} A) Community-based odds ratio for adenoma.
B) Community-based odds ratio for carcinoma. Colors represent the
different variable regions used within the respective study.

\textbf{Figure 3: The AUC of Indivdiual Significant OR Taxa to classify
Carcinoma.} A) Stool samples. B) Unmatched tissue samples. The larger
circle represents the median AUC of all studies and the smaller circles
represent the individual AUC for a particular study. The dotted line
denotes an AUC of 0.5.

\textbf{Figure 4: Stool Random Forest Model Train AUCs.} A) Adenoma
random forest model AUCs between all genera, all OTU, and select model
based on significant OR taxa. B) Carcinoma random forest model AUCs
between all genera, all OTU, and select model based on significant OR
taxa. The black line represents the median AUC for the respective group.
If no values are present in the singificant OR taxa group then there
were no significant taxa identified and no model was tested.

\textbf{Figure 5: Stool Random Forest Genus-Based Model Test AUCs.} A)
Test AUCs of adenoma models using all genera across study. B) Test AUCs
of carcinoma models using all genera or significant OR taxa only. The
black line represents the AUC at 0.5. The red lines represent the median
AUC of all test AUCs for a specific study.

\textbf{Figure 6: Most Common Taxa Across Carcinoma Full Community Stool
Study Models.} A) Common taxa in the top 10 percent for carcinoma Random
Forest all taxa-based models. B) Common taxa in the top 10 percent for
carcinoma Random Forest all OTU-based models.

\newpage

\textbf{Figure S1: Odds Ratio for Adenoma or Carcinoma based on
Bacterial Community Metrics in Tissue.} A) Community-based odds ratio
for adenoma. B) Community-based odds ratio for carcinoma. Colors
represent the different variable regions used within the respective
study.

\textbf{Figure S2: Tissue Random Forest Model Train AUCs.} A) Adenoma
random forest model AUCs between all genera, all OTU, and select model
based on significant OR taxa in unmatched and matched tissue. B)
Carcinoma random forest model AUCs between all genera, all OTU, and
select model based on significant OR taxa in unmatched and matched
tissue. The black line represents the median AUC for the respective
group. If no values are present in the singificant OR taxa group then
there were no significant taxa identified and no model was tested.

\textbf{Figure S3: Tissue Random Forest Genus-Based Model Test AUCs.} A)
Test AUCs of adenoma models using all genera across study. B) Test AUCs
of carcinoma models using all genera for matched tissue studies. C) Test
AUCs of carcinoma models using all genera or significant OR taxa only
for unmatched tissue studies The black line represents the AUC at 0.5.
The red lines represent the median AUC of all test AUCs for a specific
study.

\textbf{Figure S4: Most Common Genera Across Full Community Tissue Study
Models.} A) Common genera in the top 10 percent for matched carcinoma
Random Forest all genera-based models. B) Common genera in the top 10
percent for unmatched carcinoma Random Forest all genera-based models.
C) Common genera in the top 10 percent for matched carcinoma Random
Forest all OTU-based models. D) Common genera in the top 10 percent for
unmatched carcinoma Random Forest all OTU-based models.

\newpage


\end{document}
