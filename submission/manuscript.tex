\documentclass[12pt,]{article}
\usepackage{lmodern}
\usepackage{amssymb,amsmath}
\usepackage{ifxetex,ifluatex}
\usepackage{fixltx2e} % provides \textsubscript
\ifnum 0\ifxetex 1\fi\ifluatex 1\fi=0 % if pdftex
  \usepackage[T1]{fontenc}
  \usepackage[utf8]{inputenc}
\else % if luatex or xelatex
  \ifxetex
    \usepackage{mathspec}
  \else
    \usepackage{fontspec}
  \fi
  \defaultfontfeatures{Ligatures=TeX,Scale=MatchLowercase}
\fi
% use upquote if available, for straight quotes in verbatim environments
\IfFileExists{upquote.sty}{\usepackage{upquote}}{}
% use microtype if available
\IfFileExists{microtype.sty}{%
\usepackage{microtype}
\UseMicrotypeSet[protrusion]{basicmath} % disable protrusion for tt fonts
}{}
\usepackage[margin=1.0in]{geometry}
\usepackage{hyperref}
\hypersetup{unicode=true,
            pdfborder={0 0 0},
            breaklinks=true}
\urlstyle{same}  % don't use monospace font for urls
\usepackage{longtable,booktabs}
\usepackage{graphicx,grffile}
\makeatletter
\def\maxwidth{\ifdim\Gin@nat@width>\linewidth\linewidth\else\Gin@nat@width\fi}
\def\maxheight{\ifdim\Gin@nat@height>\textheight\textheight\else\Gin@nat@height\fi}
\makeatother
% Scale images if necessary, so that they will not overflow the page
% margins by default, and it is still possible to overwrite the defaults
% using explicit options in \includegraphics[width, height, ...]{}
\setkeys{Gin}{width=\maxwidth,height=\maxheight,keepaspectratio}
\IfFileExists{parskip.sty}{%
\usepackage{parskip}
}{% else
\setlength{\parindent}{0pt}
\setlength{\parskip}{6pt plus 2pt minus 1pt}
}
\setlength{\emergencystretch}{3em}  % prevent overfull lines
\providecommand{\tightlist}{%
  \setlength{\itemsep}{0pt}\setlength{\parskip}{0pt}}
\setcounter{secnumdepth}{0}
% Redefines (sub)paragraphs to behave more like sections
\ifx\paragraph\undefined\else
\let\oldparagraph\paragraph
\renewcommand{\paragraph}[1]{\oldparagraph{#1}\mbox{}}
\fi
\ifx\subparagraph\undefined\else
\let\oldsubparagraph\subparagraph
\renewcommand{\subparagraph}[1]{\oldsubparagraph{#1}\mbox{}}
\fi

%%% Use protect on footnotes to avoid problems with footnotes in titles
\let\rmarkdownfootnote\footnote%
\def\footnote{\protect\rmarkdownfootnote}

%%% Change title format to be more compact
\usepackage{titling}

% Create subtitle command for use in maketitle
\newcommand{\subtitle}[1]{
  \posttitle{
    \begin{center}\large#1\end{center}
    }
}

\setlength{\droptitle}{-2em}
  \title{}
  \pretitle{\vspace{\droptitle}}
  \posttitle{}
  \author{}
  \preauthor{}\postauthor{}
  \date{}
  \predate{}\postdate{}

\usepackage{helvet} % Helvetica font
\renewcommand*\familydefault{\sfdefault} % Use the sans serif version of the font
\usepackage[T1]{fontenc}

\usepackage[none]{hyphenat}

\usepackage{setspace}
\doublespacing
\setlength{\parskip}{1em}

\usepackage{lineno}

\usepackage{pdfpages}

\usepackage{amsmath}

\usepackage{mathtools}

\begin{document}

\section{Making Sense of the Noise: Leveraging Existing 16S rRNA Gene
Surveys to Identify Key Community Members in Colorectal
Cancer}\label{making-sense-of-the-noise-leveraging-existing-16s-rrna-gene-surveys-to-identify-key-community-members-in-colorectal-cancer}

\begin{center}
\vspace{25mm}

Marc A Sze${^1}$ and Patrick D Schloss${^1}$${^\dagger}$

\vspace{20mm}

$\dagger$ To whom correspondence should be addressed: pschloss@umich.edu

$1$ Department of Microbiology and Immunology, University of Michigan, Ann Arbor, MI




\end{center}

Co-author e-mails:

\begin{itemize}
\tightlist
\item
  \href{mailto:marcsze@med.umich.edu}{\nolinkurl{marcsze@med.umich.edu}}
\end{itemize}

\newpage

\linenumbers

\subsection{Abstract}\label{abstract}

\textbf{Background.} An increasing body of literature suggests that
there is a crucial role for the microbiota in colorectal cancer (CRC)
pathogenesis. Important drivers within this context have ranged from
individual microbes to the whole community. Our study expands on a
recent meta-analysis investigating microbial biomarkers for CRC by
testing the hypothesis that the bacterial community has important
associations to both early (adenoma) and late (carcinoma) stage disease.
To test this hypothesis we examined both feces (n = 1737) and colon
tissue (492 total samples from 350 individuals) across 14 previously
published 16S rRNA gene sequencing studies on CRC and the microbiota.

\textbf{Results.} Fecal samples had a significant decrease for both
Shannon diversity and evenness, after correcting for study effect and
variable region sequenced, with more severe disease (P-value \textless{}
0.05). This reduction in evenness translated into small increases in
relative risk for adenoma (P-value = 0.032) and carcinoma stages of CRC
(P-value = 0.00034) while the reduction in Shannon diversity only
translated into an increased relative risk for developing carcinomas
(P-value = 0.0047). Increases in mouth-associated microbes were commonly
in the top 5 most significantly increased relative risk of adenoma and
carcinoma for both stool and tissue samples. A prediction model for
adenoma and carcinoma was built using either the whole community or
selected genera with highest and lowest relative risk from fecal and
tissue samples. Both approaches resulted in similar classification
success according to Area Under the Curve (AUC) regardless of whether
genera or OTUs were used to build the model. The most important groups
within the full community models consistently belonged to genera such as
\emph{Ruminococcus}, \emph{Bacteroides}, and \emph{Roseburia} across
studies. Although a number of associations between the microbiota and
CRC were identified, the majority of studies that we used in this
meta-analysis were only individually adequately powered for large effect
sizes.

\textbf{Conclusions.} These data provide support for the importance of
the bacterial community to both adenoma and carcinoma genesis. The
evidence collected within this study on the role of the microbiota in
CRC identifies a number of correlations that may not have been detected
because of the low power associated with the majority of studies that
have been performed to date.

\subsubsection{Keywords}\label{keywords}

microbiota; colorectal cancer; polyps; adenoma; meta-analysis.

\newpage

\subsection{Background}\label{background}

Colorectal cancer (CRC) is a growing world-wide health problem in which
the microbiota has been purported to play an active role in disease
pathogenesis {[}1,2{]}. Numerous studies have shown the importance of
both individual microbes {[}3--7{]} and the overall community
{[}8--10{]} in tumorgenesis using mouse models of CRC. There have also
been numerous case/control studies investigating the microbiota in the
formation of both adenoma and carcinoma. A recent meta-analysis
investigated whether specific biomarkers could be consistently
identified using multiple data sets {[}11{]}. This meta-analysis focused
on identifying microbial signatures of CRC (biomarkers) but did so on a
small total number of individuals and only investigated stool.

Although there has been an intense focus on microbiota-based biomarker
discovery for CRC, the number of candidate genera seem to be endless.
Some studies point towards mouth-associated genera such as
\emph{Fusobacterium}, \emph{Peptostreptococcus}, \emph{Parvimonas}, and
\emph{Porphyromonas} as key enriched genera {[}6,12--18{]}. Yet, even in
these studies, mouth-associated genera are far from the only microbes
identified to be associated with CRC. These other genera include, but
are not limited to, \emph{Providencia}, \emph{Mogibacterium},
\emph{Enterococcus}, \emph{Escherichia/Shigella}, \emph{Klebsiella}, and
\emph{Streptococcus} {[}14--16{]}. In fact, there is good \emph{in vivo}
evidence that \emph{Escherichia/Shigella} and \emph{Streptococcus} can
be important in the pathogenesis of CRC {[}5,19,20{]}. Other studies
have also identified \emph{Akkermansia muciniphila} and
\emph{Bacteroides fragilis} as potential markers of CRC with good
mechanistic studies for the latter {[}15,21,22{]}. A recent
meta-analysis confirmed the correlations of certain mouth-associated
genera and \emph{Akkermansia muciniphila} with carcinoma {[}11{]}.
However, the sample size (n = 509) is equal to or less than some more
recent individual studies investigating the microbiota and CRC, making
it hard to know how extrapolatable these findings are. That particular
meta-analysis also added more potential microbial associations to both
carcinoma (\emph{Pantoea agglomerans} \emph{Ruminococcus},
\emph{Lactobacillus}) and adenoma (\emph{Prevotella},
\emph{Methanosphaera}, \emph{Succinovibrio}, \emph{Haemophilus
parainfluenzae}, \emph{Ruminococcus}, \emph{Lactobacillus}) stages of
CRC that need to be investigated further, since a number of these genera
have been found to be enriched in controls and not disease
{[}13,16,17{]}. Additionally, genera like \emph{Roseburia} have been
found in some studies to be increased in CRC but in others to either be
decreased or have no difference {[}15,18,23,24{]}.

Most of these studies have focused on carcinoma but the adenoma
observations are not any clearer at identifying candidate genera of
disease. Groups focusing on broad scale community metrics have found
that metrics such as richness are decreased in the adenoma stage of CRC
versus controls. Other studies have identified \emph{Lactococcus},
\emph{Pseudomonas}, \emph{Acidovorax}, \emph{Cloacibacterium},
\emph{Helicobacter}, \emph{Lactobacillus}, \emph{Bilophila},
\emph{Desulfovibrio}, and \emph{Mogibacterium} to be increased in
adenoma {[}25--27{]}. Based on the studies mentioned, there seems to be
very little overlap between the genera identified to be associated with
adenoma and carcinoma, with \emph{Lactobacillus} being one of the few
commonalities.

Targeting the identification of CRC microbial biomarkers within stool
seems logical since it offers an easy and cost-effective way to stratify
risk and the current gold standard for diagnosis, a colonoscopy, can be
time-consuming and is not without risk of complications. Although stool
represents an easy and less invasive way to assess risk, it is not clear
how well this sample reflects adenoma- and carcinoma- associated
microbial communities. Some studies have tried to assess this in health
and disease but are limited by their sample size {[}18,28{]}. Sampling
the microbiota directly associated with colon tissue may provide clearer
answers but is not without limitations. The community present for
sampling following the colonoscopy bowel prep may reflect the better
adhered microbiota versus the resident community. Additionally, these
samples contain more host DNA, potentially limiting the types of
analysis that can be done. It is well known that low biomass samples can
be very difficult to work with and results can be study dependent due to
the randomness of contamination {[}29{]}.

In comparison to the previous meta-analysis, this study significantly
increases the total stool samples investigated, re-examines important
genera across adenoma and carcinoma across study, and examines
differences and similarities between stool and tissue microbiota in the
context of CRC. Importantly, this analysis and approach could provide
valuable insights into the common genera that are both protective and
detrimental in CRC and whether broad bacterial community measurements
can account for these changes that were not provided by earlier
meta-analysis studies {[}11{]}.

Using both feces (n = 1737) and colon tissues (492 samples from 350
individuals) totaling over 2229 total samples across 14 studies
{[}12--18,21,23--27,30{]} {[}Table 1 \& 2{]}, we expand both the breadth
and scope of the previous meta-analysis to investigate whether the
bacterial community or specific members are more important risk factors
for both adenoma and carcinoma stages of CRC. To accomplish this we
first assessed whether bacterial diversity changes throughout disease
(control to adenoma to carcinoma) and if it results in an increased
relative risk (RR) for adenoma or carcinoma stages of CRC. We then
assessed what genera, if any, increase or decrease the RR of adenoma or
carcinoma stages of CRC. Next, using Random Forest models, we analyzed
whether the full community or only the combined top 5 increased and top
5 decreased RR genera resulted in better model classification, based on
the area under the curve (AUC). Finally, we also examined at what effect
and sample size the studies used were powered for and the sample size
needed to get to the traditionally accepted 80\% power. Our results from
these analyses suggests that the bacterial community changes as disease
severity worsens, that specific members are important for disease
classification, and that many of the studies are underpowered for
assessing small effect sizes.

\newpage

\subsection{Results}\label{results}

\textbf{\emph{Lower Bacterial Diversity is Associated with Increased RR
of CRC:}} To assess differences in broad scale community metrics as
disease severity worsens Operational Taxonomic Unit (OTU) richness,
evenness, and Shannon diversity measurements were power transformed and
Z-score normalized. These metrics are commonly used to assess the total
number of OTUs, the equality of their abundance, and the overall
diversity, respectively. Using linear mixed-effect models to control for
study and variable region we assessed whether OTU richness, evenness, or
Shannon diversity changed in a step-wise manner with disease severity.
In stool, there was a significant decrease in both evenness and Shannon
diversity as disease severity moved from control to adenoma to carcinoma
(P-value = 0.025 and 0.043, respectively) {[}Figure 1A{]}. We next
tested whether the detectable differences in community significantly
increased in RR of having an adenoma or carcinoma. For fecal samples, a
decrease versus the overall median in evenness resulted in a
significantly increased RR for carcinoma (RR = 1.36 (1.15 - 1.61),
P-value = 0.00034) and adenoma (RR = 1.16 (1.01 - 1.34), P-value =
0.032) while a decrease versus the overall median in Shannon diversity
only increased the RR for carcinoma (RR = 1.33 (1.09 - 1.62), P-value =
0.0047) {[}Figure 2{]}. Using the Bray-Curtis distance metric and
PERMANOVA, it was also possible to identify significant bacterial
community changes, in specific studies, for both carcinoma-associated
and adenoma-associated microbiota versus control {[}Table S1 \& S2{]}.

Using similar transformations for tissue samples, linear mixed-effect
models were used on the transformed combined data to control for study,
re-sampling of the same individual, and 16S variable region to test
whether OTU richness, evenness, or Shannon diversity changed in a
step-wise manner as disease severity increased. For colon tissue, there
were no significant changes in OTU richness, evenness, or Shannon
diversity as disease severity progressed from control to adenoma to
carcinoma (P-value \textgreater{} 0.05) {[}Figure 1B \& C{]}. We next
analyzed the RR, for matched (unaffected tissue and an adenoma or
carcinoma from the same individual) and unmatched (control and adenoma
or carcinoma tissue not from the same individual) colon tissue samples.
For individuals at either an adenoma or carcinoma stage of disease there
was no significant change in RR based on lower than median values for
OTU richness, evenness, and Shannon diversity {[}Table S3-S5{]}. Similar
to stool samples, significant differences in bacterial community,
assessed by PERMANOVA, were identified in unmatched tissue samples, for
those at either adenoma or carcinoma stage of CRC {[}Table S6 \& S7{]}.
For studies with matched samples no differences in bacterial community
were observed when assessed with PERMANOVA {[}Table S6 \& S7{]}. These
tissue results suggest that the microbiota within an individual are
similar to each other regardless of disease status.

\textbf{\emph{Mouth-Associated Genera are Associated with an Increased
RR of CRC:}} Next, we asked if being higher than the median relative
abundance, for any specific genera, resulted in an altered RR for
adenoma or carcinoma, in stool and colon tissue, due to our previous
observations of small increases in RR using OTU richness and Shannon
diversity. To investigate this we analyzed all common genera across each
study, in colon tissue or stool, and assessed whether a relative
abundance higher than the median results in an increase or decrease in
RR. Mouth-associated genera were commonly found in the top 5 genera
associated with an increased RR of having an adenoma
(\emph{Pyramidobacter} {[}Figure 3A{]} and \emph{Rothia} {[}Figure
3C{]}) and carcinoma (\emph{Fusobacterium}, \emph{Parvimonas},
\emph{Porphyromonas}, and \emph{Peptostreptococcus} {[}Figure 3B{]} and
\emph{Fusobacterium} {[}Figure 3D{]}) for both stool and colon tissue
samples. Conversely, genera commonly associated with a normal
gastrointestinal tract were correlated with a decreased RR for both
adenoma and carcinoma for both stool and colon tissue samples {[}Figure
3{]}. Even though mouth-associated genera were identified across disease
stage, there was little direct overlap of the top 5 increased or
decreased RR genera between both stages and sample site.

When observing RRs with a P-value less than 0.05 there was almost no
overlap between genera from stool or colon tissue and when they were
similar the RR was in opposite directions (e.g. \emph{Lactococcus})
{[}Table S8 \& S9{]}. Many of the genera that had RRs with a P-value
under 0.05 for colon tissue are also highly prevalent in contamination,
specifically, \emph{Novosphingobium}, \emph{Selemonas}, and
\emph{Achromobacter} {[}Figure 3 \& Table S8-S9{]}. For carcinoma stage
of CRC, certain mouth-associated genera (\emph{Fusobacterium},
\emph{Parvimonas}, \emph{Peptostreptococcus}) had a high RR for both
colon tissue and stool samples {[}Table S10 \& S11{]}. Finally, these
data suggest that the most significantly increased RR genera for tissue
was \emph{Camplyobacter} while in stool it was \emph{Peptostreptococcus}
{[}Table S10 \& S11{]}.

\textbf{\emph{Select Genera Models can Recapitulate Whole Community
Models:}} Since specific genera increased RR for carcinoma over
diversity metrics we assessed whether the bacterial community was better
at classifying disease versus only a select group of genera. We selected
these genera based on their RR and P-value significance and used two
approaches to test this question. The first approach used genus level
data and tested for differences in AUC between all genera and selected
genera. A single study was used for training the model prior to testing
on all other studies and this was repeated for every study in the
meta-analysis. The second approach used OTU level data and tested for a
generalized decrease in the 10-fold cross validation (CV) model AUC
which is a common approach used to guard against over-fitting. This was
applied across study and the AUC of the all OTUs model was compared
against the model that used only OTUs that taxonomically classified to
selected genera.

For the first approach using the genera-based models, the training set
median AUC for model classification was similar for both the select
genera and full genera models, for both tissue and stool studies
{[}Figure S2-S3{]}. When analyzing the tests sets that were comprised of
genera data from other studies, both models had a similar ability to
detect individuals with adenomas or carcinomas, with the select genera
models performing better in some instances {[}Figure S4-S6{]}.
Conversely, the second approach that used OTU-based models showed a
slight decrease in median AUC between the full and select models, with
one exception to this generalization being the carcinoma models for
matched colon tissue {[}Figure 4 \& 5{]}.

In stool, the most common genera in the top 10 most important variables,
in the full community models using the first approach, were
\emph{Ruminococcus}, \emph{Bacteroides}, and \emph{Roseburia} {[}Figure
6A \& B{]}. Regardless of sample type, mouth-associated genera were
present in models for the carcinoma stage of CRC {[}Figure 6A \& B{]}.
Yet, none were present in the majority of studies and
\emph{Fusobacterium} was the only genus present in the adenoma stage of
CRC {[}Figure 6A \& B{]}. For the second approach that utilized full
OTU-based models, \emph{Ruminococcaceae} was present in the top 10
consistently for both adenoma and carcinoma models while
\emph{Roseburia} was only present in many adenoma models and
\emph{Bacteroides} was present in the overwhelming majority of the
carcinoma models {[}Figure 6C \& 6D{]}.

Unlike the stool-based Random Forest models, the tissue-based models,
for the full genera from the first approach, showed no consistent
representation of \emph{Ruminococcaceae}, \emph{Ruminococcus},
\emph{Bacteroides}, and \emph{Roseburia} in the top 10 most important
model variables across study {[}Figure S7{]}. The vast majority of the
top 10 model variables for the genera- and OTU-based models using colon
tissue tended to be study specific. Further, there was very little
overlap in the top 10 important variables between adenoma and carcinoma
stage models, regardless of whether colon tissue or stool was used
{[}Figure 6 \& S7{]}. This discordance between stool and colon tissue
samples also applies to the mouth-associated genera with one noticeable
skew being that \emph{Fusobacterium} and \emph{Fusobacteriaceae} occur
more often in the top 10 of matched versus unmatched colon tissue Random
Forest models {[}Figure S7B-C \& S7E-F{]}. This suggests that either the
colon tissue microbiota is study and person dependent or that kit and/or
other types of contamination associated with low biomass samples may be
skewing the results.

\textbf{\emph{CRC Studies are Underpowered for Detecting Small Effect
Sizes:}} Next, we assessed how much confidence should be placed in the
reported outcomes from each individual study by calculating the ability
to detect a difference (power) and sample size needed for small, medium,
and large effect size differences between cases and controls. When
assessing the power of each study at different effect sizes the majority
of studies achieved 80\% power to detect a 30\% or greater difference
between groups {[}Figure 7A \& B{]}. No study that we analyzed had the
standard 80\% power to detect an effect size difference equal to or
below 10\% {[}Figure 7A \& B{]}. In order to achieve a power of 80\%,
for small effect sizes, studies used in our meta-analysis would need to
recruit over 1000 individuals for both the case and control arms
{[}Figure 7C{]}

\newpage

\subsection{Discussion}\label{discussion}

Our study identifies clear differences in diversity, both at the
community level and for individual genera, present in patients with and
without CRC {[}Figure 1-3{]}. Although there was a step-wise decrease in
diversity as disease progressed from control to adenoma to carcinoma,
this did not translate into large effect sizes for the RR of CRC. Even
though mouth-associated genera increased the RR of having a carcinoma,
they did not consistently increase the RR of having an adenoma.
Additionally, our observations suggest that by combining
mouth-associated and CRC protective microbes we can classify either
adenoma or carcinoma stage of disease as well as models that use the
full community.

The data presented herein support the importance of select genera for
carcinoma, but not necessarily adenoma, formation. The results that we
have presented show that both the genera and OTU select and full models,
for the carcinoma stage of CRC, had similar AUCs {[}Figure 4 \& 5{]}.
This suggests that an interplay between a select number of potentially
protective and exacerbating microbes within the GI community is crucial
for carcinoma formation. Importantly, it suggests that there may be key
members of the GI community that might be studied further to potentially
reduce the risk of carcinoma. Conversely, using the present data, it is
clear that new approaches may be needed to identify members of the
community associated with adenoma stage of disease. Regardless of sample
type and whether a full or select model was used, our Random Forest
models consistently performed poorly. Yet, the step-wise decrease in
diversity suggests that the adenoma-associated community is not normal
but has changed subtly {[}Figure 1{]}. This change in diversity, at this
early stage of disease, could be focal to the adenoma itself. One
possible explanation is that how the host interacts with these subtle
changes at early stages of the disease is what leads to a thoroughly
dysfunctional community that is supportive of CRC genesis.

Within stool, common GI microbes were most consistently present in the
top 10 genera or OTUs across studies {[}Figure 6{]}. Changes in
\emph{Bacteroides}, \emph{Ruminococcaceae}, \emph{Ruminococcus}, and
\emph{Roseburia} were consistently found to be in the top 10 most
important variables across the different studies for both adenoma and
carcinoma {[}Figure 6{]}. These data suggest that whether the
non-resident bacterium is \emph{Fusobacteria} or
\emph{Peptostreptococcus} may not be as important as how these bacteria
interact with the changing resident community. Based on these
observations, it is possible to hypothesize that small changes in
community structure lead to new niches in which any one of the
mouth-associated genera can gain a foothold, exacerbating the initial
changes in community and facilitating the transition from adenoma to
carcinoma stage of disease.

The colon tissue-based studies did not provide a clearer understanding
of how the microbiota may be associated with CRC. Generally, the full
OTU-based models of unmatched and matched colon tissue samples were
concordant with stool samples showing that GI resident microbes were the
most prevalent in the top 10 most important variables across study
{[}Figure S7E \& F{]}. Unlike in stool, \emph{Fusobacterium} was the
only mouth-associated bacteria consistently present in the top 10 most
important variables of the full carcinoma stage models {[}Figure S7B-C
\& E-F{]}. The majority of the colon tissue-based results seem to be
study specific with many of the top 10 taxa being present only in a
single study. Additionally, the presence of genera associated with
contamination, within the top 10 most important variables for the genera
and OTU models is worrying. The low bacterial biomass of tissue samples
coupled with potential contamination could explain why these results
seem to be more sporadic than the stool results.

One important caveat to this study is that even though genera associated
with certain species such as \emph{Bacteroides fragilis} and
\emph{Streptococcus gallolyticus} subsp. \emph{gallolyticus} were not
identified, it does not necessarily mean that these specific species are
not important in human CRC {[}20,22{]}. Since we are limited in our
aggregation of the data to the genus level, it is not possible to
clearly delineate which species are contributing to overall disease
progression. Our observations are not inconsistent with the previous
literature on either \emph{Bacteroides fragilis} or \emph{Streptococcus
gallolyticus} subsp. \emph{gallolyticus}. As an example, the stool-based
full community models consistently identified the genus
\emph{Bacteroides}, as well as OTUs that classified as
\emph{Bacteroides}, to be important model components across studies.
This suggests that even though \emph{Bacteroides} may not increase the
RR of CRC and may not vary in relative abundance, like
\emph{Fusobacterium}, it is still important in CRC. Additionally,
\emph{Streptococcus gallolyticus} subsp. \emph{gallolyticus} is a
mouth-associated microbe, and the results from this study suggest that
regardless of sample type, mouth-associated genera are commonly
associated with an increased RR for both adenoma and carcinoma stage of
disease.

The associations between the microbiota and adenoma stage of disease are
inconclusive, in part, because many studies may not be powered
effectively to observe small effect sizes. None of the studies analyzed
were properly powered to detect a 10\% or lower change between cases and
controls. The results within our meta-analysis suggest that a small
effect size may well be the scope in which differences consistently
occur between controls and adenoma stage of disease. Future studies
investigating adenoma stage and the microbiota need to take power into
consideration to reproducibly study whether the microbiota contributes
to polyp formation. In contrast to adenoma stage of disease, our
observations suggest that most studies analyzed have sufficient power to
detect many changes in the carcinoma-associated microbiota because of
large effect size differences between cases and controls {[}Figure 7{]}.

\newpage

\subsection{Conclusion}\label{conclusion}

By aggregating together a large collection of studies analyzing both
fecal and colon tissue samples, we are able to provide evidence
supporting the importance of the bacterial community in CRC. Further,
the data presented here suggests that mouth-associated microbes can gain
a foothold within the colon and are are commonly associated with the
greatest RR of having a carcinoma. No conclusive signal with these
mouth-associated microbes could be detected for adenoma stage of
disease. Our observations also highlight the importance of power and
sample number considerations when investigating the microbiota and
adenoma stage of disease due to the subtle changes in the community.
Overall, associations of microbiota with the carcinoma stage of CRC are
much stronger than those with the adenoma stage.

\newpage

\subsection{Methods}\label{methods}

\textbf{\emph{Obtaining Data Sets:}} The studies used for this
meta-analysis were identified through the review articles written by
Keku, \emph{et al.} and Vogtmann, \emph{et al.} {[}31,32{]} and
additional studies not mentioned in the reviews were obtained based on
the authors' knowledge of the literature. Studies that used tissue or
feces as their sample source for 454 or Illumina 16S rRNA gene
sequencing analysis and had data sets with sequences available for
analysis were included. Some studies were excluded because they did not
have publicly available sequences or did not have metadata in which the
authors were able to share. After these filtering steps, the following
studies remained: Ahn, \emph{et al.} {[}12{]}, Baxter, \emph{et al.}
{[}13{]}, Brim, \emph{et al.} {[}30{]}, Burns, \emph{et al.} {[}16{]},
Chen, \emph{et al.} {[}14{]}, Dejea, \emph{et al.} {[}24{]}, Flemer,
\emph{et al.} {[}18{]}, Geng, \emph{et al.} {[}23{]}, Hale, \emph{et
al.} {[}27{]}, Kostic, \emph{et al.} {[}33{]}, Lu, \emph{et al.}
{[}26{]}, Sanapareddy, \emph{et al.} {[}25{]}, Wang, \emph{et al.}
{[}15{]}, Weir, \emph{et al.} {[}21{]}, and Zeller, \emph{et al.}
{[}17{]}. The Zackular {[}34{]} study was not included because the 90
individuals analyzed within the study are contained within the larger
Baxter study {[}13{]}. After sequence processing, all the case samples
for the Kostic study had 100 or less sequences remaining and was
excluded, leaving a total of 14 studies that analysis could be completed
on.

\textbf{\emph{Data Set Breakdown:}} In total, there were seven studies
with only fecal samples (Ahn, Baxter, Brim, Hale, Wang, Weir, and
Zeller), five studies with only tissue samples (Burns, Dejea, Geng, Lu,
Sanapareddy), and two studies with both fecal and tissue samples (Chen
and Flemer). The total number of individuals analyzed after sequence
processing for feces was 1737 {[}Table 1{]}. The total number of matched
and unmatched tissue samples that were analyzed after sequence
processing was 492 {[}Table 2{]}.

\textbf{\emph{Sequence Processing:}} For the majority of studies, raw
sequences were downloaded from the Sequence Read Archive (SRA)
(\url{ftp://ftp-trace.ncbi.nih.gov/sra/sra-instant/reads/ByStudy/sra/SRP/})
and metadata were obtained by searching the respective accession number
of the study at the following website:
\url{http://www.ncbi.nlm.nih.gov/Traces/study/}. Of the studies that did
not have sequences and metadata on the SRA, data was obtained from DBGap
(n = 1, {[}12{]}) and directly from the authors (n = 4,
{[}18,21,25,27{]}). Each study was processed using the mothur (v1.39.3)
software program {[}35{]} and quality filtering utilized the default
methods for both 454 and Illumina based sequencing. If it was not
possible to use the defaults, the stated quality cut-offs, from the
study itself, were used instead. Sequences that were made up of an
artificial combination of two or more different sequences and commonly
known as chimeras were identified and removed using VSEARCH {[}36{]}
before \emph{de novo} OTU clustering at 97\% similarity was completed
using the OptiClust algorithm {[}37{]}.

\textbf{\emph{Statistical Analysis:}} All statistical analysis after
sequence processing utilized the R (v3.4.3) software package {[}38{]}.
For OTU richness, evenness, and Shannon diversity analysis, values were
power transformed using the rcompanion (v1.11.1) package {[}39{]} and
then Z-score normalized using the car (v2.1.6) package {[}40{]}. Testing
for \(\alpha\)-diversity differences utilized linear mixed-effect models
created using the lme4 (v1.1.15) package {[}41{]} to correct for study,
repeat sampling of individuals (tissue only), and 16S hyper-variable
region used. Relative risk was analyzed using both the epiR (v0.9.93)
and metafor (v2.0.0) packages {[}42,43{]} by assessing how many
individuals with and without disease were above and below the overall
median value within each specific study. Relative risk significance
testing utilized the chi-squared test. \(\beta\)-diversity differences
utilized a Bray-Curtis distance matrix and PERMANOVA executed with the
vegan (v2.4.5) package {[}44{]}. Random Forest models were built using
both the caret (v6.0.78) and randomForest (v4.6.12) packages
{[}45,46{]}. Power analysis and estimations were made using the pwr
(v1.2.1) and statmod (v1.4.30) packages {[}47,48{]}. All figures were
created using both ggplot2 (v2.2.1) and gridExtra (v2.3) packages
{[}49,50{]}.

\textbf{\emph{Study Analysis Overview:}} OTU richness, evenness, and
Shannon diversity was first assessed for differences between controls,
adenoma stage, and carcinoma stage using both linear mixed-effect models
and RR. The Bray-Curtis index was used to assess, for each individual
study, differences between control-adenoma and control-carcinoma. Next,
all common genera were assessed for differences in RR for having an
adenoma or carcinoma and ranked based on P-value. We then built Random
Forest models based on full or selected community, based on the top 5
increased and top 5 decreased RR based on P-value, and these models were
trained on one study then tested on the remaining studies. This process
was repeated for every study in the meta-analysis. A similar approach
was then applied at the OTU level with the exception that a 10-fold CV
over 100 different models, based on random 80/20 splitting of the data,
was used to generate a range of expected AUCs. For these OTU-based
models, the selected model included all OTUs that had a taxonomic
classification to a taxa in the top 5 increased and top 5 decreased RR
based on P-value. Finally, the power of each study was assessed for an
effect size ranging from 1\% to 30\% and an estimated sample size, for
these effect sizes, was generated based on 80\% power. For comparisons
in which only control versus adenoma stage were made, the carcinoma
samples were excluded from each respective study. Similarly, for
comparisons in which control versus carcinoma stage were made the
adenoma samples were excluded from each respective study. The data were
split between feces and tissue samples. Within the tissue groups the
data were further divided between samples from the same individual
(matched) and those from different individuals (unmatched). Where
applicable for each study, predictions for adenoma and carcinoma stage
of disease were then tested for feces, matched tissue, and unmatched
tissue.

\textbf{\emph{Obtaining Genera Relative Abundance and Selected Models:}}
For the genera analysis of the RR, OTUs were added together based on the
genus or lowest available taxonomic classification level and the total
average counts, for 100 different subsamplings, were collected. The OTU
based Random Forest Models using selected OTUs utilized a similar
approach except that the OTUs were not aggregated together by taxonomic
identity but kept as separate OTUs. OTU Random Forest models using the
full community included all OTUs while those for the selected model
included only those OTUs that had a taxonomic classification to a
variable in the top 5 increased of top 5 decreased RR based on P-value.

\textbf{\emph{Matched versus Unmatched Tissue Samples:}} In general,
tissue samples with control and CRC samples from different individuals
were classified as unmatched while samples that belonged to the same
individual were classified as matched. Studies with matched data
included Burns, Dejea, Geng, and Lu while those with unmatched data were
from Burns, Flemer, Chen, and Sanapareddy. For some studies samples
became unmatched when a corresponding matched sample did not make it
through sequence processing. All samples, from both tissue sample types,
were analyzed together for the linear mixed-effect models with samples
from the same individual corrected for. For all other analysis, not
mentioned herein, matched and unmatched samples were analyzed separately
using the statistical approaches mentioned in the Statistical Analysis
section.

\textbf{\emph{Assessing Important Random Forest Model Variables:}} Using
Mean Decrease in Accuracy (MDA) the top 10 most important variables to
the Random Forest model were obtained in two different ways depending on
whether the model used genera or OTU data. For the genus based models,
the number of times that a genus showed up in the top 10 of the training
set across each study was counted while, for the OTU based models, the
medians for each OTU across 100 different 80/20 splits of the data was
generated and the top 10 OTUs then counted for each study. Common taxa,
for the OTU based models, were identified by using the lowest
classification within the RDP database for each of the specific OTUs
obtained from the previous counts and the number of times this
classification occurred in the top 10, in each study, was recorded. The
two studies that had adenoma tissue were equally divided between matched
and unmatched groups and were grouped together for the counting of the
top 10 genera and OTUs.

\textbf{\emph{Reproducible Methods:}} The code and analysis can be found
at
\url{https://github.com/SchlossLab/Sze_CRCMetaAnalysis_Microbiome_2017}.
Unless otherwise mentioned, the accession number of raw sequences from
the studies used in this analysis can be found directly in the
respective batch file in the GitHub repository or in the original
manuscript.

\newpage

\subsection{Declarations}\label{declarations}

\subsubsection{Ethics approval and consent to
participate}\label{ethics-approval-and-consent-to-participate}

Ethics approval and informed consent for each of the studies used is
mentioned in the respective manuscripts used in this meta-analysis.

\subsubsection{Consent for publication}\label{consent-for-publication}

Not applicable.

\subsubsection{Availability of data and
material}\label{availability-of-data-and-material}

A detailed and reproducible description of how the data were processed
and analyzed for each study can be found at
\url{https://github.com/SchlossLab/Sze_CRCMetaAnalysis_Microbiome_2017}.
Raw sequences can be downloaded from the SRA in most cases and can be
found in the respective study batch file in the GitHub repository or
within the original publication. For instances when sequences are not
publicly available, they may be accessed by contacting the corresponding
authors from whence the data came.

\subsubsection{Competing Interests}\label{competing-interests}

All authors declare that they do not have any relevant competing
interests to report.

\subsubsection{Funding}\label{funding}

MAS is supported by a Canadian Institute of Health Research fellowship
and a University of Michigan Postdoctoral Translational Scholar Program
grant.

\subsubsection{Authors' contributions}\label{authors-contributions}

All authors helped to design and conceptualize the study. MAS identified
and analyzed the data. MAS and PDS interpreted the data. MAS wrote the
first draft of the manuscript and both he and PDS reviewed and revised
updated versions. All authors approved the final manuscript.

\subsubsection{Acknowledgements}\label{acknowledgements}

The authors would like to thank all the study participants who were a
part of each of the individual studies utilized. We would also like to
thank each of the study authors for making their data available for use.
Finally, we would like to thank the members of the Schloss lab for
valuable feed back and proof reading during the formulation of this
manuscript.

\newpage

\subsection{References}\label{references}

\hypertarget{refs}{}
\hypertarget{ref-siegel_cancer_2016}{}
1. Siegel RL, Miller KD, Jemal A. Cancer statistics, 2016. CA: a cancer
journal for clinicians. 2016;66:7--30.

\hypertarget{ref-flynn_metabolic_2016}{}
2. Flynn KJ, Baxter NT, Schloss PD. Metabolic and Community Synergy of
Oral Bacteria in Colorectal Cancer. mSphere. 2016;1.

\hypertarget{ref-goodwin_polyamine_2011}{}
3. Goodwin AC, Destefano Shields CE, Wu S, Huso DL, Wu X, Murray-Stewart
TR, et al. Polyamine catabolism contributes to enterotoxigenic
Bacteroides fragilis-induced colon tumorigenesis. Proceedings of the
National Academy of Sciences of the United States of America.
2011;108:15354--9.

\hypertarget{ref-abed_fap2_2016}{}
4. Abed J, Emgård JEM, Zamir G, Faroja M, Almogy G, Grenov A, et al.
Fap2 Mediates Fusobacterium nucleatum Colorectal Adenocarcinoma
Enrichment by Binding to Tumor-Expressed Gal-GalNAc. Cell Host \&
Microbe. 2016;20:215--25.

\hypertarget{ref-arthur_intestinal_2012}{}
5. Arthur JC, Perez-Chanona E, Mühlbauer M, Tomkovich S, Uronis JM, Fan
T-J, et al. Intestinal inflammation targets cancer-inducing activity of
the microbiota. Science (New York, NY). 2012;338:120--3.

\hypertarget{ref-kostic_fusobacterium_2013}{}
6. Kostic AD, Chun E, Robertson L, Glickman JN, Gallini CA, Michaud M,
et al. Fusobacterium nucleatum potentiates intestinal tumorigenesis and
modulates the tumor-immune microenvironment. Cell Host \& Microbe.
2013;14:207--15.

\hypertarget{ref-wu_human_2009}{}
7. Wu S, Rhee K-J, Albesiano E, Rabizadeh S, Wu X, Yen H-R, et al. A
human colonic commensal promotes colon tumorigenesis via activation of T
helper type 17 T cell responses. Nature Medicine. 2009;15:1016--22.

\hypertarget{ref-zackular_manipulation_2016}{}
8. Zackular JP, Baxter NT, Chen GY, Schloss PD. Manipulation of the Gut
Microbiota Reveals Role in Colon Tumorigenesis. mSphere. 2016;1.

\hypertarget{ref-zackular_gut_2013}{}
9. Zackular JP, Baxter NT, Iverson KD, Sadler WD, Petrosino JF, Chen GY,
et al. The gut microbiome modulates colon tumorigenesis. mBio.
2013;4:e00692--00613.

\hypertarget{ref-baxter_structure_2014}{}
10. Baxter NT, Zackular JP, Chen GY, Schloss PD. Structure of the gut
microbiome following colonization with human feces determines colonic
tumor burden. Microbiome. 2014;2:20.

\hypertarget{ref-shah_leveraging_2017}{}
11. Shah MS, DeSantis TZ, Weinmaier T, McMurdie PJ, Cope JL, Altrichter
A, et al. Leveraging sequence-based faecal microbial community survey
data to identify a composite biomarker for colorectal cancer. Gut. 2017;

\hypertarget{ref-ahn_human_2013}{}
12. Ahn J, Sinha R, Pei Z, Dominianni C, Wu J, Shi J, et al. Human gut
microbiome and risk for colorectal cancer. Journal of the National
Cancer Institute. 2013;105:1907--11.

\hypertarget{ref-baxter_microbiota-based_2016}{}
13. Baxter NT, Ruffin MT, Rogers MAM, Schloss PD. Microbiota-based model
improves the sensitivity of fecal immunochemical test for detecting
colonic lesions. Genome Medicine. 2016;8:37.

\hypertarget{ref-chen_human_2012}{}
14. Chen W, Liu F, Ling Z, Tong X, Xiang C. Human intestinal lumen and
mucosa-associated microbiota in patients with colorectal cancer. PloS
One. 2012;7:e39743.

\hypertarget{ref-wang_structural_2012}{}
15. Wang T, Cai G, Qiu Y, Fei N, Zhang M, Pang X, et al. Structural
segregation of gut microbiota between colorectal cancer patients and
healthy volunteers. The ISME journal. 2012;6:320--9.

\hypertarget{ref-burns_virulence_2015}{}
16. Burns MB, Lynch J, Starr TK, Knights D, Blekhman R. Virulence genes
are a signature of the microbiome in the colorectal tumor
microenvironment. Genome Medicine. 2015;7:55.

\hypertarget{ref-zeller_potential_2014}{}
17. Zeller G, Tap J, Voigt AY, Sunagawa S, Kultima JR, Costea PI, et al.
Potential of fecal microbiota for early-stage detection of colorectal
cancer. Molecular Systems Biology. 2014;10:766.

\hypertarget{ref-flemer_tumour-associated_2017}{}
18. Flemer B, Lynch DB, Brown JMR, Jeffery IB, Ryan FJ, Claesson MJ, et
al. Tumour-associated and non-tumour-associated microbiota in colorectal
cancer. Gut. 2017;66:633--43.

\hypertarget{ref-ecoli_Arthur_2014}{}
19. Arthur JC, Gharaibeh RZ, Mühlbauer M, Perez-Chanona E, Uronis JM,
McCafferty J, et al. Microbial genomic analysis reveals the essential
role of inflammation in bacteria-induced colorectal cancer. Nature
Communications {[}Internet{]}. Springer Nature; 2014;5:4724. Available
from: \url{https://doi.org/10.1038/ncomms5724}

\hypertarget{ref-strep_Aymeric_2017}{}
20. Aymeric L, Donnadieu F, Mulet C, Merle L du, Nigro G, Saffarian A,
et al. Colorectal cancer specific conditions promoteStreptococcus
gallolyticusgut colonization. Proceedings of the National Academy of
Sciences {[}Internet{]}. Proceedings of the National Academy of
Sciences; 2017;115:E283--91. Available from:
\url{https://doi.org/10.1073/pnas.1715112115}

\hypertarget{ref-weir_stool_2013}{}
21. Weir TL, Manter DK, Sheflin AM, Barnett BA, Heuberger AL, Ryan EP.
Stool microbiome and metabolome differences between colorectal cancer
patients and healthy adults. PloS One. 2013;8:e70803.

\hypertarget{ref-bfrag_Boleij_2014}{}
22. Boleij A, Hechenbleikner EM, Goodwin AC, Badani R, Stein EM, Lazarev
MG, et al. The bacteroides fragilis toxin gene is prevalent in the colon
mucosa of colorectal cancer patients. Clinical Infectious Diseases
{[}Internet{]}. Oxford University Press (OUP); 2014;60:208--15.
Available from: \url{https://doi.org/10.1093/cid/ciu787}

\hypertarget{ref-geng_diversified_2013}{}
23. Geng J, Fan H, Tang X, Zhai H, Zhang Z. Diversified pattern of the
human colorectal cancer microbiome. Gut Pathogens. 2013;5:2.

\hypertarget{ref-dejea_microbiota_2014}{}
24. Dejea CM, Wick EC, Hechenbleikner EM, White JR, Mark Welch JL,
Rossetti BJ, et al. Microbiota organization is a distinct feature of
proximal colorectal cancers. Proceedings of the National Academy of
Sciences of the United States of America. 2014;111:18321--6.

\hypertarget{ref-sanapareddy_increased_2012}{}
25. Sanapareddy N, Legge RM, Jovov B, McCoy A, Burcal L, Araujo-Perez F,
et al. Increased rectal microbial richness is associated with the
presence of colorectal adenomas in humans. The ISME journal.
2012;6:1858--68.

\hypertarget{ref-lu_mucosal_2016}{}
26. Lu Y, Chen J, Zheng J, Hu G, Wang J, Huang C, et al. Mucosal
adherent bacterial dysbiosis in patients with colorectal adenomas.
Scientific Reports. 2016;6:26337.

\hypertarget{ref-hale_shifts_2017}{}
27. Hale VL, Chen J, Johnson S, Harrington SC, Yab TC, Smyrk TC, et al.
Shifts in the Fecal Microbiota Associated with Adenomatous Polyps.
Cancer Epidemiology, Biomarkers \& Prevention: A Publication of the
American Association for Cancer Research, Cosponsored by the American
Society of Preventive Oncology. 2017;26:85--94.

\hypertarget{ref-Flynn_preprint_2017}{}
28. Flynn KJ, Ruffin MT, Turgeon DK, Schloss PD. Spatial variation of
the native colon microbiota in healthy adults. Cold Spring Harbor
Laboratory; 2017; Available from: \url{https://doi.org/10.1101/189886}

\hypertarget{ref-Salter_contamination_2014}{}
29. Salter SJ, Cox MJ, Turek EM, Calus ST, Cookson WO, Moffatt MF, et
al. Reagent and laboratory contamination can critically impact
sequence-based microbiome analyses. BMC Biology {[}Internet{]}. Springer
Nature; 2014;12. Available from:
\url{https://doi.org/10.1186/s12915-014-0087-z}

\hypertarget{ref-brim_microbiome_2013}{}
30. Brim H, Yooseph S, Zoetendal EG, Lee E, Torralbo M, Laiyemo AO, et
al. Microbiome analysis of stool samples from African Americans with
colon polyps. PloS One. 2013;8:e81352.

\hypertarget{ref-keku_gastrointestinal_2015}{}
31. Keku TO, Dulal S, Deveaux A, Jovov B, Han X. The gastrointestinal
microbiota and colorectal cancer. American Journal of Physiology -
Gastrointestinal and Liver Physiology {[}Internet{]}. 2015 {[}cited 2017
Oct 30{]};308:G351--63. Available from:
\url{http://ajpgi.physiology.org/lookup/doi/10.1152/ajpgi.00360.2012}

\hypertarget{ref-vogtmann_epidemiologic_2016}{}
32. Vogtmann E, Goedert JJ. Epidemiologic studies of the human
microbiome and cancer. British Journal of Cancer {[}Internet{]}. 2016
{[}cited 2017 Oct 30{]};114:237--42. Available from:
\url{http://www.nature.com/doifinder/10.1038/bjc.2015.465}

\hypertarget{ref-kostic_genomic_2012}{}
33. Kostic AD, Gevers D, Pedamallu CS, Michaud M, Duke F, Earl AM, et
al. Genomic analysis identifies association of Fusobacterium with
colorectal carcinoma. Genome Research. 2012;22:292--8.

\hypertarget{ref-zackular_human_2014}{}
34. Zackular JP, Rogers MAM, Ruffin MT, Schloss PD. The human gut
microbiome as a screening tool for colorectal cancer. Cancer Prevention
Research (Philadelphia, Pa). 2014;7:1112--21.

\hypertarget{ref-schloss_introducing_2009}{}
35. Schloss PD, Westcott SL, Ryabin T, Hall JR, Hartmann M, Hollister
EB, et al. Introducing mothur: Open-Source, Platform-Independent,
Community-Supported Software for Describing and Comparing Microbial
Communities. ApplEnvironMicrobiol {[}Internet{]}. 2009 {[}cited 12AD Jan
1{]};75:7537--41. Available from:
\url{http://aem.asm.org/cgi/content/abstract/75/23/7537}

\hypertarget{ref-rognes_vsearch_2016}{}
36. Rognes T, Flouri T, Nichols B, Quince C, Mahé F. VSEARCH: A
versatile open source tool for metagenomics. PeerJ. 2016;4:e2584.

\hypertarget{ref-westcott_opticlust_2017}{}
37. Westcott SL, Schloss PD. OptiClust, an Improved Method for Assigning
Amplicon-Based Sequence Data to Operational Taxonomic Units. mSphere.
2017;2.

\hypertarget{ref-r_citation_2017}{}
38. R Core Team. R: A language and environment for statistical computing
{[}Internet{]}. Vienna, Austria: R Foundation for Statistical Computing;
2017. Available from: \url{https://www.R-project.org/}

\hypertarget{ref-rcompanion_citation_2017}{}
39. Mangiafico S. Rcompanion: Functions to support extension education
program evaluation {[}Internet{]}. 2017. Available from:
\url{https://CRAN.R-project.org/package=rcompanion}

\hypertarget{ref-car_citation_2011}{}
40. Fox J, Weisberg S. An R companion to applied regression
{[}Internet{]}. Second. Thousand Oaks CA: Sage; 2011. Available from:
\url{http://socserv.socsci.mcmaster.ca/jfox/Books/Companion}

\hypertarget{ref-lme4_citation_2015}{}
41. Bates D, Mächler M, Bolker B, Walker S. Fitting linear mixed-effects
models using lme4. Journal of Statistical Software. 2015;67:1--48.

\hypertarget{ref-epir_citation_2017}{}
42. Telmo Nunes MS with contributions from, Heuer C, Marshall J, Sanchez
J, Thornton R, Reiczigel J, et al. EpiR: Tools for the analysis of
epidemiological data {[}Internet{]}. 2017. Available from:
\url{https://CRAN.R-project.org/package=epiR}

\hypertarget{ref-metafor_citation_2010}{}
43. Viechtbauer W. Conducting meta-analyses in R with the metafor
package. Journal of Statistical Software {[}Internet{]}. 2010;36:1--48.
Available from: \url{http://www.jstatsoft.org/v36/i03/}

\hypertarget{ref-vegan_citation_2017}{}
44. Oksanen J, Blanchet FG, Friendly M, Kindt R, Legendre P, McGlinn D,
et al. Vegan: Community ecology package {[}Internet{]}. 2017. Available
from: \url{https://CRAN.R-project.org/package=vegan}

\hypertarget{ref-caret_citation_2017}{}
45. Jed Wing MKC from, Weston S, Williams A, Keefer C, Engelhardt A,
Cooper T, et al. Caret: Classification and regression training
{[}Internet{]}. 2017. Available from:
\url{https://CRAN.R-project.org/package=caret}

\hypertarget{ref-randomforest_citation_2002}{}
46. Liaw A, Wiener M. Classification and regression by randomForest. R
News {[}Internet{]}. 2002;2:18--22. Available from:
\url{http://CRAN.R-project.org/doc/Rnews/}

\hypertarget{ref-pwr_citation_2017}{}
47. Champely S. Pwr: Basic functions for power analysis {[}Internet{]}.
2017. Available from: \url{https://CRAN.R-project.org/package=pwr}

\hypertarget{ref-statmod_citation_2016}{}
48. Giner G, Smyth GK. Statmod: Probability calculations for the inverse
gaussian distribution. R Journal. 2016;8:339--51.

\hypertarget{ref-ggplot2_citation_2009}{}
49. Wickham H. Ggplot2: Elegant graphics for data analysis
{[}Internet{]}. Springer-Verlag New York; 2009. Available from:
\url{http://ggplot2.org}

\hypertarget{ref-gridextra_citation_2017}{}
50. Auguie B. GridExtra: Miscellaneous functions for ``grid'' graphics
{[}Internet{]}. 2017. Available from:
\url{https://CRAN.R-project.org/package=gridExtra}

\newpage

\textbf{Table 1: Total Individuals in each Study Included in the Stool
Analysis}

\footnotesize

\begin{longtable}[]{@{}cccccc@{}}
\toprule
Study & Data Stored & 16S Region & Control (n) & Adenoma (n) & Carcinoma
(n)\tabularnewline
\midrule
\endhead
Ahn & DBGap & V3-4 & 148 & 0 & 62\tabularnewline
Baxter & SRA & V4 & 172 & 198 & 120\tabularnewline
Brim & SRA & V1-3 & 6 & 6 & 0\tabularnewline
Flemer & Author & V3-4 & 37 & 0 & 43\tabularnewline
Hale & Author & V3-5 & 473 & 214 & 17\tabularnewline
Wang & SRA & V3 & 56 & 0 & 46\tabularnewline
Weir & Author & V4 & 4 & 0 & 7\tabularnewline
Zeller & SRA & V4 & 50 & 37 & 41\tabularnewline
\bottomrule
\end{longtable}

\normalsize
\newpage

\textbf{Table 2: Studies with Tissue Samples Included in the Analysis}

\footnotesize

\begin{longtable}[]{@{}cccccc@{}}
\toprule
Study & Data Stored & 16S Region & Control (n) & Adenoma (n) & Carcinoma
(n)\tabularnewline
\midrule
\endhead
Burns & SRA & V5-6 & 18 & 0 & 16\tabularnewline
Chen & SRA & V1-V3 & 9 & 0 & 9\tabularnewline
Dejea & SRA & V3-5 & 31 & 0 & 32\tabularnewline
Flemer & Author & V3-4 & 103 & 37 & 94\tabularnewline
Geng & SRA & V1-2 & 16 & 0 & 16\tabularnewline
Lu & SRA & V3-4 & 20 & 20 & 0\tabularnewline
Sanapareddy & Author & V1-2 & 38 & 0 & 33\tabularnewline
\bottomrule
\end{longtable}

\normalsize
\newpage

\textbf{Figure 1: Community Differences between Control, Adenoma, and
Carcinoma Across Sampling Site.} A) Stool sample community differences
by disease group. B) Unmatched tissue samples differences by disease
group. C) Matched tissue sample differences by group disease group. The
dashed line represents a Z-score of 0 or no difference from the median.

\textbf{Figure 2: Relative Risk for Adenoma or Carcinoma based on
Bacterial Community Metrics in Stool.} A) Community-based relative risk
for adenoma. B) Community-based relative risk for carcinoma. Colors
represent the different variable regions used within the respective
study.

\textbf{Figure 3: Top 5 Genera that Decrease and Increase Relative Risk
for Lesion.} A) Adenoma relative risk in stool. B) Carcinoma relative
risk in stool. C) Adenoma relative risk in tissue. D) Carcinoma relative
risk in tissue. For all panels the relative risk was also compared to
whether one, two, three, or four of the CRC-associated genera were
present.

\textbf{Figure 4: Stool OTU Random Forest Model Across Studies.} A)
Adenoma random forest model between the full and select community OTUs
only. B) Carcinoma random forest model between the full and select
community OTUs only. The dotted line represents an AUC of 0.5 and the
lines represent the range in which the AUC for the 100 different 80/20
runs fell between. The solid red line represents the median AUC of all
the studies for either the full or select community OTUS only model.

\textbf{Figure 5: Tissue OTU Random Forest Model Across Studies.} A)
Adenoma random forest model between the full and select community OTUs
only. B) Carcinoma random forest model between the full and select
community OTUs only. The dotted line represents an AUC of 0.5 and the
lines represent the range in which the AUC for the 100 different 80/20
runs fell between. The solid red line represents the median AUC of all
the studies for either the full community or select genera OTUS only
model.

\textbf{Figure 6: Most Common Genera Across Full Community Stool Study
Models.} A) Common genera in the top 10 for adenoma Random Forest genus
models. B) Common genera in the top 10 for carcinoma Random Forest genus
models. C) Common genera in the top 10 for adenoma Random Forest OTU
models. D) Common genera in the top 10 for carcinoma Random Forest OTU
models.

\textbf{Figure 7: Power and Effect Size Analysis of Studies Included.}
A) Power based on effect size for studies with adenoma individuals. B)
Power based on effect size for studies with carcinoma individuals. C)
The estimated sample number needed for each arm of each study to detect
an effect size of 1-30\%. The dotted red lines in A) and B) represent a
power of 0.8.

\newpage

\textbf{Figure S1: Relative Risk for Adenoma or Carcinoma based on
Bacterial Community Metrics in Tissue.} A) Community-based relative risk
for adenoma. B) Community-based relative risk for carcinoma. Colors
represent the different variable regions used within the respective
study.

\textbf{Figure S2: Stool Random Forest Genus Model AUC for each Study.}
A) AUC of adenoma models using all genera or select genera only. B) AUC
of carcinoma models using all genera or select genera only. The black
line represents the median within each group.

\textbf{Figure S3: Tissue Random Forest Genus Model AUC for each Study.}
A) AUC of adenoma models using all genera or only select genera divided
between matched and unmatched tissue. B) AUC of carcinoma models using
all genera or select genera only. The black line represents the median
within each group divided between matched and unmatched tissue.

\textbf{Figure S4: Stool Random Forest Prediction Success Using Genera
Across Studies.} A) AUC for prediction in adenoma using all genera or
select genera only. B) AUC for prediction in carcinoma using all genera
or select genera only. The dotted line represents an AUC of 0.5. The
x-axis is the data set in which the model was initially trained on. The
red lines represent the median AUC using that specific study as the
training set.

\textbf{Figure S5: Tissue Random Forest Prediction Success of Carcinoma
Using Genera Across Studies.} A) AUC for prediction in unmatched tissue
for all genera or select genera only. B) AUC for prediction in matched
tissue using all genera or select genera only. The dotted line
represents an AUC of 0.5. The x-axis is the data set in which the model
was initially trained on. The red lines represent the median AUC using
that specific study as the training set.

\textbf{Figure S6: Tissue Random Forest Prediction Success of Adenoma
Using Genera Across Studies.} The red lines represent the median AUC
using that specific study as the training set.

\textbf{Figure S7: Most Common Genera Across Full Community Tissue Study
Models.} A) Common genera in the top 10 for adenoma Random Forest genus
models. B) Common genera in the top 10 for unmatched carcinoma Random
Forest genus models. B) Common genera in the top 10 for matched
carcinoma Random Forest genus models. D) Common genera in the top 10 for
adenoma Random Forest OTU models. E) Common genera in the top 10 for
unmatched carcinoma Random Forest OTU models. F) Common genera in the
top 10 for matched carcinoma Random Forest OTU models.

\newpage


\end{document}
