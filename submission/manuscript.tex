\documentclass[12pt,]{article}
\usepackage{lmodern}
\usepackage{amssymb,amsmath}
\usepackage{ifxetex,ifluatex}
\usepackage{fixltx2e} % provides \textsubscript
\ifnum 0\ifxetex 1\fi\ifluatex 1\fi=0 % if pdftex
  \usepackage[T1]{fontenc}
  \usepackage[utf8]{inputenc}
\else % if luatex or xelatex
  \ifxetex
    \usepackage{mathspec}
  \else
    \usepackage{fontspec}
  \fi
  \defaultfontfeatures{Ligatures=TeX,Scale=MatchLowercase}
\fi
% use upquote if available, for straight quotes in verbatim environments
\IfFileExists{upquote.sty}{\usepackage{upquote}}{}
% use microtype if available
\IfFileExists{microtype.sty}{%
\usepackage{microtype}
\UseMicrotypeSet[protrusion]{basicmath} % disable protrusion for tt fonts
}{}
\usepackage[margin=1.0in]{geometry}
\usepackage{hyperref}
\hypersetup{unicode=true,
            pdfborder={0 0 0},
            breaklinks=true}
\urlstyle{same}  % don't use monospace font for urls
\usepackage{graphicx,grffile}
\makeatletter
\def\maxwidth{\ifdim\Gin@nat@width>\linewidth\linewidth\else\Gin@nat@width\fi}
\def\maxheight{\ifdim\Gin@nat@height>\textheight\textheight\else\Gin@nat@height\fi}
\makeatother
% Scale images if necessary, so that they will not overflow the page
% margins by default, and it is still possible to overwrite the defaults
% using explicit options in \includegraphics[width, height, ...]{}
\setkeys{Gin}{width=\maxwidth,height=\maxheight,keepaspectratio}
\IfFileExists{parskip.sty}{%
\usepackage{parskip}
}{% else
\setlength{\parindent}{0pt}
\setlength{\parskip}{6pt plus 2pt minus 1pt}
}
\setlength{\emergencystretch}{3em}  % prevent overfull lines
\providecommand{\tightlist}{%
  \setlength{\itemsep}{0pt}\setlength{\parskip}{0pt}}
\setcounter{secnumdepth}{0}
% Redefines (sub)paragraphs to behave more like sections
\ifx\paragraph\undefined\else
\let\oldparagraph\paragraph
\renewcommand{\paragraph}[1]{\oldparagraph{#1}\mbox{}}
\fi
\ifx\subparagraph\undefined\else
\let\oldsubparagraph\subparagraph
\renewcommand{\subparagraph}[1]{\oldsubparagraph{#1}\mbox{}}
\fi

%%% Use protect on footnotes to avoid problems with footnotes in titles
\let\rmarkdownfootnote\footnote%
\def\footnote{\protect\rmarkdownfootnote}

%%% Change title format to be more compact
\usepackage{titling}

% Create subtitle command for use in maketitle
\newcommand{\subtitle}[1]{
  \posttitle{
    \begin{center}\large#1\end{center}
    }
}

\setlength{\droptitle}{-2em}
  \title{}
  \pretitle{\vspace{\droptitle}}
  \posttitle{}
  \author{}
  \preauthor{}\postauthor{}
  \date{}
  \predate{}\postdate{}

\usepackage{helvet} % Helvetica font
\renewcommand*\familydefault{\sfdefault} % Use the sans serif version of the font
\usepackage[T1]{fontenc}

\usepackage[none]{hyphenat}

\usepackage{setspace}
\doublespacing
\setlength{\parskip}{1em}

\usepackage{lineno}

\usepackage{pdfpages}

\begin{document}

\section{The Microbiota and Individual Community Members in Colorectal
Cancer: Is There a Common
Theme?}\label{the-microbiota-and-individual-community-members-in-colorectal-cancer-is-there-a-common-theme}

\begin{center}
\vspace{25mm}

Marc A Sze${^1}$ and Patrick D Schloss${^1}$${^\dagger}$

\vspace{20mm}

$\dagger$ To whom correspondence should be addressed: pschloss@umich.edu

$1$ Department of Microbiology and Immunology, University of Michigan, Ann Arbor, MI




\end{center}

Co-author e-mails:

\begin{itemize}
\tightlist
\item
  \href{mailto:marcsze@med.umich.edu}{\nolinkurl{marcsze@med.umich.edu}}
\end{itemize}

\newpage

\linenumbers

\subsection{Abstract}\label{abstract}

\textbf{Background.}

\textbf{Results.}

\textbf{Conclusions.}

\subsubsection{Keywords}\label{keywords}

microbiota; colorectal cancer; polyps; adenoma; meta-analysis.

\newpage

\subsection{Background}\label{background}

\newpage

\subsection{Results}\label{results}

\textbf{\emph{Fecal Diversity is Lower in Those with Carcinoma and
Increases Relative Risk:}} Using power transformed and Z-score
normalized alpha diversity metrics both evenness and the Shannon
diversity metrics in feces are lower in those with carcinoma then in
controls but not for tissue samples {[}Figure 1{]}. Using linear
mixed-effects to control for study and variable region there was a
significant decrease from control to adenoma to carcinoma for both
evenness (P-value = 0.025) and Shannon diversity (P-value = 0.043). This
effect was not observed in tissue when additionally controlling for
whether the sample came from the same individual (P-value \textgreater{}
0.05). For fecal samples a decrease in Shannon diversity and evenness
resulted in a significantly increased relative risk for carcinoma
(P-value = 0.01 and P-value = 0.0011, respectively) {[}Figure 2{]}.
Although these values were significant the effect size was relatively
small for both metrics (Shannon RR = 1.31 and evenness RR = 1.34)
{[}Figure 2{]}. There was no increased relative risk for these metrics
for adenoma or for tissue in general {[}Figure S1-3{]}.

Using the Bray-Curtis distance metric, the fecal microbiota did not have
a different community diversity between adenoma and control but did for
carcinoma across studies {[}Table S1 \& S2{]}. The majority of unmatched
tissue samples had a significant difference for both adenoma and
carcinoma versus controls {[}Table S3 \& S4{]}. All matched tissue
samples accross studies had no difference between any of the compared
groups {[}Table S3 \& S4{]}.

\textbf{\emph{Genera Previously Associated with Carcinoma Increases
Relative Risk More than Alpha Diversity:}} Both fecal and tissue samples
had a significantly increased RR for carcinoma but not for adenoma
{[}Figure 3{]} which was greater than either evenness or Shannon
diversity {[}Figure 2 \& 3{]}. The relative risk did not increase when
considering the total abundance or increasing number of carcinoma
associated genera {[}Figure 3{]}. The RR effect size was greater for
stool (RR range = 1.78 - 2.64) then for tissue (RR range = 1.33 - 1.53).
This decrease may be explained by the fact that tissue samples include
matched samples.

\textbf{\emph{Using the Whole Community Increases Model AUC over CRC
Associated Genera:}} For both fecal and tissue samples (matched and
unmatched) there was a decrease in AUC when only OTUs from the CRC
associated genera are used {[}Figure 4 \& 5{]}. This decrease is
observed in both adenoma and carcinoma groups {[}Figure 4 \& 5{]}. The
genus models generally had similar trends as observed for the OTU based
models with the full genera models performing better then the CRC
associated genera models {[}Figure S4-S5{]}. Both genus models perform
similarily in their ability to be able to predict lesion (adenoma or
carcinoma) with carcinoma having a higher AUC then adenoma {[}Figure
S6-S8{]}. Matched tissue samples for those with carcinoma had an AUC
that was more similar to the adenoma models {[}Figure S6A, S7B, \& S8{]}
then carcinoma models {[}Figure S6B \& S7A{]}.

\textbf{\emph{Majority of Studies are Underpowered for Detecting Small
Effect Size Differences:}} When assessing the power of each study at
different effect sizes the majority of studies for both adenoma and
carcinoma have an 80\% power to detect a 30\% difference {[}Figure 6A \&
B{]}. No single study that was analyzed had the standard 80\% power to
detect a difference that was eqaul to or below 10\% {[}Figure 6A \&
B{]}. In order to achieve adequate power for small effect sizes it would
be necessary to recruit over 1000 individuals for each arm of the study
{[}Figure 6C{]}.

\newpage

\subsection{Discussion}\label{discussion}

Our study identifies clear diversity changes both at the community level
and within individual genera that are present in indivdiuals with
carcinoma versus those without the disease. Although there was a step
wise decrease in diversity from control to adenoma to carcinoma; this
did not translate into large effect sizes for the relative risk of
either of these two conditions. These clear changes were not easily
recapitulated in those with adenoma. Even though CRC associated genera
increase the relative risk of carcinoma they do not increase the
relative risk of adenoma. This information suggests that these specific
genera may not be the primary members of the microbial community that
contributes to the formation of an adenoma but is for a carcinoma.
Additionally, our data shows that by using the whole community our
models perform better then when they only use the CRC associated genera.
CRC associated genera are clearly important to carcinoma but the context
or community in which these microbes are a part of can drastically
increase the ability of models to make predictions. This data supports
the concept that small localized changes within the community may be
occuring that are important in the disease progression of colorectal
cancer and that they may not directly involve CRC associated genera.

The driver-passenger model of the microbial role in CRC, as summarized
by Flynn {[}1{]}, can be supported with this data for carcinoma but not
necessarily for adenoma. The drasitically increased relative risk of
disease when considering the CRC associated genera is highly supportive
of this type of process. In a driver-passenger scenario it is possible
that simply having the driver present or only identifying the passenger
is a good enough proxy that the event is occuring. This would account
for the observation that there is no synergistic incrase in relative
risk when accounting for either the total number or increasing abundance
of these genera. The initial establishment of the driver within the
system is also dependent on the community that is present and this could
explain why adding the community context to our models increases the
prediction. Although the driver-passenger model fits well with the
transtion of adenoma to carcinoma for our carcinoma observations it does
not fit well with our observed data on control to adenoma.

\begin{itemize}
\item
  Builiding on this idea moving forward how CRC associated genera
  interact with the overall community rather than in isolation will be
  important since they do not increase the RR of polyps
\item
  suggesting localized changes to community may be more important
\item
  In closing to get at these more difficult questions the Power
  calculations and effect size in future studies need to be accounted
  for especially for adenoma where the predicted effect sizes are going
  to be small
\end{itemize}

\subsection{Conclusion}\label{conclusion}

-aggregate studies done with feces and stool to investigate the role of
the microbiota in adenoma and carcinoma.

\begin{itemize}
\item
  Significantly adds to the existing research using this approach by
  expanding both the total individuals included and the scope.
\item
  leads to clear identifications of the challenges that lie ahead with
  respect to adenoma
\end{itemize}

\newpage

\subsection{Methods}\label{methods}

\textbf{\emph{Obtaining Data Sets:}} Studies used for this meta-analysis
were identified through the review articles written by Keku, et al. and
Vogtmann, et al. {[}2,3{]}. All studies were included that used tissue
or feces as their sample source for 16S rRNA gene sequencing analysis.
Studies using either 454 or Illumina sequencing technology were
included. Only data sets that had the raw sequences available for
analysis were included. Some studies did not have publically available
raw sequences or did not have meta data in which the authors were able
to share. After this filtering step the following studies remained: Ahn
{[}4{]}, Baxter {[}5{]}, Brim {[}6{]}, Burns {[}7{]}, Chen {[}8{]},
Dejea {[}9{]}, Flemer {[}10{]}, Geng {[}11{]}, Hale {[}12{]}, Kostic
{[}13{]}, Lu {[}14{]}, Sanapareddy {[}15{]}, Wang {[}16{]}, Weir
{[}17{]}, and Zeller {[}18{]}. The Zackular {[}19{]} study was not
included becasue the 90 individuals analyzed within the study are
contained within the larger Baxter study. The Kostic study was not used
since after sequence processing all the case samples did not have more
than 100 sequences remaining. This left a total of 13 studies in which
complete analysis could be completed.

\textbf{\emph{Data Set Breakdown:}} In total there were 7 studies with
only fecal samples (Ahn, Baxter, Brim, Hale, Wang, Weir, and Zeller), 5
studies with only tissue samples (Burns, Dejea, Geng, Lu, Sanapareddy),
and 2 studies with both fecal and tissue samples (Chen and Flemer). The
total number of individuals initially run through the sequence
processing for the fecal samples was 1899 and for the tissue samples was
462.

\textbf{\emph{Sequence Processing:}} For the majority of studies raw
sequences were downloaded from the SRA
(\url{ftp://ftp-trace.ncbi.nih.gov/sra/sra-instant/reads/ByStudy/sra/SRP/})
and metadata was obtained from the following website:
\url{http://www.ncbi.nlm.nih.gov/Traces/study/} by searching the
respective accession number of the study. Of the studies that did not
have sequences and meta data on the SRA one study had the data stored on
DBGap {[}4{]} and four studies the data was obtained directly from the
authors {[}10,12,15,17{]}. Each study was processed using the mothur
(v1.39.3) software program {[}20{]}. Where possible quality filtering
utilized the default methods used in mothur for either 454 or Illumina
based sequencing. If it was not possible to use these defaults the
author stated quality cut-offs were used instead. Chimeras were
identifed and removed using the VSEARCH {[}{\textbf{???}}{]} program and
\emph{de novo} OTU clustering at 97\% similarity using the OptiClust
algorithm {[}21{]} was utilized.

\textbf{\emph{Statistical Analysis:}} All statistical analysis after
sequence processing utilized the R software package (v3.4.2). For the
alpha diversity analysis values were power transformed using the
rcompanion (v1.10.1) package and then Z-score normalized using the car
(v2.1.5) package. Testing for alpha diversity differences utilized
linear mixed-effect models created using the lme4 (v1.1.14) package to
correct for both study and variable region effect in the diversity
measures when analyzing colorectal cancer groups. Relative Risk was
analyzed using both the epiR (v0.9.87) and metafor (v2.0.0) packages.
Relative risk significance testing utilized the chi-squred test.
Beta-diversity differences utilized a Bray-Curtis distance matrix and
PERMANOVA executed with the vegan (v2.4.4) package. Random Forest models
were built using both the caret (v6.0.77) and randomForest (v4.6.12)
packages. Random Forest testing of the obtained AUC versus a random
model AUC utilized T-tests. Power analysis and estimations were made
using the pwr (v1.2.1) and statmod (v1.4.30) packages. All figures were
created using both ggplot2 (v2.2.1) and gridExtra (v2.3) packages.

\textbf{\emph{Study Analysis Overview:}} Alpha diversity was first
assessed for differences between controls and adenoma versus cancer and
controls versus adenoma. We analyzed the data using linear mixed-effect
models, and relative risk. Beta-diversity was then assessed for each
inidividual study. Next, four specific CRC-associated genera
(\emph{Fusobacterium}, \emph{Parvimonas}, \emph{Peptostreptococcus}, and
\emph{Porphyromonas}) were assessed for differences in relative risk. We
then built Random Forest models based on all genera or the select
CRC-associated genera. The models were trained on one study then tested
on the remaining studies for every study. The data was split between
feces and tissue samples. Within the tissue groups the data was further
divided between matched and unmatched tissue samples. Both prediction
for adenoma and carcinoma were tested. This same approach was then
applied at the OTU level with the exception that instead of testing on
the other studies a 10-fold cross validation was utilized and 100
different models were created based on random 80/20 splitting of the
data to generate a range of expected AUCs. For OTU based models the CRC
Associated Genera included all OTUs that had a taxonomic classification
to \emph{Fusobacterium}, \emph{Parvimonas}, \emph{Peptostreptococcus},
or \emph{Porphyromonas}. The power of each study was assessed for and
effect size ranging from 1\% to 30\%. An estimated sample n for these
effect sizes was also generated based on 80\% power.

\textbf{\emph{Reproducible Methods:}} The code and analysis can be found
here
\url{https://github.com/SchlossLab/Sze_CRCMetaAnalysis_Microbiome_2017}.
Unless mentioned otherwise the accession number for the raw sequences
for the studies used in this analysis can be found directly in the
respective batch file, on the GitHub repository or in the original
manuscript.

\newpage

\subsection{Declarations}\label{declarations}

\subsubsection{Ethics approval and consent to
participate}\label{ethics-approval-and-consent-to-participate}

Ethics approval and informed consent for each of the studies used is
mentioned in the respective manuscript used in this meta-analysis.

\subsubsection{Consent for publication}\label{consent-for-publication}

Not applicable.

\subsubsection{Availability of data and
material}\label{availability-of-data-and-material}

A detailed and reporducible description of how the data were processed
and analyzed for each study can be found at
\url{https://github.com/SchlossLab/Sze_CRCMetaAnalysis_Microbiome_2017}.
Raw sequences can be downloaded from the SRA in most cases and can be
found in the respective studies batch file in the GitHub repo or within
the original publication. When sequences were not publicly available
contacting the corresponding author for raw sequences needs to be
undertaken.

\subsubsection{Competing Interests}\label{competing-interests}

All authors declare that they do not have any relevant competing
interests to report.

\subsubsection{Funding}\label{funding}

MAS is supported by a CIHR fellowship and a University of Michigan PTSP
fellowship grant.

\subsubsection{Authors' contributions}\label{authors-contributions}

All authors helped to design and conceptualize the study. MAS identified
and analyzed the data. MAS and PDS interpreted the data. MAS wrote the
first draft of the manuscript and both he and PDS reviewed and revised
updated versions. All authors approved the final manuscript.

\subsubsection{Acknowledgements}\label{acknowledgements}

The authors would like to thank all the study participants who were
apart of each of the individual studies uitlized. We would also like to
thank each of the study authors for making their data available for use.
Finally we would like to thank the members of the Schloss lab for
valuable feed back and proof reading during the formulation of this
manuscript.

\newpage

\subsection{References}\label{references}

\hypertarget{refs}{}
\hypertarget{ref-flynn_metabolic_2016}{}
1. Flynn KJ, Baxter NT, Schloss PD. Metabolic and Community Synergy of
Oral Bacteria in Colorectal Cancer. mSphere. 2016;1.

\hypertarget{ref-keku_gastrointestinal_2015}{}
2. Keku TO, Dulal S, Deveaux A, Jovov B, Han X. The gastrointestinal
microbiota and colorectal cancer. American Journal of Physiology -
Gastrointestinal and Liver Physiology {[}Internet{]}. 2015 {[}cited 2017
Oct 30{]};308:G351--63. Available from:
\url{http://ajpgi.physiology.org/lookup/doi/10.1152/ajpgi.00360.2012}

\hypertarget{ref-vogtmann_epidemiologic_2016}{}
3. Vogtmann E, Goedert JJ. Epidemiologic studies of the human microbiome
and cancer. British Journal of Cancer {[}Internet{]}. 2016 {[}cited 2017
Oct 30{]};114:237--42. Available from:
\url{http://www.nature.com/doifinder/10.1038/bjc.2015.465}

\hypertarget{ref-ahn_human_2013}{}
4. Ahn J, Sinha R, Pei Z, Dominianni C, Wu J, Shi J, et al. Human gut
microbiome and risk for colorectal cancer. Journal of the National
Cancer Institute. 2013;105:1907--11.

\hypertarget{ref-baxter_microbiota-based_2016}{}
5. Baxter NT, Ruffin MT, Rogers MAM, Schloss PD. Microbiota-based model
improves the sensitivity of fecal immunochemical test for detecting
colonic lesions. Genome Medicine. 2016;8:37.

\hypertarget{ref-brim_microbiome_2013}{}
6. Brim H, Yooseph S, Zoetendal EG, Lee E, Torralbo M, Laiyemo AO, et
al. Microbiome analysis of stool samples from African Americans with
colon polyps. PloS One. 2013;8:e81352.

\hypertarget{ref-burns_virulence_2015}{}
7. Burns MB, Lynch J, Starr TK, Knights D, Blekhman R. Virulence genes
are a signature of the microbiome in the colorectal tumor
microenvironment. Genome Medicine. 2015;7:55.

\hypertarget{ref-chen_human_2012}{}
8. Chen W, Liu F, Ling Z, Tong X, Xiang C. Human intestinal lumen and
mucosa-associated microbiota in patients with colorectal cancer. PloS
One. 2012;7:e39743.

\hypertarget{ref-dejea_microbiota_2014}{}
9. Dejea CM, Wick EC, Hechenbleikner EM, White JR, Mark Welch JL,
Rossetti BJ, et al. Microbiota organization is a distinct feature of
proximal colorectal cancers. Proceedings of the National Academy of
Sciences of the United States of America. 2014;111:18321--6.

\hypertarget{ref-flemer_tumour-associated_2017}{}
10. Flemer B, Lynch DB, Brown JMR, Jeffery IB, Ryan FJ, Claesson MJ, et
al. Tumour-associated and non-tumour-associated microbiota in colorectal
cancer. Gut. 2017;66:633--43.

\hypertarget{ref-geng_diversified_2013}{}
11. Geng J, Fan H, Tang X, Zhai H, Zhang Z. Diversified pattern of the
human colorectal cancer microbiome. Gut Pathogens. 2013;5:2.

\hypertarget{ref-hale_shifts_2017}{}
12. Hale VL, Chen J, Johnson S, Harrington SC, Yab TC, Smyrk TC, et al.
Shifts in the Fecal Microbiota Associated with Adenomatous Polyps.
Cancer Epidemiology, Biomarkers \& Prevention: A Publication of the
American Association for Cancer Research, Cosponsored by the American
Society of Preventive Oncology. 2017;26:85--94.

\hypertarget{ref-kostic_genomic_2012}{}
13. Kostic AD, Gevers D, Pedamallu CS, Michaud M, Duke F, Earl AM, et
al. Genomic analysis identifies association of Fusobacterium with
colorectal carcinoma. Genome Research. 2012;22:292--8.

\hypertarget{ref-lu_mucosal_2016}{}
14. Lu Y, Chen J, Zheng J, Hu G, Wang J, Huang C, et al. Mucosal
adherent bacterial dysbiosis in patients with colorectal adenomas.
Scientific Reports. 2016;6:26337.

\hypertarget{ref-sanapareddy_increased_2012}{}
15. Sanapareddy N, Legge RM, Jovov B, McCoy A, Burcal L, Araujo-Perez F,
et al. Increased rectal microbial richness is associated with the
presence of colorectal adenomas in humans. The ISME journal.
2012;6:1858--68.

\hypertarget{ref-wang_structural_2012}{}
16. Wang T, Cai G, Qiu Y, Fei N, Zhang M, Pang X, et al. Structural
segregation of gut microbiota between colorectal cancer patients and
healthy volunteers. The ISME journal. 2012;6:320--9.

\hypertarget{ref-weir_stool_2013}{}
17. Weir TL, Manter DK, Sheflin AM, Barnett BA, Heuberger AL, Ryan EP.
Stool microbiome and metabolome differences between colorectal cancer
patients and healthy adults. PloS One. 2013;8:e70803.

\hypertarget{ref-zeller_potential_2014}{}
18. Zeller G, Tap J, Voigt AY, Sunagawa S, Kultima JR, Costea PI, et al.
Potential of fecal microbiota for early-stage detection of colorectal
cancer. Molecular Systems Biology. 2014;10:766.

\hypertarget{ref-zackular_human_2014}{}
19. Zackular JP, Rogers MAM, Ruffin MT, Schloss PD. The human gut
microbiome as a screening tool for colorectal cancer. Cancer Prevention
Research (Philadelphia, Pa.). 2014;7:1112--21.

\hypertarget{ref-schloss_introducing_2009}{}
20. Schloss PD, Westcott SL, Ryabin T, Hall JR, Hartmann M, Hollister
EB, et al. Introducing mothur: Open-Source, Platform-Independent,
Community-Supported Software for Describing and Comparing Microbial
Communities. Appl.Environ.Microbiol. {[}Internet{]}. 2009 {[}cited 12AD
Jan 1{]};75:7537--41. Available from:
\url{http://aem.asm.org/cgi/content/abstract/75/23/7537}

\hypertarget{ref-westcott_opticlust_2017}{}
21. Westcott SL, Schloss PD. OptiClust, an Improved Method for Assigning
Amplicon-Based Sequence Data to Operational Taxonomic Units. mSphere.
2017;2.

\newpage

\textbf{Table 1:}

\newpage

\textbf{Figure 1: }

\textbf{Figure 2: }

\textbf{Figure 3: }

\textbf{Figure 4: }

\textbf{Figure 5: }

\textbf{Figure 6: }

\newpage

\textbf{Figure S1: }

\textbf{Figure S2: }

\textbf{Figure S3: }

\textbf{Figure S4: }

\textbf{Figure S5: }

\textbf{Figure S6: }

\textbf{Figure S7: }

\textbf{Figure S8: }

\newpage


\end{document}
