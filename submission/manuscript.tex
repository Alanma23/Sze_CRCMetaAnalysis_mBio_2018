\documentclass[12pt,]{article}
\usepackage{lmodern}
\usepackage{amssymb,amsmath}
\usepackage{ifxetex,ifluatex}
\usepackage{fixltx2e} % provides \textsubscript
\ifnum 0\ifxetex 1\fi\ifluatex 1\fi=0 % if pdftex
  \usepackage[T1]{fontenc}
  \usepackage[utf8]{inputenc}
\else % if luatex or xelatex
  \ifxetex
    \usepackage{mathspec}
  \else
    \usepackage{fontspec}
  \fi
  \defaultfontfeatures{Ligatures=TeX,Scale=MatchLowercase}
\fi
% use upquote if available, for straight quotes in verbatim environments
\IfFileExists{upquote.sty}{\usepackage{upquote}}{}
% use microtype if available
\IfFileExists{microtype.sty}{%
\usepackage{microtype}
\UseMicrotypeSet[protrusion]{basicmath} % disable protrusion for tt fonts
}{}
\usepackage[margin=1.0in]{geometry}
\usepackage{hyperref}
\hypersetup{unicode=true,
            pdfborder={0 0 0},
            breaklinks=true}
\urlstyle{same}  % don't use monospace font for urls
\usepackage{longtable,booktabs}
\usepackage{graphicx,grffile}
\makeatletter
\def\maxwidth{\ifdim\Gin@nat@width>\linewidth\linewidth\else\Gin@nat@width\fi}
\def\maxheight{\ifdim\Gin@nat@height>\textheight\textheight\else\Gin@nat@height\fi}
\makeatother
% Scale images if necessary, so that they will not overflow the page
% margins by default, and it is still possible to overwrite the defaults
% using explicit options in \includegraphics[width, height, ...]{}
\setkeys{Gin}{width=\maxwidth,height=\maxheight,keepaspectratio}
\IfFileExists{parskip.sty}{%
\usepackage{parskip}
}{% else
\setlength{\parindent}{0pt}
\setlength{\parskip}{6pt plus 2pt minus 1pt}
}
\setlength{\emergencystretch}{3em}  % prevent overfull lines
\providecommand{\tightlist}{%
  \setlength{\itemsep}{0pt}\setlength{\parskip}{0pt}}
\setcounter{secnumdepth}{0}
% Redefines (sub)paragraphs to behave more like sections
\ifx\paragraph\undefined\else
\let\oldparagraph\paragraph
\renewcommand{\paragraph}[1]{\oldparagraph{#1}\mbox{}}
\fi
\ifx\subparagraph\undefined\else
\let\oldsubparagraph\subparagraph
\renewcommand{\subparagraph}[1]{\oldsubparagraph{#1}\mbox{}}
\fi

%%% Use protect on footnotes to avoid problems with footnotes in titles
\let\rmarkdownfootnote\footnote%
\def\footnote{\protect\rmarkdownfootnote}

%%% Change title format to be more compact
\usepackage{titling}

% Create subtitle command for use in maketitle
\newcommand{\subtitle}[1]{
  \posttitle{
    \begin{center}\large#1\end{center}
    }
}

\setlength{\droptitle}{-2em}
  \title{}
  \pretitle{\vspace{\droptitle}}
  \posttitle{}
  \author{}
  \preauthor{}\postauthor{}
  \date{}
  \predate{}\postdate{}

\usepackage{helvet} % Helvetica font
\renewcommand*\familydefault{\sfdefault} % Use the sans serif version of the font
\usepackage[T1]{fontenc}

\usepackage[none]{hyphenat}

\usepackage{setspace}
\doublespacing
\setlength{\parskip}{1em}

\usepackage{lineno}

\usepackage{pdfpages}

\usepackage{amsmath}

\usepackage{mathtools}

\begin{document}

\section{Making Sense of the Noise: Leveraging Existing 16S rRNA Gene
Surveys to Identify Key Community Members in Colorectal
Tumors}\label{making-sense-of-the-noise-leveraging-existing-16s-rrna-gene-surveys-to-identify-key-community-members-in-colorectal-tumors}

\begin{center}
\vspace{25mm}

Marc A Sze${^1}$ and Patrick D Schloss${^1}$${^\dagger}$

\vspace{20mm}

$\dagger$ To whom correspondence should be addressed: pschloss@umich.edu

$1$ Department of Microbiology and Immunology, University of Michigan, Ann Arbor, MI




\end{center}

Co-author e-mails:

\begin{itemize}
\tightlist
\item
  \href{mailto:marcsze@med.umich.edu}{\nolinkurl{marcsze@med.umich.edu}}
\end{itemize}

\newpage

\linenumbers

\subsection{Abstract}\label{abstract}

\textbf{Background.} An increasing body of literature suggests that both
individual and collections of bacteria are associated with the
progression of colorectal cancer. As the number of studies investigating
these associations increases and the number of subjects in each study
increases, a meta-analysis to identify the associations that are the
most predictive of disease progression is warranted. For our
meta-analysis, we analyzed previously published 16S rRNA gene sequencing
data collected from feces (1737 individuals from 8 studies) and colon
tissue (492 total samples from 350 individuals from 7 studies).

\textbf{Results.} We quantified the odds ratios for individual bacterial
genera that were associated with an individual having tumors relative to
a normal colon. Among the stool samples, there were no genera that had a
significant odds ratio associated with adenoma and there were 8 genera
with significant odds ratios (ORs) associated with carcinoma. Similarly,
among the tissue samples, there were no genera that had a significant
odds ratio associated with adenoma and there were 3 genera with
significant odds ratios associated with carcinoma. Among the significant
odds ratios, the association between individual taxa and tumor diagnosis
was equal or below 7.11. Because individual taxa had limited association
with tumor diagnosis, we trained Random Forest classification models
using only the taxa that had significant ORs, using the entire
collection of taxa found in each study, and using operational taxonomic
units defined based on a 97\% similarity threshold. All training
approaches yielded similar classification success as measured using the
Area Under the Curve. The ability to correctly classify individuals with
adenomas was poor and the ability to classify individuals with
carcinomas was considerably better using sequences from stool or tissue.

\newpage

\textbf{Importance.} This meta-analysis confirms previous results
indicating that individuals with adenomas cannot be readily classified
based on their bacterial community, but that those with carcinomas can.
Regardless of the dataset, we found a subset of the fecal community that
was associated with carcinomas was as predictive as the full community.

\subsubsection{Keywords}\label{keywords}

microbiota; colorectal cancer; polyps; adenoma; tumor; meta-analysis.

\newpage

\subsection{Background}\label{background}

Colorectal cancer (CRC) is a growing world-wide health problem in which
the microbiota has been hypothesized to have a role in disease
progression (1, 2). Numerous studies using murine models of CRC have
shown the importance of both individual microbes (3--7) and the overall
community (8--10) in tumorigenesis. Numerous case-control studies have
characterized the microbiota of individuals with colonic adenomas and
carcinomas in an attempt to identify biomarkers of disease progression
(6, 11--17). Because current CRC screening recommendations are poorly
adhered to due to socioeconomic status, test invasiveness, and frequency
of tests, development and validation of microbiome-associated biomarkers
for CRC progression could further attempts to develop non-invasive
diagnostics (18).

Recently, there has been an intense focus on identifying
microbiota-based biomarker yielding a seemingly endless number of
candidate taxa. Some studies point towards mouth-associated genera such
as Fusobacterium, Peptostreptococcus, Parvimonas, and Porphyromonas that
are enriched in people with carcinomas (6, 11--17). Other studies have
identified members of \emph{Akkermansia}, \emph{Bacteroides},
\emph{Enterococcus}, \emph{Escherichia}, \emph{Klebsiella},
\emph{Mogibacterium}, \emph{Streptococcus}, and \emph{Providencia} are
also associated with carciomas (13--15). Additionally, \emph{Roseburia}
has been found in some studies to be more abundant in people with tumors
but in other studies it has been found to be either less abundant or no
different than what is found in subjects with normal colons (14, 17, 19,
20). There are strong results from tissue culture and murine models that
Fusobacterium nucleatum, pks-positive strains of Escherichia coli,
Streptococcus gallolyticus, and an entertoxin-producing strain of
Bacteroides fragilis are important in the pathogenesis of CRC (5, 14,
21--24). These results point to a causative role for the microbiota in
CRC pathogenesis as well as their potential as diagnostic biomarkers.

Most studies have focused on identifying biomarkers in patients with
carcinomas but there is a greater clinical need to identify biomarkers
associated with adenomas. Studies focusing on broad scale community
metrics have found that measures such as the total number of Operational
Taxonomic Units (OTUs) are decreased in those with adenomas versus
controls (25). Other studies have identified \emph{Acidovorax},
\emph{Bilophila}, \emph{Cloacibacterium}, \emph{Desulfovibrio},
\emph{Helicobacter}, \emph{Lactobacillus}, \emph{Lactococcus},
\emph{Mogibacterium}, and \emph{Pseudomonas} to be enriched in those
with adenomas (25--27). There are few genera that are enriched in
patients with adenoma or carcinoma tumors.

Confirming some of these previous findings, a recent meta-analysis found
that 16S rRNA gene sequences from members of the \emph{Akkermansia},
\emph{Fusobacterium}, and \emph{Parvimonas} were fecal biomarkers for
the presence of carcinomas (28). Contrary to previous studies they found
sequences similar to members of \emph{Lactobacillus} and
\emph{Ruminococcus} to be enriched in patients with adenoma or carcinoma
relative to those with normal colons (12, 15, 16). In addition, they
found 16S rRNA gene sequences from members of \emph{Haemophilus},
\emph{Methanosphaera}, \emph{Prevotella}, \emph{Succinovibrio} were
enriched in patients with adenoma and \emph{Pantoea} were enriched in
patients with carcinomas. Although this meta-analysis was helpful for
distilling a large number of possible biomarkers, the aggregate number
of samples included in the analysis (n = 509) was smaller than several
larger case-control studies that have since been published (12, 27)

Here we provide an updated meta-analysis using 16S rRNA gene sequence
data from both feces (n = 1737) and colon tissue (492 samples from 350
individuals) from 14 studies (11--17, 19, 20, 23, 25--27, 29) {[}Table 1
\& 2{]}. We expand both the breadth and scope of the previous
meta-analysis to investigate whether biomarkers describing the bacterial
community or specific members of the community can more accurately
classify patients as having adenoma or carcinoma. Our results suggest
that the bacterial community changes as disease severity worsens and
that that a subset of the microbial community can be used to diagnose
the presence of carcinoma.

\newpage

\subsection{Results}\label{results}

\textbf{\emph{Lower bacterial diversity is associated with increased
odds ratio (OR) of tumors:}} We first assessed whether variation in
broad community metrics like total number of operational taxonomic units
(OTUs) (i.e.~richness), the evenness of their abundance, and the overall
diversity was associated with disease stage after controlling for study
and variable region differences. In stool, there was a significant
decrease in both evenness and diversity as disease severity progressed
from normal to adenoma to carcinoma (P-value = 0.025 and 0.043,
respectively) {[}Figure 1{]}; there was not a significant difference for
richness (P-value = 0.21). We next tested whether the decrease in these
community metrics translated into significant ORs for having an adenoma
or carcinoma. For fecal samples, the ORs for richness were not
significantly greater than 1.0 for adenoma or carcinoma (P-value = 0.40)
{[}Figure 2A{]}. The ORs for evenness were significantly higher than 1.0
for adenoma (OR = 1.3 (1.02 - 1.65), P-value = 0.035) and carcinoma (OR
= 1.66 (1.2 - 2.3), P-value = 0.0021) {[}Figure 2B{]}. The ORs for
diversity were only significantly greater than 1.0 for carcinoma (OR =
1.61 (1.14 - 2.28), P-value = 0.0069), but not for adenoma (P-value =
0.11) {[}Figure 2C{]}. Although these OR are significantly greater than
1.0, it is doubtful that these are clinically meaningful values.

Similar to our analysis of sequences obtained from stool samples, we
repeated the analysis using sequences obtained from colon tissue. There
were no significant changes in richness, evenness, or diversity as
disease severity progressed from control to adenoma to carcinoma
(P-value \textgreater{} 0.05). We next analyzed the OR, for matched
(i.e.~where unaffected tissue and tumors were obtained from the same
individual) and unmatched (i.e.~where unaffected tissue and tumor tissue
were not obtained from the same individual) tissue samples. The ORs for
adenoma and carcinoma by any measure were not significantly different
from 1.0 (P-value \textgreater{} 0.05) {[}Figure S1 \& Table S1{]}. This
is likely due to the combination of a small effect size, as suggested
from the results using stool, and the relatively small number of studies
and size of studies used in the analysis.

\textbf{\emph{Disease progression is associated with community-wide
changes in composition and abundance:}} Based on the differences in
evenness and diversity, we next asked whether there were community-wide
differences in the structure of the communities associated with
different disease stages. We identified significant bacterial community
differences in the stool of patients with adenomas relative to those
with normal colons in 1 of 4 studies and in patients with carcinomas
relative to those with normal colons in 6 of 7 studies (PERMANOVA;
P-value \textless{} 0.05) {[}Table S2{]}. Similar to the analyses using
stool samples, there were significant differences in bacterial community
structure between subjects with normal colons and those with adenoma (1
of 2 studies) and carcinoma (1 of 3 studies) {[}Table S2{]}. For studies
that used matched samples no differences in bacterial community
structures were observed {[}Table S2{]}. Combined, these results
indicate that there consistent and significant community-wide changes in
the fecal community structure of subjects with carcinomas. However, the
signal observed in subjects with adenomas or when using tissue samples
was not as consistent. This is likely due to a smaller effect size or
the relatively small sample sizes among the studies that characterized
the tissue microbiota.

\textbf{\emph{Individual taxa are associated with significant ORs for
carcinomas:}} Next we identified those taxa were associated with ORs
that were significantly associated with having a normal colon or the
presence of adenomas or carcinomas. No taxa had a significant OR for the
presence of adenomas when we used data collected from stool or tissue
samples (Table S3 \& S4). In contrast, 8 taxa had significant ORs for
the presence of carcinomas using data from stool samples. Of these, 4
are commonly associated with the oral cavity: \emph{Fusobacterium} (OR =
2.74 (1.95 - 3.85)), \emph{Parvimonas} (OR = 3.07 (2.11 - 4.46)),
\emph{Porphyromonas} (OR = 3.2 (2.26 - 4.54)), and
\emph{Peptostreptococcus} (OR = 7.11 (3.84 - 13.17)) {[}Table S3{]}. The
other 4 were \emph{Clostridium XI} (OR = 0.65 (0.49 - 0.86)),
\emph{Enterobacteriaceae} (OR = 1.79 (1.33 - 2.41)), \emph{Escherichia}
(OR = 2.15 (1.57 - 2.95)), and \emph{Ruminococcus} (OR = 0.63 (0.48 -
0.83)). Among the data collected from tissue samples, only unmatched
carcinoma samples had taxa with a significant OR. Those included
\emph{Dorea} (OR = 0.35 (0.22 - 0.55)), \emph{Blautia} (OR = 0.47 (0.3 -
0.73)), and \emph{Weissella} (OR = 5.15 (2.02 - 13.14)).
Mouth-associated genera were not significantly associated with an
increased OR for carcinoma in tissue samples {[}Table S4{]}. For
example, \emph{Fusobacterium} had an OR of 3.98 (1.19 - 13.24; however,
due to the small number of studies and considerable variation in the
data, the Benjimani-Hochberg-corrected P-value was 0.93 {[}Table S4{]}.
It is interesting to note that \emph{Ruminococcus} and members of
\emph{Clostridium group XI} in stool and \emph{Dorea} and \emph{Blautia}
in tissue had ORs that were significantly less than 1.0, which suggests
that these populations are protective against the development of
carcinomas. Overall, there was no overlap in the taxa with significant
OR between stool and tissue samples.

\textbf{\emph{Individual taxa with a significant OR do a poor job of
differentiating subjects with normal colons and those with carcinoma:}}
We next asked whether those taxa that had a significant OR associated
with having a normal colon or carcinomas could be used individually, to
classify subjects as having a normal colon or carcinomas. OR values were
caluclated based on whether the relative abundance for a taxon in a
subject was above or below the median relative abundance for that taxon
across all subjects in a study. To measure the ability of these taxa to
classify individuals we instead generated receiver operator
characteristic (ROC) curves for each taxon in each study and calculated
the area under the curve (AUC). This allowed us to use a more fluid
relative abundance threshold for defining disease status. Using data
from stool samples, the 8 taxa did no better at classifying the subjects
than one would expect by chance (i.e.~AUC=0.50) {[}Figure 3A{]}. The
taxa that performed the best included \emph{Clostridium XI},
\emph{Ruminococcus}, and \emph{Escherichia}, however, these had median
AUC values less than 0.588. Likewise, in unmatched tissue samples the 8
taxa with significant ORs taxa were marginally better than one would
expect by chance {[}Figure 3B{]}. The relative abundance of Dorea was
the best predictor of carcinomas and its median AUC was only 0.62. These
results suggest that although these taxa are associated with a decreased
or increased OR for the presences of carcinomas, individually, they do a
poor job of classifying a subject's disease status.

\textbf{\emph{Combined taxa model classifies subjects better than using
individual taxa:}} Instead of attempting to classify subjects based on
individual taxa, next we combined information from the individual taxa
and evaluated the ability to classify a subject's disease status using
Random Forest models. For data from stool samples, the combined model
had an AUC of 0.75, which was significantly higher than any of the AUC
values for the individual taxa (P-value \textless{} 0.033). For the full
taxa models using stool, \emph{Bacteroides} and \emph{Lachnospiraceae}
were the most common taxa in the top 10\% mean decrease in accuracy
(MDA) across studies {[}Figure S2{]}. Of the 3 taxa with significant
ORs, all 3 were among the top 10\% most important taxa as measured by
mean decrease in accuracy, in at least one study. The most important
taxa across study within the significant OR taxa only models for stool
were \emph{Ruminococcus} and \emph{Clostridium XI} {[}Figure 5A{]}.
Similarly, using data from the unmatched tissue samples, the combined
model had an AUC of 0.77, which was significantly higher than the AUC
values for Blautia and Weissella (P-value \textless{} 0.037). For the
full taxa models using unmatched tissue, \emph{Lachnospiraceae},
\emph{Bacteroidaceae}, and \emph{Ruminococcaceae} were the most common
taxa in the top 10\% mean decrease in accuracy across studies {[}Figure
S3{]}. For the singificant OR taxa unmatched tissue models both
\emph{Dorea} and \emph{Blautia} were the important based on mean
decrease in accuracy {[}Figure 5B{]}. Pooling the information from the
taxa with significant ORs results in a model that outperforms
classifications made using individual taxa.

\textbf{\emph{Performance of models based on taxa relative abundance in
full community are better than those based on taxa with significant
ORs:}} Next, we asked whether a Random Forest classification model built
using all of the taxa found in the communities would outperform the
models generated using those taxa with a significant OR. Similar to our
inability to identify taxa associated with a significant OR for the
presence of adenomas, the median AUCs to classify subjects as having
normal colons or having adenomas using data from stool or tissue samples
were only marginally better than 0.5 for any study (median AUC =
0.5486298 {[}0.3671667 - 0.971{]}) {[}Figure 4A \& S4A{]}. In contrast,
the models for classifying subjects as having normal colons or having
carcinomas using data from stool or tissue samples yielded AUC values
meaningfully higher than 0.5 {[}Figure 4B \& S4B-C{]}. When we compared
the models based on all of the taxa in a community to models based on
the taxa with significant ORs, the results were mixed. Using the data
from stool samples we found that the AUC for 6 of 7 studies increased by
an average of -14.8\%) and AUC for the Flemer study decreased by
0.54\%). The overall improvement in performance was statistically
significant (mean = 12.61\%, one-tailed paired T-test; P-value = 0.005).
Similarly, using the data from unmatched tissue samples we found that
the AUC of studies increased by an average of 19.11\% when we used all
of the taxa rather than the 3 taxa with significant ORs (one-tailed
paired T-test; P-value = 0.03). Although the significant OR taxa models
can classify indidivuals with and without carcinoma tumors, they are
still missing taxa from the full community models that can increase the
model accuracy.

\textbf{\emph{Performance of models based on OTU relative abundance in
full community are not significantly better than those based on taxa
with significant ORs:}} The previous models were based on relative
abundance data where sequences were classified to coarse taxonomic
assignments (i.e.~typically genus or family level). To determine whether
model performance improved with finer scale classification, we assigned
sequences to operational taxonomic units (OTUs) where the similarity
among sequences within an OTU was more than 97\%. We again found that
classification models built using all of the sequence data for a
community did a poor job of differentiating between subjects with normal
colons and those with adenomas (median AUC: 0.53 {[}0.37- 0.56{]}), but
did a good job of differentiating between subjects with normal colons
and those with carcinomas (median AUC: 0.71 {[}0.5- 0.9{]}). The
OTU-based models performed similarly to those constructed using the taxa
with significant ORs (one-tailed paired T-test; P-value = 0.966) and
those using all taxa (one-tailed paired T-test; P-value = 0.146)
{[}Figure 4{]}. Among the OTUs that had the highest mean decrease in
accuracy for the OTU-based models, we found that OTUs that affiliated
with all of the 8 taxa that had a significant OR were within the top
10\% for at least one study. This result was surprising as it indicated
that a finer scale classification of the sequences and thus a larger
number of features to select from, did not yield improved classification
of the subjects.

\textbf{\emph{Generalizability of taxon-based models trained on one
dataset to the other datasets:}} Considering the good performance of the
Random Forest models using taxa with a significant OR and using all of
the taxa, we next asked how well the models would perform when given
data from a different subject cohort. For instance, if a model was
trained using data from the Ahn study, we wanted to know how well it
would perform using the data from the Baxter study. We found the models
trained using the taxa with a significant OR all had a higher median AUC
than the models trained using all of the taxa when tested on the other
datasets {[}Figure 6 \& S5{]}. As might be expected, the difference
between the performance of the modelling approaches appeared to vary
with the size of the training cohort (R\textsuperscript{2} = 0.66)
{[}Figure 6{]}. These data suggest that given a sufficient number of
subjects with normal colons and carcinomas, Random Forest models trained
using a small number of taxa can accurately classify individuals from a
different cohort.

\newpage

\subsection{Discussion}\label{discussion}

Although we expected that the full OTU models would perform the best at
classifying individuals with and without carcinomas, our observations
suggest that both the full and significant OR taxa models performed
equally well. These results suggest that lower level classification to
species and strain may not add extra useful information with respect to
prediction models. This has been suggested in previous literature where
metagenomics did not perform better than 16S rRNA gene sequencing data
at classifying individuals with normal colons and those with carcinomas
(30). One possible reason as to why lower level classification may not
result in better models is that the communities are patchy and higher
level taxonomic information pools some of this patchiness, allowing for
better prediction models. There may also be a fair bit of data
redundancy within models that utilize more of the community. An example
of this redundancy would be when we trained models on one study and
tested it on the other studies and the AUCs of the models created with
the select OR taxa performed as well as full taxa models {[}Figure
6B{]}.

Our observations also suggest that a small collection of taxa can
classify disease as well as full OTU-based models but that these taxa
individually perform quite poorly {[}Figure 3{]}. This result supports
the contention that there might be redundancy of function amongst the
taxa included in the significant OR models. As an example, multiple
different microbes could be similarily stimulating the activation of
inflammatory pathways and by doing so exacerbate disease progression.
Multiple reports within the literature have found that different
bacteria, such as \emph{Escherichia coli} and \emph{Fusobacterium
nucleatum}, can similarily worsen inflammation in mouse models of
tumorigenesis (5, 6, 21). Although the inflammatory taxa were patchy in
their importance and presence across studies those that were not
typically associated with inflammation were consistently important for
every study {[}Figure 6{]}. The loss of these taxa (\emph{Ruminococcus}
and \emph{Clostridium XI} in stool and \emph{Dorea} and \emph{Blautia}
in unmatched tissue) is particularly interesting because many are
commonly thought to be beneficial due to their involvement in production
of short chain fatty acids (31--33).

The adenoma models as a whole performed poorly in classifying
individuals with and without adenomas. This outcome is not inconsistent
with what has been published previously (27, 34). However, the modeling
results are at odds with results obtained in Baxter, \emph{et al.} (12).
There are some major differences between the models generated in this
meta-analysis and what was used in Baxter, \emph{et al}. First, the
prevoius report's models investigated the classification of lesions
(individuals with adenoma or carcinoma) and not adenoma alone. The
Baxter, \emph{et al.} models also contained Fecal Immunoglobulin Test
data while our meta-analysis models only contained 16S rRNA gene
sequencing data. Although being able to classify individuals with
adenomas is important, the classification of advanced adenomas is a more
clinically meaningful diagnostic tool (i.e.~those that are at highest
risk of progressing to a carcinoma). It is possible that we might have
been able to detect differences in the bacterial community if advanced
adenomas were separated from adenomas but that data was not available
for the majority of studies analyzed. Additionally, the initial changes
to the bacterial community could be focal to where the initial adenoma
develops and would not be easily assessed with a fecal sample.

Although stool represents an easy and less invasive way to assess risk,
it is not clear how well this sample reflects adenoma- and carcinoma-
tissue associated microbial communities. The colon tissue-based studies
did not provide a clearer understanding of how the microbiota may be
associated with tumors. Generally, the full OTU-based models of
unmatched and matched colon tissue samples were concordant with stool
samples showing that GI resident microbes were the most prevalent in the
top 10 most important variables across study {[}Figure S3{]}.
\emph{Fusobacterium} was not consistently identified across studies and
this could be due to both a small number of studies and a small sample
size within these studies. The majority of the colon tissue-based
results were also study specific with many of the top 10 taxa being
present only in a single study. Additionally, the presence of genera
associated with contamination, within the top 10 most important
variables for the genera and OTU models is worrying (e.g.
\emph{Novosphingobium}, \emph{Acidobacteria Gp2}, \emph{Sphingomonas},
etc. (35)). The low bacterial biomass of tissue samples coupled with
potential contamination and small sample sizes could explain why these
results seem to be more sporadic than the stool results.

One important caveat to this study is that even though genera associated
with certain species such as \emph{Bacteroides fragilis} and
\emph{Streptococcus gallolyticus} subsp. \emph{gallolyticus} were not
identified, it does not necessarily mean that these specific species are
not important in human CRC (22, 24). There are reports that
\emph{Bacteroides fragilis}, positive for the enterotoxigenic gene, are
found at specific locations along the colon but the samples we were able
to use in this meta-analysis could not identify these types of
differences (36). Additionally, since we are limited in our aggregation
of the data to the genus level, it is not possible to clearly delineate
which species are contributing to overall disease progression. Our
observations are not inconsistent with the previous literature on either
\emph{Bacteroides fragilis} or \emph{Streptococcus gallolyticus} subsp.
\emph{gallolyticus}. As an example, the stool-based full community
models consistently identified the \emph{Bacteroides} taxa to be
important model components across studies. This suggests that even
though \emph{Bacteroides} may not increase the OR of individuals having
an adenoma or carcinoma and may not vary in relative abundance, like
\emph{Fusobacterium}, it is still important in CRC.

Despite these limitations to the findings, meta-analyses are a useful
tool in microbiota research because they can both validate existing
research and make new discoveries by pooling many independent
investigations together. Yet, it is still difficult to perform these
studies because of inaccessible 16S sequencing data, missing or vague
metadata (e.g.~which samples are carcioma and which are not), varying
sequence quality, and multiple small data sets. Better attention to
these specific problems could help to increase the reproducibility and
replicability of microbiota studies and make it easier to perform these
crucial meta-analyses. Moving forward, meta-analyses will be important
tools to help aggregate and find commonalities across studies when
investigating the microbiota in the context of a specific disease and
more are needed (28, 37--39).

By aggregating together a large collection of studies analyzing both
fecal and colon tissue samples, we are able to provide evidence
supporting the importance of the bacterial community in carcinoma
tumors. Although further validation of the biomarkers presented here
need to be undertaken, the replicability of the AUC of a specific
collection of taxa across multiple studies suggests a strong potential
for the use of the microbiota as a risk stratification tool for
individuals with carcinomas.

\newpage

\subsection{Methods}\label{methods}

\textbf{\emph{Datasets:}} The studies used for this meta-analysis were
identified through the review articles written by Keku, et al. (40) and
Vogtmann, et al. (41). Additional studies, not mentioned in those
reviews were obtained based on the authors' knowledge of the literature.
Studies were included that used tissue or feces as their sample source
for 454 or Illumina 16S rRNA gene sequencing. Some studies (N = 12) were
excluded because they did not have publicly available sequences or did
not have metadata in which the authors were able to share. We were able
to obtain sequence data and metadata from the following studies: Ahn,
\emph{et al.} (11), Baxter, \emph{et al.} (12), Brim, \emph{et al.}
(29), Burns, \emph{et al.} (15), Chen, \emph{et al.} (13), Dejea,
\emph{et al.} (20), Flemer, \emph{et al.} (17), Geng, \emph{et al.}
(19), Hale, \emph{et al.} (27), Kostic, \emph{et al.} (42), Lu, \emph{et
al.} (26), Sanapareddy, \emph{et al.} (25), Wang, \emph{et al.} (14),
Weir, \emph{et al.} (23), and Zeller, \emph{et al.} (16). The Zackular
(43) study was excluded because their 90 individuals were contained
within the larger Baxter study (12). The Kostic study was excluded
because after we processed the sequences, all of the case samples had
100 or fewer sequences. The final analysis included 14 studies (Tables 1
and 2). There were seven studies with only fecal samples (Ahn, Baxter,
Brim, Hale, Wang, Weir, and Zeller), five studies with only tissue
samples (Burns, Dejea, Geng, Lu, Sanapareddy), and two studies with both
fecal and tissue samples (Chen and Flemer). After curating the
sequences, 1737 stool samples and 492 tissue samples remained in the
analysis {[}Tables 1 and 2{]}.

\textbf{\emph{Sequence Processing:}} Raw sequence data and metadata were
primarily obtained from the Sequence Read Archive (SRA) and dbGaP. Other
sequence and metadata were obtained directly from the authors (n = 4,
(17, 23, 25, 27)). Each dataset was processed separately using mothur
(v1.39.3) (44) using the default quality filtering methods for both 454
and Illumina sequence data. If it was not possible to use the defaults
because the sequences were trimmed too much, then the stated quality
cut-offs from the original study were used. Chimeric sequences were
identified and removed using VSEARCH (45). The curated sequences were
assigned to OTUs at 97\% similarity using the OptiClust algorithm (46)
and classified to the deepest taxonomic level that had 80\% support
using the naïve Bayesian classifier trained on the RDP taxonomy outline
(version 14, (47)).

\textbf{\emph{Community analysis:}} We calculated alpha diversity
metrics (i.e.~OTU richness, evenness, and Shannon diversity) for each
sample. Within each dataset, we ensured that the data followed a normal
distribution using power transformations. Using the transformed data, we
tested the hypothesis that individuals with normal colons, adenomas, and
carcinomas had significantly different alpha diversity metrics using
linear mixed-effect models. We also calculated the OR for each study and
metric by considering any value above the median alpha diversity value
to be positive. We measured the dissimilarity between individuals by
calculating the pairwise Bray-Curtis index and used PERMANOVA (48) to
test whether individuals with normal colons were significantly different
from those with adenomas or carcinomas. Finally, after binning sequences
into the deepest taxa that the naïve Bayesian classifier could calssify
the sequences, we quantified the ORs for individuals having an adenoma
or carcinoma and corrected for multiple comparisons using the
Benjamini-Hochberg method (49). Again, for each taxon, if the relative
abundance was greater than the median relative abundance for that taxon
in the study, the individual was considered to be positive.

\textbf{\emph{Random Forest classification analysis:}} To classify
individuals as having normal colons or tumors, we built Random Forest
classification models for each dataset and comparison using taxa with
significant ORs (after multiple comparison correction), all taxa, or
OTUs. Because no taxa were identified as having a significant OR
associated with adenomas using stool samples or tissue samples,
classification models based on OR data were not constructed to classify
individuals as having normal colons or adenomas. Within the training
dataset, 10-fold cross validation (5-fold cross validation for small
datasets) was used to build a model using the default mtry setting that
was then evaluated on the testing set. For the models constructed using
the taxa with significant ORs, the default mtry setting was used to
train the model and this model was tested on the other datasets in the
meta-analysis. The reported AUC values are the average AUCs for the
application of the model on the test sets. For the OTU-based models,
hyperparameters were fit splitting the dataset into training (80\% of
samples) and testing (20\%) sets. Within the training dataset, 10-fold
cross validation was used to identify the best hyperparameters, which
were then used to build a model that was then evaluated on the testing
set. The original 80/20 split and fitting was repeated 100 times. The
Mean Decrease in Accuracy (MDA), a measure of the importance of each
taxon to the overall model was used to rank the taxa used in each model.

\textbf{\emph{Statistical Analysis:}} All statistical analysis after
sequence processing utilized the R (v3.4.3) software package (50). For
OTU richness, evenness, and Shannon diversity analysis, values were
power transformed using the rcompanion (v1.11.1) package (51) and then
Z-score normalized using the car (v2.1.6) package (52). Testing for OTU
richness, evenness, and Shannon diversity differences utilized linear
mixed-effect models created using the lme4 (v1.1.15) package (53) to
correct for study, repeat sampling of individuals (tissue only), and 16S
hyper-variable region used. Odds ratios (OR) were analyzed using both
the epiR (v0.9.93) and metafor (v2.0.0) packages (54, 55) by assessing
how many individuals with and without disease were above and below the
overall median value within each specific study. OR significance testing
utilized the chi-squared test. Diversity differences measured by the
Bray-Curtis index utilized the creation of distance matrix and testing
with PERMANOVA executed with the vegan (v2.4.5) package (56). Random
Forest models were built using both the caret (v6.0.78) and randomForest
(v4.6.12) packages (57, 58). All figures were created using both ggplot2
(v2.2.1) and gridExtra (v2.3) packages (59, 60).

\textbf{\emph{Reproducible Methods:}} The code and analysis can be found
at
\url{https://github.com/SchlossLab/Sze_CRCMetaAnalysis_Microbiome_2017}.
Unless otherwise mentioned, the accession number of raw sequences from
the studies used in this analysis can be found directly in the
respective batch file in the GitHub repository or in the original
manuscript.

\newpage

\subsection{Declarations}\label{declarations}

\subsubsection{Ethics approval and consent to
participate}\label{ethics-approval-and-consent-to-participate}

Ethics approval and informed consent for each of the studies used is
mentioned in the respective manuscripts used in this meta-analysis.

\subsubsection{Consent for publication}\label{consent-for-publication}

Not applicable.

\subsubsection{Availability of data and
material}\label{availability-of-data-and-material}

A detailed and reproducible description of how the data were processed
and analyzed for each study can be found at
\url{https://github.com/SchlossLab/Sze_CRCMetaAnalysis_Microbiome_2017}.
Raw sequences can be downloaded from the SRA in most cases and can be
found in the respective study batch file in the GitHub repository or
within the original publication. For instances when sequences are not
publicly available, they may be accessed by contacting the corresponding
authors from whence the data came.

\subsubsection{Competing Interests}\label{competing-interests}

All authors declare that they do not have any relevant competing
interests to report.

\subsubsection{Funding}\label{funding}

MAS is supported by a Canadian Institute of Health Research fellowship
and a University of Michigan Postdoctoral Translational Scholar Program
grant.

\subsubsection{Authors' contributions}\label{authors-contributions}

All authors helped to design and conceptualize the study. MAS identified
and analyzed the data. MAS and PDS interpreted the data. MAS wrote the
first draft of the manuscript and both he and PDS reviewed and revised
updated versions. All authors approved the final manuscript.

\subsubsection{Acknowledgements}\label{acknowledgements}

The authors would like to thank all the study participants who were a
part of each of the individual studies utilized. We would also like to
thank each of the study authors for making their data available for use.
Finally, we would like to thank the members of the Schloss lab for
valuable feed back and proof reading during the formulation of this
manuscript.

\newpage

\subsection{References}\label{references}

\hypertarget{refs}{}
\hypertarget{ref-siegel_cancer_2016}{}
1. \textbf{Siegel, R. L.}, \textbf{K. D. Miller}, and \textbf{A. Jemal}.
2016. Cancer statistics, 2016. CA: a cancer journal for clinicians
\textbf{66}:7--30.

\hypertarget{ref-flynn_metabolic_2016}{}
2. \textbf{Flynn, K. J.}, \textbf{N. T. Baxter}, and \textbf{P. D.
Schloss}. 2016. Metabolic and Community Synergy of Oral Bacteria in
Colorectal Cancer. mSphere \textbf{1}.

\hypertarget{ref-goodwin_polyamine_2011}{}
3. \textbf{Goodwin, A. C.}, \textbf{C. E. Destefano Shields}, \textbf{S.
Wu}, \textbf{D. L. Huso}, \textbf{X. Wu}, \textbf{T. R. Murray-Stewart},
\textbf{A. Hacker-Prietz}, \textbf{S. Rabizadeh}, \textbf{P. M. Woster},
\textbf{C. L. Sears}, and \textbf{R. A. Casero}. 2011. Polyamine
catabolism contributes to enterotoxigenic Bacteroides fragilis-induced
colon tumorigenesis. Proceedings of the National Academy of Sciences of
the United States of America \textbf{108}:15354--15359.

\hypertarget{ref-abed_fap2_2016}{}
4. \textbf{Abed, J.}, \textbf{J. E. M. Emgård}, \textbf{G. Zamir},
\textbf{M. Faroja}, \textbf{G. Almogy}, \textbf{A. Grenov}, \textbf{A.
Sol}, \textbf{R. Naor}, \textbf{E. Pikarsky}, \textbf{K. A. Atlan},
\textbf{A. Mellul}, \textbf{S. Chaushu}, \textbf{A. L. Manson},
\textbf{A. M. Earl}, \textbf{N. Ou}, \textbf{C. A. Brennan}, \textbf{W.
S. Garrett}, and \textbf{G. Bachrach}. 2016. Fap2 Mediates Fusobacterium
nucleatum Colorectal Adenocarcinoma Enrichment by Binding to
Tumor-Expressed Gal-GalNAc. Cell Host \& Microbe \textbf{20}:215--225.

\hypertarget{ref-arthur_intestinal_2012}{}
5. \textbf{Arthur, J. C.}, \textbf{E. Perez-Chanona}, \textbf{M.
Mühlbauer}, \textbf{S. Tomkovich}, \textbf{J. M. Uronis}, \textbf{T.-J.
Fan}, \textbf{B. J. Campbell}, \textbf{T. Abujamel}, \textbf{B. Dogan},
\textbf{A. B. Rogers}, \textbf{J. M. Rhodes}, \textbf{A. Stintzi},
\textbf{K. W. Simpson}, \textbf{J. J. Hansen}, \textbf{T. O. Keku},
\textbf{A. A. Fodor}, and \textbf{C. Jobin}. 2012. Intestinal
inflammation targets cancer-inducing activity of the microbiota. Science
(New York, N.Y.) \textbf{338}:120--123.

\hypertarget{ref-kostic_fusobacterium_2013}{}
6. \textbf{Kostic, A. D.}, \textbf{E. Chun}, \textbf{L. Robertson},
\textbf{J. N. Glickman}, \textbf{C. A. Gallini}, \textbf{M. Michaud},
\textbf{T. E. Clancy}, \textbf{D. C. Chung}, \textbf{P. Lochhead},
\textbf{G. L. Hold}, \textbf{E. M. El-Omar}, \textbf{D. Brenner},
\textbf{C. S. Fuchs}, \textbf{M. Meyerson}, and \textbf{W. S. Garrett}.
2013. Fusobacterium nucleatum potentiates intestinal tumorigenesis and
modulates the tumor-immune microenvironment. Cell Host \& Microbe
\textbf{14}:207--215.

\hypertarget{ref-wu_human_2009}{}
7. \textbf{Wu, S.}, \textbf{K.-J. Rhee}, \textbf{E. Albesiano},
\textbf{S. Rabizadeh}, \textbf{X. Wu}, \textbf{H.-R. Yen}, \textbf{D. L.
Huso}, \textbf{F. L. Brancati}, \textbf{E. Wick}, \textbf{F.
McAllister}, \textbf{F. Housseau}, \textbf{D. M. Pardoll}, and
\textbf{C. L. Sears}. 2009. A human colonic commensal promotes colon
tumorigenesis via activation of T helper type 17 T cell responses.
Nature Medicine \textbf{15}:1016--1022.

\hypertarget{ref-zackular_manipulation_2016}{}
8. \textbf{Zackular, J. P.}, \textbf{N. T. Baxter}, \textbf{G. Y. Chen},
and \textbf{P. D. Schloss}. 2016. Manipulation of the Gut Microbiota
Reveals Role in Colon Tumorigenesis. mSphere \textbf{1}.

\hypertarget{ref-zackular_gut_2013}{}
9. \textbf{Zackular, J. P.}, \textbf{N. T. Baxter}, \textbf{K. D.
Iverson}, \textbf{W. D. Sadler}, \textbf{J. F. Petrosino}, \textbf{G. Y.
Chen}, and \textbf{P. D. Schloss}. 2013. The gut microbiome modulates
colon tumorigenesis. mBio \textbf{4}:e00692--00613.

\hypertarget{ref-baxter_structure_2014}{}
10. \textbf{Baxter, N. T.}, \textbf{J. P. Zackular}, \textbf{G. Y.
Chen}, and \textbf{P. D. Schloss}. 2014. Structure of the gut microbiome
following colonization with human feces determines colonic tumor burden.
Microbiome \textbf{2}:20.

\hypertarget{ref-ahn_human_2013}{}
11. \textbf{Ahn, J.}, \textbf{R. Sinha}, \textbf{Z. Pei}, \textbf{C.
Dominianni}, \textbf{J. Wu}, \textbf{J. Shi}, \textbf{J. J. Goedert},
\textbf{R. B. Hayes}, and \textbf{L. Yang}. 2013. Human gut microbiome
and risk for colorectal cancer. Journal of the National Cancer Institute
\textbf{105}:1907--1911.

\hypertarget{ref-baxter_microbiota-based_2016}{}
12. \textbf{Baxter, N. T.}, \textbf{M. T. Ruffin}, \textbf{M. A. M.
Rogers}, and \textbf{P. D. Schloss}. 2016. Microbiota-based model
improves the sensitivity of fecal immunochemical test for detecting
colonic lesions. Genome Medicine \textbf{8}:37.

\hypertarget{ref-chen_human_2012}{}
13. \textbf{Chen, W.}, \textbf{F. Liu}, \textbf{Z. Ling}, \textbf{X.
Tong}, and \textbf{C. Xiang}. 2012. Human intestinal lumen and
mucosa-associated microbiota in patients with colorectal cancer. PloS
One \textbf{7}:e39743.

\hypertarget{ref-wang_structural_2012}{}
14. \textbf{Wang, T.}, \textbf{G. Cai}, \textbf{Y. Qiu}, \textbf{N.
Fei}, \textbf{M. Zhang}, \textbf{X. Pang}, \textbf{W. Jia}, \textbf{S.
Cai}, and \textbf{L. Zhao}. 2012. Structural segregation of gut
microbiota between colorectal cancer patients and healthy volunteers.
The ISME journal \textbf{6}:320--329.

\hypertarget{ref-burns_virulence_2015}{}
15. \textbf{Burns, M. B.}, \textbf{J. Lynch}, \textbf{T. K. Starr},
\textbf{D. Knights}, and \textbf{R. Blekhman}. 2015. Virulence genes are
a signature of the microbiome in the colorectal tumor microenvironment.
Genome Medicine \textbf{7}:55.

\hypertarget{ref-zeller_potential_2014}{}
16. \textbf{Zeller, G.}, \textbf{J. Tap}, \textbf{A. Y. Voigt},
\textbf{S. Sunagawa}, \textbf{J. R. Kultima}, \textbf{P. I. Costea},
\textbf{A. Amiot}, \textbf{J. Böhm}, \textbf{F. Brunetti}, \textbf{N.
Habermann}, \textbf{R. Hercog}, \textbf{M. Koch}, \textbf{A. Luciani},
\textbf{D. R. Mende}, \textbf{M. A. Schneider}, \textbf{P.
Schrotz-King}, \textbf{C. Tournigand}, \textbf{J. Tran Van Nhieu},
\textbf{T. Yamada}, \textbf{J. Zimmermann}, \textbf{V. Benes},
\textbf{M. Kloor}, \textbf{C. M. Ulrich}, \textbf{M. von Knebel
Doeberitz}, \textbf{I. Sobhani}, and \textbf{P. Bork}. 2014. Potential
of fecal microbiota for early-stage detection of colorectal cancer.
Molecular Systems Biology \textbf{10}:766.

\hypertarget{ref-flemer_tumour-associated_2017}{}
17. \textbf{Flemer, B.}, \textbf{D. B. Lynch}, \textbf{J. M. R. Brown},
\textbf{I. B. Jeffery}, \textbf{F. J. Ryan}, \textbf{M. J. Claesson},
\textbf{M. O'Riordain}, \textbf{F. Shanahan}, and \textbf{P. W.
O'Toole}. 2017. Tumour-associated and non-tumour-associated microbiota
in colorectal cancer. Gut \textbf{66}:633--643.

\hypertarget{ref-GimenoGarca2012}{}
18. \textbf{García, A. Z. G.} 2012. Factors influencing colorectal
cancer screening participation. Gastroenterology Research and Practice.
Hindawi Limited \textbf{2012}:1--8.

\hypertarget{ref-geng_diversified_2013}{}
19. \textbf{Geng, J.}, \textbf{H. Fan}, \textbf{X. Tang}, \textbf{H.
Zhai}, and \textbf{Z. Zhang}. 2013. Diversified pattern of the human
colorectal cancer microbiome. Gut Pathogens \textbf{5}:2.

\hypertarget{ref-dejea_microbiota_2014}{}
20. \textbf{Dejea, C. M.}, \textbf{E. C. Wick}, \textbf{E. M.
Hechenbleikner}, \textbf{J. R. White}, \textbf{J. L. Mark Welch},
\textbf{B. J. Rossetti}, \textbf{S. N. Peterson}, \textbf{E. C.
Snesrud}, \textbf{G. G. Borisy}, \textbf{M. Lazarev}, \textbf{E. Stein},
\textbf{J. Vadivelu}, \textbf{A. C. Roslani}, \textbf{A. A. Malik},
\textbf{J. W. Wanyiri}, \textbf{K. L. Goh}, \textbf{I. Thevambiga},
\textbf{K. Fu}, \textbf{F. Wan}, \textbf{N. Llosa}, \textbf{F.
Housseau}, \textbf{K. Romans}, \textbf{X. Wu}, \textbf{F. M.
McAllister}, \textbf{S. Wu}, \textbf{B. Vogelstein}, \textbf{K. W.
Kinzler}, \textbf{D. M. Pardoll}, and \textbf{C. L. Sears}. 2014.
Microbiota organization is a distinct feature of proximal colorectal
cancers. Proceedings of the National Academy of Sciences of the United
States of America \textbf{111}:18321--18326.

\hypertarget{ref-ecoli_Arthur_2014}{}
21. \textbf{Arthur, J. C.}, \textbf{R. Z. Gharaibeh}, \textbf{M.
Mühlbauer}, \textbf{E. Perez-Chanona}, \textbf{J. M. Uronis}, \textbf{J.
McCafferty}, \textbf{A. A. Fodor}, and \textbf{C. Jobin}. 2014.
Microbial genomic analysis reveals the essential role of inflammation in
bacteria-induced colorectal cancer. Nature Communications. Springer
Nature \textbf{5}:4724.

\hypertarget{ref-strep_Aymeric_2017}{}
22. \textbf{Aymeric, L.}, \textbf{F. Donnadieu}, \textbf{C. Mulet},
\textbf{L. du Merle}, \textbf{G. Nigro}, \textbf{A. Saffarian},
\textbf{M. Bérard}, \textbf{C. Poyart}, \textbf{S. Robine}, \textbf{B.
Regnault}, \textbf{P. Trieu-Cuot}, \textbf{P. J. Sansonetti}, and
\textbf{S. Dramsi}. 2017. Colorectal cancer specific conditions
promoteStreptococcus gallolyticusgut colonization. Proceedings of the
National Academy of Sciences. Proceedings of the National Academy of
Sciences \textbf{115}:E283--E291.

\hypertarget{ref-weir_stool_2013}{}
23. \textbf{Weir, T. L.}, \textbf{D. K. Manter}, \textbf{A. M. Sheflin},
\textbf{B. A. Barnett}, \textbf{A. L. Heuberger}, and \textbf{E. P.
Ryan}. 2013. Stool microbiome and metabolome differences between
colorectal cancer patients and healthy adults. PloS One
\textbf{8}:e70803.

\hypertarget{ref-bfrag_Boleij_2014}{}
24. \textbf{Boleij, A.}, \textbf{E. M. Hechenbleikner}, \textbf{A. C.
Goodwin}, \textbf{R. Badani}, \textbf{E. M. Stein}, \textbf{M. G.
Lazarev}, \textbf{B. Ellis}, \textbf{K. C. Carroll}, \textbf{E.
Albesiano}, \textbf{E. C. Wick}, \textbf{E. A. Platz}, \textbf{D. M.
Pardoll}, and \textbf{C. L. Sears}. 2014. The bacteroides fragilis toxin
gene is prevalent in the colon mucosa of colorectal cancer patients.
Clinical Infectious Diseases. Oxford University Press (OUP)
\textbf{60}:208--215.

\hypertarget{ref-sanapareddy_increased_2012}{}
25. \textbf{Sanapareddy, N.}, \textbf{R. M. Legge}, \textbf{B. Jovov},
\textbf{A. McCoy}, \textbf{L. Burcal}, \textbf{F. Araujo-Perez},
\textbf{T. A. Randall}, \textbf{J. Galanko}, \textbf{A. Benson},
\textbf{R. S. Sandler}, \textbf{J. F. Rawls}, \textbf{Z. Abdo},
\textbf{A. A. Fodor}, and \textbf{T. O. Keku}. 2012. Increased rectal
microbial richness is associated with the presence of colorectal
adenomas in humans. The ISME journal \textbf{6}:1858--1868.

\hypertarget{ref-lu_mucosal_2016}{}
26. \textbf{Lu, Y.}, \textbf{J. Chen}, \textbf{J. Zheng}, \textbf{G.
Hu}, \textbf{J. Wang}, \textbf{C. Huang}, \textbf{L. Lou}, \textbf{X.
Wang}, and \textbf{Y. Zeng}. 2016. Mucosal adherent bacterial dysbiosis
in patients with colorectal adenomas. Scientific Reports
\textbf{6}:26337.

\hypertarget{ref-hale_shifts_2017}{}
27. \textbf{Hale, V. L.}, \textbf{J. Chen}, \textbf{S. Johnson},
\textbf{S. C. Harrington}, \textbf{T. C. Yab}, \textbf{T. C. Smyrk},
\textbf{H. Nelson}, \textbf{L. A. Boardman}, \textbf{B. R. Druliner},
\textbf{T. R. Levin}, \textbf{D. K. Rex}, \textbf{D. J. Ahnen},
\textbf{P. Lance}, \textbf{D. A. Ahlquist}, and \textbf{N. Chia}. 2017.
Shifts in the Fecal Microbiota Associated with Adenomatous Polyps.
Cancer Epidemiology, Biomarkers \& Prevention: A Publication of the
American Association for Cancer Research, Cosponsored by the American
Society of Preventive Oncology \textbf{26}:85--94.

\hypertarget{ref-shah_leveraging_2017}{}
28. \textbf{Shah, M. S.}, \textbf{T. Z. DeSantis}, \textbf{T.
Weinmaier}, \textbf{P. J. McMurdie}, \textbf{J. L. Cope}, \textbf{A.
Altrichter}, \textbf{J.-M. Yamal}, and \textbf{E. B. Hollister}. 2017.
Leveraging sequence-based faecal microbial community survey data to
identify a composite biomarker for colorectal cancer. Gut.

\hypertarget{ref-brim_microbiome_2013}{}
29. \textbf{Brim, H.}, \textbf{S. Yooseph}, \textbf{E. G. Zoetendal},
\textbf{E. Lee}, \textbf{M. Torralbo}, \textbf{A. O. Laiyemo},
\textbf{B. Shokrani}, \textbf{K. Nelson}, and \textbf{H. Ashktorab}.
2013. Microbiome analysis of stool samples from African Americans with
colon polyps. PloS One \textbf{8}:e81352.

\hypertarget{ref-Hannigan2017}{}
30. \textbf{Hannigan, G. D.}, \textbf{M. B. Duhaime}, \textbf{M. T.
Ruffin}, \textbf{C. C. Koumpouras}, and \textbf{P. D. Schloss}. 2017.
Diagnostic potential \& the interactive dynamics of the colorectal
cancer virome. Cold Spring Harbor Laboratory.

\hypertarget{ref-Venkataraman2016}{}
31. \textbf{Venkataraman, A.}, \textbf{J. R. Sieber}, \textbf{A. W.
Schmidt}, \textbf{C. Waldron}, \textbf{K. R. Theis}, and \textbf{T. M.
Schmidt}. 2016. Variable responses of human microbiomes to dietary
supplementation with resistant starch. Microbiome. Springer Nature
\textbf{4}.

\hypertarget{ref-Herrmann2018}{}
32. \textbf{Herrmann, E.}, \textbf{W. Young}, \textbf{V.
Reichert-Grimm}, \textbf{S. Weis}, \textbf{C. Riedel}, \textbf{D.
Rosendale}, \textbf{H. Stoklosinski}, \textbf{M. Hunt}, and \textbf{M.
Egert}. 2018. In vivo assessment of resistant starch degradation by the
caecal microbiota of mice using RNA-based stable isotope probingA
proof-of-principle study. Nutrients. MDPI AG \textbf{10}:179.

\hypertarget{ref-Reichardt2017}{}
33. \textbf{Reichardt, N.}, \textbf{M. Vollmer}, \textbf{G. Holtrop},
\textbf{F. M. Farquharson}, \textbf{D. Wefers}, \textbf{M. Bunzel},
\textbf{S. H. Duncan}, \textbf{J. E. Drew}, \textbf{L. M. Williams},
\textbf{G. Milligan}, \textbf{T. Preston}, \textbf{D. Morrison},
\textbf{H. J. Flint}, and \textbf{P. Louis}. 2017. Specific
substrate-driven changes in human faecal microbiota composition contrast
with functional redundancy in short-chain fatty acid production. The
ISME Journal. Springer Nature \textbf{12}:610--622.

\hypertarget{ref-Sze2017}{}
34. \textbf{Sze, M. A.}, \textbf{N. T. Baxter}, \textbf{M. T. Ruffin},
\textbf{M. A. M. Rogers}, and \textbf{P. D. Schloss}. 2017.
Normalization of the microbiota in patients after treatment for colonic
lesions. Microbiome. Springer Nature \textbf{5}.

\hypertarget{ref-Salter_contamination_2014}{}
35. \textbf{Salter, S. J.}, \textbf{M. J. Cox}, \textbf{E. M. Turek},
\textbf{S. T. Calus}, \textbf{W. O. Cookson}, \textbf{M. F. Moffatt},
\textbf{P. Turner}, \textbf{J. Parkhill}, \textbf{N. J. Loman}, and
\textbf{A. W. Walker}. 2014. Reagent and laboratory contamination can
critically impact sequence-based microbiome analyses. BMC Biology.
Springer Nature \textbf{12}.

\hypertarget{ref-Purcell2017}{}
36. \textbf{Purcell, R. V.}, \textbf{J. Pearson}, \textbf{A. Aitchison},
\textbf{L. Dixon}, \textbf{F. A. Frizelle}, and \textbf{J. I. Keenan}.
2017. Colonization with enterotoxigenic bacteroides fragilis is
associated with early-stage colorectal neoplasia. PLOS ONE. Public
Library of Science (PLoS) \textbf{12}:e0171602.

\hypertarget{ref-Sze2016}{}
37. \textbf{Sze, M. A.}, and \textbf{P. D. Schloss}. 2016. Looking for a
signal in the noise: Revisiting obesity and the microbiome. mBio.
American Society for Microbiology \textbf{7}:e01018--16.

\hypertarget{ref-Walters2014}{}
38. \textbf{Walters, W. A.}, \textbf{Z. Xu}, and \textbf{R. Knight}.
2014. Meta-analyses of human gut microbes associated with obesity and
IBD. FEBS Letters. Wiley-Blackwell \textbf{588}:4223--4233.

\hypertarget{ref-Finucane2014}{}
39. \textbf{Finucane, M. M.}, \textbf{T. J. Sharpton}, \textbf{T. J.
Laurent}, and \textbf{K. S. Pollard}. 2014. A taxonomic signature of
obesity in the microbiome? Getting to the guts of the matter. PLoS ONE.
Public Library of Science (PLoS) \textbf{9}:e84689.

\hypertarget{ref-keku_gastrointestinal_2015}{}
40. \textbf{Keku, T. O.}, \textbf{S. Dulal}, \textbf{A. Deveaux},
\textbf{B. Jovov}, and \textbf{X. Han}. 2015. The gastrointestinal
microbiota and colorectal cancer. American Journal of Physiology -
Gastrointestinal and Liver Physiology \textbf{308}:G351--G363.

\hypertarget{ref-vogtmann_epidemiologic_2016}{}
41. \textbf{Vogtmann, E.}, and \textbf{J. J. Goedert}. 2016.
Epidemiologic studies of the human microbiome and cancer. British
Journal of Cancer \textbf{114}:237--242.

\hypertarget{ref-kostic_genomic_2012}{}
42. \textbf{Kostic, A. D.}, \textbf{D. Gevers}, \textbf{C. S.
Pedamallu}, \textbf{M. Michaud}, \textbf{F. Duke}, \textbf{A. M. Earl},
\textbf{A. I. Ojesina}, \textbf{J. Jung}, \textbf{A. J. Bass},
\textbf{J. Tabernero}, \textbf{J. Baselga}, \textbf{C. Liu}, \textbf{R.
A. Shivdasani}, \textbf{S. Ogino}, \textbf{B. W. Birren}, \textbf{C.
Huttenhower}, \textbf{W. S. Garrett}, and \textbf{M. Meyerson}. 2012.
Genomic analysis identifies association of Fusobacterium with colorectal
carcinoma. Genome Research \textbf{22}:292--298.

\hypertarget{ref-zackular_human_2014}{}
43. \textbf{Zackular, J. P.}, \textbf{M. A. M. Rogers}, \textbf{M. T.
Ruffin}, and \textbf{P. D. Schloss}. 2014. The human gut microbiome as a
screening tool for colorectal cancer. Cancer Prevention Research
(Philadelphia, Pa.) \textbf{7}:1112--1121.

\hypertarget{ref-schloss_introducing_2009}{}
44. \textbf{Schloss, P. D.}, \textbf{S. L. Westcott}, \textbf{T.
Ryabin}, \textbf{J. R. Hall}, \textbf{M. Hartmann}, \textbf{E. B.
Hollister}, \textbf{R. A. Lesniewski}, \textbf{B. B. Oakley}, \textbf{D.
H. Parks}, \textbf{C. J. Robinson}, \textbf{J. W. Sahl}, \textbf{B.
Stres}, \textbf{G. G. Thallinger}, \textbf{D. J. Van Horn}, and
\textbf{C. F. Weber}. 2009. Introducing mothur: Open-Source,
Platform-Independent, Community-Supported Software for Describing and
Comparing Microbial Communities. Appl.Environ.Microbiol.
\textbf{75}:7537--7541.

\hypertarget{ref-rognes_vsearch_2016}{}
45. \textbf{Rognes, T.}, \textbf{T. Flouri}, \textbf{B. Nichols},
\textbf{C. Quince}, and \textbf{F. Mahé}. 2016. VSEARCH: A versatile
open source tool for metagenomics. PeerJ \textbf{4}:e2584.

\hypertarget{ref-westcott_opticlust_2017}{}
46. \textbf{Westcott, S. L.}, and \textbf{P. D. Schloss}. 2017.
OptiClust, an Improved Method for Assigning Amplicon-Based Sequence Data
to Operational Taxonomic Units. mSphere \textbf{2}.

\hypertarget{ref-rdp_Wang2007}{}
47. \textbf{Wang, Q.}, \textbf{G. M. Garrity}, \textbf{J. M. Tiedje},
and \textbf{J. R. Cole}. 2007. Naive bayesian classifier for rapid
assignment of rRNA sequences into the new bacterial taxonomy. Applied
and Environmental Microbiology. American Society for Microbiology
\textbf{73}:5261--5267.

\hypertarget{ref-permanova_Anderson2013}{}
48. \textbf{Anderson, M. J.}, and \textbf{D. C. I. Walsh}. 2013.
PERMANOVA, ANOSIM, and the mantel test in the face of heterogeneous
dispersions: What null hypothesis are you testing? Ecological
Monographs. Wiley-Blackwell \textbf{83}:557--574.

\hypertarget{ref-benjamini_controlling_1995}{}
49. \textbf{Benjamini, Y.}, and \textbf{Y. Hochberg}. 1995. Controlling
the false discovery rate: A practical and powerful approach to multiple
testing. Journal of the Royal Statistical Society. Series B
(Methodological) \textbf{57}:289--300.

\hypertarget{ref-r_citation_2017}{}
50. \textbf{R Core Team}. 2017. R: A language and environment for
statistical computing. R Foundation for Statistical Computing, Vienna,
Austria.

\hypertarget{ref-rcompanion_citation_2017}{}
51. \textbf{Mangiafico, S.} 2017. Rcompanion: Functions to support
extension education program evaluation.

\hypertarget{ref-car_citation_2011}{}
52. \textbf{Fox, J.}, and \textbf{S. Weisberg}. 2011. An R companion to
applied regressionSecond. Sage, Thousand Oaks CA.

\hypertarget{ref-lme4_citation_2015}{}
53. \textbf{Bates, D.}, \textbf{M. Mächler}, \textbf{B. Bolker}, and
\textbf{S. Walker}. 2015. Fitting linear mixed-effects models using
lme4. Journal of Statistical Software \textbf{67}:1--48.

\hypertarget{ref-epir_citation_2017}{}
54. \textbf{Telmo Nunes, M. S. with contributions from}, \textbf{C.
Heuer}, \textbf{J. Marshall}, \textbf{J. Sanchez}, \textbf{R. Thornton},
\textbf{J. Reiczigel}, \textbf{J. Robison-Cox}, \textbf{P. Sebastiani},
\textbf{P. Solymos}, \textbf{K. Yoshida}, \textbf{G. Jones}, \textbf{S.
Pirikahu}, \textbf{S. Firestone}, and \textbf{R. Kyle.} 2017. EpiR:
Tools for the analysis of epidemiological data.

\hypertarget{ref-metafor_citation_2010}{}
55. \textbf{Viechtbauer, W.} 2010. Conducting meta-analyses in R with
the metafor package. Journal of Statistical Software \textbf{36}:1--48.

\hypertarget{ref-vegan_citation_2017}{}
56. \textbf{Oksanen, J.}, \textbf{F. G. Blanchet}, \textbf{M. Friendly},
\textbf{R. Kindt}, \textbf{P. Legendre}, \textbf{D. McGlinn}, \textbf{P.
R. Minchin}, \textbf{R. B. O'Hara}, \textbf{G. L. Simpson}, \textbf{P.
Solymos}, \textbf{M. H. H. Stevens}, \textbf{E. Szoecs}, and \textbf{H.
Wagner}. 2017. Vegan: Community ecology package.

\hypertarget{ref-caret_citation_2017}{}
57. \textbf{Jed Wing, M. K. C. from}, \textbf{S. Weston}, \textbf{A.
Williams}, \textbf{C. Keefer}, \textbf{A. Engelhardt}, \textbf{T.
Cooper}, \textbf{Z. Mayer}, \textbf{B. Kenkel}, \textbf{the R Core
Team}, \textbf{M. Benesty}, \textbf{R. Lescarbeau}, \textbf{A. Ziem},
\textbf{L. Scrucca}, \textbf{Y. Tang}, \textbf{C. Candan}, and
\textbf{T. Hunt.} 2017. Caret: Classification and regression training.

\hypertarget{ref-randomforest_citation_2002}{}
58. \textbf{Liaw, A.}, and \textbf{M. Wiener}. 2002. Classification and
regression by randomForest. R News \textbf{2}:18--22.

\hypertarget{ref-ggplot2_citation_2009}{}
59. \textbf{Wickham, H.} 2009. Ggplot2: Elegant graphics for data
analysis. Springer-Verlag New York.

\hypertarget{ref-gridextra_citation_2017}{}
60. \textbf{Auguie, B.} 2017. GridExtra: Miscellaneous functions for
``grid'' graphics.

\newpage

\textbf{Table 1: Total Individuals in each Study Included in the Stool
Analysis}

\footnotesize

\begin{longtable}[]{@{}cccccc@{}}
\toprule
Study & Data Stored & Region & Control (n) & Adenoma (n) & Carcinoma
(n)\tabularnewline
\midrule
\endhead
Ahn & DBGap & V3-4 & 148 & 0 & 62\tabularnewline
Baxter & SRA & V4 & 172 & 198 & 120\tabularnewline
Brim & SRA & V1-3 & 6 & 6 & 0\tabularnewline
Flemer & Author & V3-4 & 37 & 0 & 43\tabularnewline
Hale & Author & V3-5 & 473 & 214 & 17\tabularnewline
Wang & SRA & V3 & 56 & 0 & 46\tabularnewline
Weir & Author & V4 & 4 & 0 & 7\tabularnewline
Zeller & SRA & V4 & 50 & 37 & 41\tabularnewline
\bottomrule
\end{longtable}

\normalsize
\newpage

\textbf{Table 2: Studies with Tissue Samples Included in the Analysis}

\footnotesize

\begin{longtable}[]{@{}cccccc@{}}
\toprule
Study & Data Stored & Region & Control (n) & Adenoma (n) & Carcinoma
(n)\tabularnewline
\midrule
\endhead
Burns & SRA & V5-6 & 18 & 0 & 16\tabularnewline
Chen & SRA & V1-3 & 9 & 0 & 9\tabularnewline
Dejea & SRA & V3-5 & 31 & 0 & 32\tabularnewline
Flemer & Author & V3-4 & 103 & 37 & 94\tabularnewline
Geng & SRA & V1-2 & 16 & 0 & 16\tabularnewline
Lu & SRA & V3-4 & 20 & 20 & 0\tabularnewline
Sanapareddy & Author & V1-2 & 38 & 0 & 33\tabularnewline
\bottomrule
\end{longtable}

\normalsize
\newpage

\textbf{Figure 1: Significant Bacterial Community Metrics for Adenoma or
Carcinoma in Stool.} A) Adenoma evenness. B) Carcinoma evenness. C)
Carcinoma Shannon diversity. Blue represents controls and red represents
either adenoma (panel A) or carcinoma (panel B and C). The black lines
represent the median value for each repsective group.

\textbf{Figure 2: Odds Ratio for Adenoma or Carcinoma based on Bacterial
Community Metrics in Stool.} A) Community-based odds ratio for adenoma.
B) Community-based odds ratio for carcinoma. Colors represent the
different variable regions used within the respective study.

\textbf{Figure 3: The AUC of Indivdiual Significant OR Taxa to classify
Carcinoma.} A) Stool samples. B) Unmatched tissue samples. The larger
circle represents the median AUC of all studies and the smaller circles
represent the individual AUC for a particular study. The dotted line
denotes an AUC of 0.5.

\textbf{Figure 4: Stool Random Forest Model Train AUCs.} A) Adenoma
random forest model AUCs between all genera, all OTU, and select model
based on significant OR taxa. B) Carcinoma random forest model AUCs
between all genera, all OTU, and select model based on significant OR
taxa. The black line represents the median AUC for the respective group.
If no values are present in the singificant OR taxa group then there
were no significant taxa identified and no model was tested.

\textbf{Figure 5: Most Important Members in Significant OR Taxa
Carcinoma Models.} A) Common taxa in the top 10 percent for carcinoma
Random Forest stool-based models. B) Common taxa in the top 10 percent
for carcinoma Random Forest unmatched tissue-based models. Blue
represents less important and red represents more important to the
Random Forest Model. White means that particular taxa was not in the top
10\%.

\textbf{Figure 6: Stool Random Forest Genus-Based Model Test AUCs.} A)
Test AUCs of adenoma models using all genera across study. B) Test AUCs
of carcinoma models using all genera or significant OR taxa only. The
black line represents the AUC at 0.5. The red lines represent the median
AUC of all test AUCs for a specific study.

\newpage

\textbf{Figure S1: Odds Ratio for Adenoma or Carcinoma based on
Bacterial Community Metrics in Tissue.} A) Community-based odds ratio
for adenoma. B) Community-based odds ratio for carcinoma. Colors
represent the different variable regions used within the respective
study.

\textbf{Figure S2: Most Common Taxa Across Carcinoma Full Community
Stool Study Models.} A) Common taxa in the top 10 percent for carcinoma
Random Forest all taxa-based models. B) Common taxa in the top 10
percent for carcinoma Random Forest all OTU-based models. Blue
represents less important and red represents more important to the
Random Forest Model. White means that particular taxa was not in the top
10\%.

\textbf{Figure S3: Most Common Genera Across Full Community Tissue Study
Models.} A) Common genera in the top 10 percent for matched carcinoma
Random Forest all genera-based models. B) Common genera in the top 10
percent for unmatched carcinoma Random Forest all genera-based models.
C) Common genera in the top 10 percent for matched carcinoma Random
Forest all OTU-based models. D) Common genera in the top 10 percent for
unmatched carcinoma Random Forest all OTU-based models. Blue represents
less important and red represents more important to the Random Forest
Model. White means that particular taxa was not in the top 10\%.

\textbf{Figure S4: Tissue Random Forest Model Train AUCs.} A) Adenoma
random forest model AUCs between all genera, all OTU, and select model
based on significant OR taxa in unmatched and matched tissue. B)
Carcinoma random forest model AUCs between all genera, all OTU, and
select model based on significant OR taxa in unmatched and matched
tissue. The black line represents the median AUC for the respective
group. If no values are present in the singificant OR taxa group then
there were no significant taxa identified and no model was tested.

\textbf{Figure S5: Tissue Random Forest Genus-Based Model Test AUCs.} A)
Test AUCs of adenoma models using all genera across study. B) Test AUCs
of carcinoma models using all genera for matched tissue studies. C) Test
AUCs of carcinoma models using all genera or significant OR taxa only
for unmatched tissue studies The black line represents the AUC at 0.5.
The red lines represent the median AUC of all test AUCs for a specific
study.

\newpage


\end{document}
