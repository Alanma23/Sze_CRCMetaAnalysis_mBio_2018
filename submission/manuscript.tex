\documentclass[12pt,]{article}
\usepackage{lmodern}
\usepackage{amssymb,amsmath}
\usepackage{ifxetex,ifluatex}
\usepackage{fixltx2e} % provides \textsubscript
\ifnum 0\ifxetex 1\fi\ifluatex 1\fi=0 % if pdftex
  \usepackage[T1]{fontenc}
  \usepackage[utf8]{inputenc}
\else % if luatex or xelatex
  \ifxetex
    \usepackage{mathspec}
  \else
    \usepackage{fontspec}
  \fi
  \defaultfontfeatures{Ligatures=TeX,Scale=MatchLowercase}
\fi
% use upquote if available, for straight quotes in verbatim environments
\IfFileExists{upquote.sty}{\usepackage{upquote}}{}
% use microtype if available
\IfFileExists{microtype.sty}{%
\usepackage{microtype}
\UseMicrotypeSet[protrusion]{basicmath} % disable protrusion for tt fonts
}{}
\usepackage[margin=1.0in]{geometry}
\usepackage{hyperref}
\hypersetup{unicode=true,
            pdfborder={0 0 0},
            breaklinks=true}
\urlstyle{same}  % don't use monospace font for urls
\usepackage{longtable,booktabs}
\usepackage{graphicx,grffile}
\makeatletter
\def\maxwidth{\ifdim\Gin@nat@width>\linewidth\linewidth\else\Gin@nat@width\fi}
\def\maxheight{\ifdim\Gin@nat@height>\textheight\textheight\else\Gin@nat@height\fi}
\makeatother
% Scale images if necessary, so that they will not overflow the page
% margins by default, and it is still possible to overwrite the defaults
% using explicit options in \includegraphics[width, height, ...]{}
\setkeys{Gin}{width=\maxwidth,height=\maxheight,keepaspectratio}
\IfFileExists{parskip.sty}{%
\usepackage{parskip}
}{% else
\setlength{\parindent}{0pt}
\setlength{\parskip}{6pt plus 2pt minus 1pt}
}
\setlength{\emergencystretch}{3em}  % prevent overfull lines
\providecommand{\tightlist}{%
  \setlength{\itemsep}{0pt}\setlength{\parskip}{0pt}}
\setcounter{secnumdepth}{0}
% Redefines (sub)paragraphs to behave more like sections
\ifx\paragraph\undefined\else
\let\oldparagraph\paragraph
\renewcommand{\paragraph}[1]{\oldparagraph{#1}\mbox{}}
\fi
\ifx\subparagraph\undefined\else
\let\oldsubparagraph\subparagraph
\renewcommand{\subparagraph}[1]{\oldsubparagraph{#1}\mbox{}}
\fi

%%% Use protect on footnotes to avoid problems with footnotes in titles
\let\rmarkdownfootnote\footnote%
\def\footnote{\protect\rmarkdownfootnote}

%%% Change title format to be more compact
\usepackage{titling}

% Create subtitle command for use in maketitle
\newcommand{\subtitle}[1]{
  \posttitle{
    \begin{center}\large#1\end{center}
    }
}

\setlength{\droptitle}{-2em}
  \title{}
  \pretitle{\vspace{\droptitle}}
  \posttitle{}
  \author{}
  \preauthor{}\postauthor{}
  \date{}
  \predate{}\postdate{}

\usepackage{helvet} % Helvetica font
\renewcommand*\familydefault{\sfdefault} % Use the sans serif version of the font
\usepackage[T1]{fontenc}

\usepackage[none]{hyphenat}

\usepackage{setspace}
\doublespacing
\setlength{\parskip}{1em}

\usepackage{lineno}

\usepackage{pdfpages}

\usepackage{amsmath}

\usepackage{mathtools}

\begin{document}

\section{Investigating the Microbiota and Colorectal Cancer: The
Importance of
Community}\label{investigating-the-microbiota-and-colorectal-cancer-the-importance-of-community}

\begin{center}
\vspace{25mm}

Marc A Sze${^1}$ and Patrick D Schloss${^1}$${^\dagger}$

\vspace{20mm}

$\dagger$ To whom correspondence should be addressed: pschloss@umich.edu

$1$ Department of Microbiology and Immunology, University of Michigan, Ann Arbor, MI




\end{center}

Co-author e-mails:

\begin{itemize}
\tightlist
\item
  \href{mailto:marcsze@med.umich.edu}{\nolinkurl{marcsze@med.umich.edu}}
\end{itemize}

\newpage

\linenumbers

\subsection{Abstract}\label{abstract}

\textbf{Background.} An increasing body of literature suggests that
there is a role for the microbiota in colorectal cancer (CRC). Important
drivers within this context have ranged from individual microbes to the
whole community. Our study expands on a recent meta-analysis
investigating microbial biomarkers for CRC by testing the hypothesis
that the bacterial community is an important driver of both early
(adenoma) and late (carcinoma) stage of disease. To test this hypothesis
we examined both feces (n = 1737) and tissue (492 total samples from 350
individuals) across 14 different studies.

\textbf{Results.} Fecal samples had a significant decrease from control
to adenoma to carcinoma for both Shannon diversity and evenness (P-value
\textless{} 0.05) after correcting for study effect and variable region
sequenced. Only evenness for adenoma (P-value \textless{} 0.05) resulted
in a slightly increased relative risk while lower Shannon diversity and
evenness in fecal samples resulted in a significant increase in relative
risk for carcinoma (P-value \textless{} 0.05). Previously associated
colorectal cancer genera (\emph{Fusobacterium}, \emph{Parvimonas},
\emph{Peptostreptococcus}, or \emph{Porphyromonas}) followed a similar
pattern with a significantly increased relative risk by their presence
for carcinoma (P-value \textless{} 0.05) but not adenoma (P-value
\textgreater{} 0.05) with the exception of \emph{Porphyromonas} (P-value
\textless{} 0.05). Using the whole community versus only CRC associated
genera to build a prediction model resulted in a higher classification
based on Area Under the Curve (AUC) for both adenoma and carcinoma using
fecal and tissue samples. For the included studies, most were adequately
powered for large effect size differences. This may be more amenable for
carcinoma than for adenoma microbiota research due to the smaller
community level changes observed in our results.

\textbf{Conclusions.} This data provides support for the importance of
the bacteral community to both adenoma and carcinoma genesis. The
evidence collected within this study on the role of the microbiota in
CRC pathogenesis is much stronger for carcinoma then adenoma. A strong
reason for this may be in part due to the low power to detect more
subtle changes in the majority of studies that have been performed to
date.

\subsubsection{Keywords}\label{keywords}

microbiota; colorectal cancer; polyps; adenoma; meta-analysis.

\newpage

\subsection{Background}\label{background}

Colorectal cancer (CRC) is a growing world-wide health problem {[}1{]}
in which the microbiota has been purported to play an active role in
disease pathogenesis {[}2{]}. Numerous studies have shown the importance
of both individual microbes {[}3--7{]} and the overall community
{[}8--10{]} in polyp formation using mouse models of CRC. There have
also been numerous case/control studies investigating the microbiota in
the formation of both adenoma and carcinoma. Recently, a meta-analysis
was published investigating whether specific biomarkers could be
consistently identified using multiple data sets {[}11{]}. Many of the
studies, along with the previous meta-analysis, focus on identifying
biomarkers or individual microbes but do not critically investigate the
role the community has in CRC.

Using both feces (n = 1737) and tissue (492 samples from 350
individuals) totalling over 2229 total samples across 14 studies
{[}12--25{]} {[}Table 1 \& 2{]}, we expand both the breadth and scope of
the previous meta-analysis to investigate whether the bacterial
community is an important risk factor for both adenoma and carcinoma. To
accomplish this we first assessed whether the diversity changes
throughout disease (control to adenoma to carcinoma) and if it results
in an increased relative risk (RR) for adenoma or carcinoma. Next, we
assessed how typical CRC associated genera (\emph{Fusobacterium},
\emph{Parvimonas}, \emph{Peptostreptococcus}, or \emph{Porphyromonas})
affect the relative risk of adenoma or carcinoma. Third, using Random
Forest models, we analyzed whether the full community or only the CRC
associated genera resulted in better model classification based on the
area under the curve (AUC). Since the changes in community were subtle
for adenoma we also examined what effect size and sample size the
studies that were used were adequately powered for.

Our analysis found a continuous decrease in Shannon diversity as disease
became more severe which correlated with a significantly increased RR
for carcinoma. Using only CRC-associated genera, the RR for carcinoma
was higher relative to what was observed for the Shannon diversity RR.
Conversely, we demonstrate that the AUC of the classification models
increased when the full community was incorporated as opposed to only
CRC-associated genera. Although we analyzed data sets which sampled
large numbers of individuals, our results indicate the individual
studies were underpowered for detecting effect size differences of 10\%
or below between the case and control groups.

\newpage

\subsection{Results}\label{results}

\textbf{\emph{Lower Community Diversity is Associated with Increased RR
of Carcinomas:}} Using power transformed and Z-score normalized
\(\alpha\)-diversity metrics, both evenness and Shannon diversity in
feces, but not tissue, were lower in those with carcinoma {[}Figure
1{]}. Using linear mixed-effect models to control for study and variable
region, there was a significant decrease from control to adenoma to
carcinoma for both evenness (P-value = 0.025) and Shannon diversity
(P-value = 0.043). However, in tissue, this effect was not observed in
tissue when resampling of the same individual was also controlled for
(P-value \textgreater{} 0.05). Within fecal samples, a decrease in
Shannon diversity and evenness resulted in a significantly increased RR
for carcinoma (P-value = 0.01 and 0.0011, respectively) {[}Figure 2{]}.
Although these values were significant, the effect size was relatively
small for both metrics (Shannon RR = 1.31 and evenness RR = 1.34)
{[}Figure 2{]}. Only a decrease in evenness had an increased RR for
adenoma (P-value = 0.032) {[}Figure 2A \& S1{]} but this effect size was
even smaller than what was observed for carcinoma (RR = 1.16).

Using the Bray-Curtis distance metric, there was a significant
difference across studies in the bacterial community of fecal samples
between carcinoma and controls, but not adenoma and controls {[}Table S1
\& S2{]}. For studies with unmatched tissue samples a similar trend was
observed {[}Table S3 \& S4{]} while studies with matched tissue samples
had no differences {[}Table S3 \& S4{]}.

\textbf{\emph{Carcinoma-Associated Genera Minimally Impacts RR of
Adenoma:}} The majority of CRC-associated genera for both feces and
tissue had a significantly increased RR for carcinoma but not for
adenoma {[}Figure 3{]}. In fecal samples the RR due to CRC associated
genera was greater than either the RR assoicated with evenness or
Shannon diversity {[}Figure 2 \& 3{]}. Additionally, the RR of carcinoma
continuously increased as individuals tested positive for more CRC
associated genera {[}Figure 3B \& 3D{]}. The RR effect size was greater
for stool (RR range = 1.61 - 2.74) than for tissue (RR range = 1.21 -
1.81). This decrease may be explained by the fact that the tissue
analysis included matched samples.

There were two significant measures for increased RR of adenoma when
investigating CRC-associated genera in stool: 1) Having a higher then
median value of \emph{Porphyromonas} (P-value = 0.023) and 2) whether
samples were positive for three CRC associated genera (P-value = 0.022)
{[}Figure 3A{]}. With tissue, there were three significant measures for
an increased RR of adenoma: 1) being positive for one CRC-associated
genera (P-value = 0.032), 2) being positive for two CRC associated
genera (P-value = 0.008), and 3) being positive for four CRC associated
genera (P-value = 0.039) {[}Figure 3C{]}.

\textbf{\emph{Using the Whole Community Instead of Only CRC-Associated
Genera Increases Model AUC:}} For both fecal and tissue (matched and
unmatched) samples, the AUC decreases when only OTUs from the
CRC-associated genera are used {[}Figure 4 \& 5{]}. This decrease is
observed in both adenoma and carcinoma groups {[}Figure 4 \& 5{]}. The
genus models generally performed similar to the OTU-based models with
the full genera models performing better than the CRC-associated genera
models {[}Figure S2-S3{]}. Both genus models were similarily predictive
in their ability to detect adenomas or carcinomas, with carcinoma having
a higher AUC then adenoma {[}Figure S4-S5{]}. Of note, matched tissue
samples for those with carcinoma had an AUC that was more similar to the
adenoma models {[}Figure S4A, S5B, \& S6{]} than carcinoma models
{[}Figure S4B \& S5A{]}.

\textbf{\emph{A Majority of Studies are Underpowered for Detecting Small
Effect Size Differences:}} When assessing the power of each study at
different effect sizes the majority of studies for both adenoma and
carcinoma have an 80\% power to detect a 30\% difference {[}Figure 6A \&
B{]}. No single study that was analyzed had the standard 80\% power to
detect an effect size difference that was equal to or below 10\%
{[}Figure 6A \& B{]}. In order to achieve adequate power for small
effect sizes, studies would need to recruit over 1000 individuals for
each arm {[}Figure 6C{]}.

\newpage

\subsection{Discussion}\label{discussion}

Our study identifies clear differences in diversity both at the
community level and within individual genera that are present in
individuals with CRC versus those without the disease. Although there
was a step-wise decrease in diversity from control to adenoma to
carcinoma, this did not translate into large effect sizes for the
relative risk of either of these two conditions. Even though
CRC-associated genera increase the RR of carcinoma, they do not
consistently increase the relative risk of adenoma. This information
suggests that these specific genera are important in carcinoma genesis
but may not be the primary members of the microbial community
contributing to the formation of an adenoma. Additionally, our data show
that by using the whole community, our models perform better than when
only the CRC-associated genera are included. CRC-associated genera are
clearly important to carcinoma pathogenesis but accounting for the
community in which these microbes exist can drastically increase the
ability of models to make predictions. These observations suggest that
small localized changes within the community may be occuring that are
important in early disease progression of CRC and that this process may
not directly involve CRC-associated genera.

The data presented herein supports the driver-passenger model of the
microbial role in CRC, as summarized by Flynn {[}2{]}, when applied to
carcinoma but not necessarily adenoma. Both the drastically increased RR
of CRC-associated genera versus diversity for carcinoma and increasing
RR with more CRC-associated generea positivity are highly supportive of
this model. It is also possible that in a driver-passenger scenario,
simply having the driver present or only identifying the passenger is a
good enough proxy that the event is occuring. This would account for the
observation that there is no constant additive effect on RR for
increasing positivity. Additionally, the initial establishment of the
driver within the system appears to be dependent on the current
community. This is supported by our finding that when adding the
community context to our models in addition to the CRC-associated
genera, the model AUC increases.

With regards to carcinoma, our results support the driver-passenger
model within the framework of the transition from adenoma to carcinoma.
Conversely, using the present data, we can only theorize observations
related to adenoma development fit this model. The changes that occur at
this timepoint are small and possibly focal to the adenoma itself. The
stepwise decrease in diversity suggests that the adenoma community is
not normal but has changed subtly. Although there appears to be
localized changes that do depend on the driver-passenger model, as
supported by an increased RR for one, two, and four positive
CRC-associated genera in tissue {[}Figure 3C{]}, there may be other
processes at play that ultimately exacerbate the condition from a subtle
localized change to a change in the global community. The poor
performance of the Random Forest models for classifying adenoma based
only on the microbiota would suggest that this is the case. It is
possible to hypothesize that at early stages of the diease, the host
interacts with these subtle changes and catalyzes what ultimately leads
to a thoroughly dysfunctional community.

Although there are still questions that need to be answered regarding
the microbiota and carcinoma, a clearer framework for their relationship
is beginning to develop as to how this occurs. From our observations
many changes in carcinoma could easily result in effect sizes that are
30\% or more between the case and control and most studies analyzed have
sufficient power to detect these types of changes {[}Figure 6{]}.
Conversely, the role of the microbiota in adenoma is less clear and part
of the reason this may be is because many studies are not powered
effectively to observe the small changes reported here. None of the
studies analyzed were properly powered to detect a 10\% or lower change
between case and controls. This small effect size range may well be the
scope in which differences consistently occur in adenoma due to the
subtle changes in community that occur between control and adenoma.
Future studies investigating adenoma and the microbiota need to take
these factors into consideration if we are to work out the role of the
microbiota in adenoma formation.

\subsection{Conclusion}\label{conclusion}

By aggregating together a large collection of studies from both feces
and tissue, we are able to provide evidence in support of the importance
of the bacterial community in both adenoma and carcinoma. We are also
able to provide support for the driver-passenger model in the context of
carcinoma. However, within the context of adenoma, it is less clear that
this relationship exists. These observations highlight the importance of
power and sample number considerations when undertaking investigations
into the microbiota and adenoma due to the subtle changes in the
community. Although there are power limitations associated with adenoma,
this report highlights the strong influence the microbiota has on CRC
development.

\newpage

\subsection{Methods}\label{methods}

\textbf{\emph{Obtaining Data Sets:}} Studies used for this meta-analysis
were identified through the review articles written by Keku, \emph{et
al.} and Vogtmann, \emph{et al.} {[}26,27{]}. Additional studies not
mentioned in the reviews were obtained based on the authors' knowledge
of the literature. Studies that used tissue or feces as their sample
source for 16S rRNA gene sequencing analysis were included. Studies
using either 454 or Illumina sequencing technology were included. Only
data sets that had sequences available for analysis were included. Some
studies did not have publically available sequences or did not have
metadata in which the authors were able to share. After these filtering
steps, the following studies remained: Ahn, \emph{et al.} {[}21{]},
Baxter, \emph{et al.} {[}24{]}, Brim, \emph{et al.} {[}17{]}, Burns,
\emph{et al.} {[}22{]}, Chen, \emph{et al.} {[}14{]}, Dejea, \emph{et
al.} {[}19{]}, Flemer, \emph{et al.} {[}13{]}, Geng, \emph{et al.}
{[}25{]}, Hale, \emph{et al.} {[}12{]}, Kostic, \emph{et al.} {[}28{]},
Lu, \emph{et al.} {[}16{]}, Sanapareddy, \emph{et al.} {[}20{]}, Wang,
\emph{et al.} {[}15{]}, Weir, \emph{et al.} {[}18{]}, and Zeller,
\emph{et al.} {[}23{]}. The Zackular {[}29{]} study was not included
becasue the 90 individuals analyzed within the study are contained
within the larger Baxter study {[}24{]}. Additionally, after sequence
processing all the case samples for the Kostic study only had 100
sequences remaining and was not used. This left a total of 14 studies
for which analysis could be completed.

\textbf{\emph{Data Set Breakdown:}} In total, there were seven studies
with only fecal samples (Ahn, Baxter, Brim, Hale, Wang, Weir, and
Zeller), five studies with only tissue samples (Burns, Dejea, Geng, Lu,
Sanapareddy), and two studies with both fecal and tissue samples (Chen
and Flemer). The total number of individuals that were analyzed after
sequence processing for feces was 1737 {[}Table 1{]}. The total number
of matched and unmatched tissue samples that were analyzed after
sequence processing was 492 {[}Table 2{]}.

\textbf{\emph{Sequence Processing:}} For the majority of studies raw
sequences were downloaded from the Sequence Read Archive (SRA)
(\url{ftp://ftp-trace.ncbi.nih.gov/sra/sra-instant/reads/ByStudy/sra/SRP/})
and metadata was obtained from the by searching the respective accession
number of the study following website:
\url{http://www.ncbi.nlm.nih.gov/Traces/study/}. Of the studies that did
not have sequences and metadata on the SRA, data was obtained from DBGap
for one study {[}21{]} and for four studies was obtained directly from
the authors {[}12,13,18,20{]}. Each study was processed using the mothur
(v1.39.3) software program {[}30{]}. Where possible, quality filtering
utilized the default methods used in mothur for either 454 or Illumina
based sequencing. If it was not possible to use these defaults, the
stated quality cut-offs were used instead. Chimeras were identifed and
removed using VSEARCH {[}31{]} before \emph{de novo} OTU clustering at
97\% similarity using the OptiClust algorithm {[}32{]} was utilized.

\textbf{\emph{Statistical Analysis:}} All statistical analysis after
sequence processing utilized the R (v3.4.2) software package {[}33{]}.
For the \(\alpha\)-diversity analysis, values were power transformed
using the rcompanion (v1.10.1) package {[}34{]} and then Z-score
normalized using the car (v2.1.5) package {[}35{]}. Testing for
\(\alpha\)-diversity differences utilized linear mixed-effect models
created using the lme4 (v1.1.14) package {[}36{]} to correct for study
and variable region effects in feces and study, variable region, and
individual effects in tissue. Relative risk was analyzed using both the
epiR (v0.9.87) and metafor (v2.0.0) packages {[}37,38{]} by assessing
how many with and without disease were above and below the overall
median value within the specific study. Relative risk significance
testing utilized the chi-squred test. \(\beta\)-diversity differences
utilized a Bray-Curtis distance matrix and PERMANOVA executed with the
vegan (v2.4.4) package {[}39{]}. Random Forest models were built using
both the caret (v6.0.77) and randomForest (v4.6.12) packages
{[}40,41{]}. Differences between the obtained AUC versus a random model
AUC was assessed using T-tests. Power analysis and estimations were made
using the pwr (v1.2.1) and statmod (v1.4.30) packages {[}42,43{]}. All
figures were created using both ggplot2 (v2.2.1) and gridExtra (v2.3)
packages {[}44,45{]}.

\textbf{\emph{Study Analysis Overview:}} \(\alpha\)-diversity was first
assessed for differences between controls, adenoma, and carcinoma. We
analyzed the data using linear mixed-effect models and relative risk.
\(\beta\)-diversity was then assessed for each inidividual study. Next,
four specific CRC-associated genera (\emph{Fusobacterium},
\emph{Parvimonas}, \emph{Peptostreptococcus}, and \emph{Porphyromonas})
were assessed for differences in relative risk. We then built Random
Forest models based on all genera or the select CRC-associated genera.
The models were trained on one study then tested on the remaining
studies for every study. The data was split between feces and tissue
samples. Within the tissue groups the data was further divided between
matched and unmatched tissue samples. Where applicable for each study,
predictions for adenoma and carcinoma were tested. This same approach
was then applied at the OTU level with the exception that instead of
testing on the other studies, a 10-fold cross validation was utilized
and 100 different models were created based on random 80/20 splitting of
the data to generate a range of expected AUCs. For OTU based models the
CRC associated genera included all OTUs that had a taxonomic
classification to \emph{Fusobacterium}, \emph{Parvimonas},
\emph{Peptostreptococcus}, or \emph{Porphyromonas}. The power of each
study was assessed for an effect size ranging from 1\% to 30\%. An
estimated sample n for these effect sizes was also generated based on
80\% power.

\textbf{\emph{Reproducible Methods:}} The code and analysis can be found
here
\url{https://github.com/SchlossLab/Sze_CRCMetaAnalysis_Microbiome_2017}.
Unless mentioned otherwise, the accession number for the raw sequences
for the studies used in this analysis can be found directly in the
respective batch file in the GitHub repository or in the original
manuscript.

\newpage

\subsection{Declarations}\label{declarations}

\subsubsection{Ethics approval and consent to
participate}\label{ethics-approval-and-consent-to-participate}

Ethics approval and informed consent for each of the studies used is
mentioned in the respective manuscripts used in this meta-analysis.

\subsubsection{Consent for publication}\label{consent-for-publication}

Not applicable.

\subsubsection{Availability of data and
material}\label{availability-of-data-and-material}

A detailed and reproducible description of how the data were processed
and analyzed for each study can be found at
\url{https://github.com/SchlossLab/Sze_CRCMetaAnalysis_Microbiome_2017}.
Raw sequences can be downloaded from the SRA in most cases and can be
found in the respective studies batch file in the GitHub repository or
within the original publication. For instances when sequences are not
publicly available, they may be accessed by contacting the corresponding
authors from whence the data came.

\subsubsection{Competing Interests}\label{competing-interests}

All authors declare that they do not have any relevant competing
interests to report.

\subsubsection{Funding}\label{funding}

MAS is supported by a Candian Institute of Health Research fellowship
and a University of Michigan Postdoctoral Translational Scholar Program
grant.

\subsubsection{Authors' contributions}\label{authors-contributions}

All authors helped to design and conceptualize the study. MAS identified
and analyzed the data. MAS and PDS interpreted the data. MAS wrote the
first draft of the manuscript and both he and PDS reviewed and revised
updated versions. All authors approved the final manuscript.

\subsubsection{Acknowledgements}\label{acknowledgements}

The authors would like to thank all the study participants who were a
part of each of the individual studies uitlized. We would also like to
thank each of the study authors for making their data available for use.
Finally we would like to thank the members of the Schloss lab for
valuable feed back and proof reading during the formulation of this
manuscript.

\newpage

\subsection{References}\label{references}

\hypertarget{refs}{}
\hypertarget{ref-siegel_cancer_2016}{}
1. Siegel RL, Miller KD, Jemal A. Cancer statistics, 2016. CA: a cancer
journal for clinicians. 2016;66:7--30.

\hypertarget{ref-flynn_metabolic_2016}{}
2. Flynn KJ, Baxter NT, Schloss PD. Metabolic and Community Synergy of
Oral Bacteria in Colorectal Cancer. mSphere. 2016;1.

\hypertarget{ref-goodwin_polyamine_2011}{}
3. Goodwin AC, Destefano Shields CE, Wu S, Huso DL, Wu X, Murray-Stewart
TR, et al. Polyamine catabolism contributes to enterotoxigenic
Bacteroides fragilis-induced colon tumorigenesis. Proceedings of the
National Academy of Sciences of the United States of America.
2011;108:15354--9.

\hypertarget{ref-abed_fap2_2016}{}
4. Abed J, Emgård JEM, Zamir G, Faroja M, Almogy G, Grenov A, et al.
Fap2 Mediates Fusobacterium nucleatum Colorectal Adenocarcinoma
Enrichment by Binding to Tumor-Expressed Gal-GalNAc. Cell Host \&
Microbe. 2016;20:215--25.

\hypertarget{ref-arthur_intestinal_2012}{}
5. Arthur JC, Perez-Chanona E, Mühlbauer M, Tomkovich S, Uronis JM, Fan
T-J, et al. Intestinal inflammation targets cancer-inducing activity of
the microbiota. Science (New York, N.Y.). 2012;338:120--3.

\hypertarget{ref-kostic_fusobacterium_2013}{}
6. Kostic AD, Chun E, Robertson L, Glickman JN, Gallini CA, Michaud M,
et al. Fusobacterium nucleatum potentiates intestinal tumorigenesis and
modulates the tumor-immune microenvironment. Cell Host \& Microbe.
2013;14:207--15.

\hypertarget{ref-wu_human_2009}{}
7. Wu S, Rhee K-J, Albesiano E, Rabizadeh S, Wu X, Yen H-R, et al. A
human colonic commensal promotes colon tumorigenesis via activation of T
helper type 17 T cell responses. Nature Medicine. 2009;15:1016--22.

\hypertarget{ref-zackular_manipulation_2016}{}
8. Zackular JP, Baxter NT, Chen GY, Schloss PD. Manipulation of the Gut
Microbiota Reveals Role in Colon Tumorigenesis. mSphere. 2016;1.

\hypertarget{ref-zackular_gut_2013}{}
9. Zackular JP, Baxter NT, Iverson KD, Sadler WD, Petrosino JF, Chen GY,
et al. The gut microbiome modulates colon tumorigenesis. mBio.
2013;4:e00692--00613.

\hypertarget{ref-baxter_structure_2014}{}
10. Baxter NT, Zackular JP, Chen GY, Schloss PD. Structure of the gut
microbiome following colonization with human feces determines colonic
tumor burden. Microbiome. 2014;2:20.

\hypertarget{ref-shah_leveraging_2017}{}
11. Shah MS, DeSantis TZ, Weinmaier T, McMurdie PJ, Cope JL, Altrichter
A, et al. Leveraging sequence-based faecal microbial community survey
data to identify a composite biomarker for colorectal cancer. Gut. 2017;

\hypertarget{ref-hale_shifts_2017}{}
12. Hale VL, Chen J, Johnson S, Harrington SC, Yab TC, Smyrk TC, et al.
Shifts in the Fecal Microbiota Associated with Adenomatous Polyps.
Cancer Epidemiology, Biomarkers \& Prevention: A Publication of the
American Association for Cancer Research, Cosponsored by the American
Society of Preventive Oncology. 2017;26:85--94.

\hypertarget{ref-flemer_tumour-associated_2017}{}
13. Flemer B, Lynch DB, Brown JMR, Jeffery IB, Ryan FJ, Claesson MJ, et
al. Tumour-associated and non-tumour-associated microbiota in colorectal
cancer. Gut. 2017;66:633--43.

\hypertarget{ref-chen_human_2012}{}
14. Chen W, Liu F, Ling Z, Tong X, Xiang C. Human intestinal lumen and
mucosa-associated microbiota in patients with colorectal cancer. PloS
One. 2012;7:e39743.

\hypertarget{ref-wang_structural_2012}{}
15. Wang T, Cai G, Qiu Y, Fei N, Zhang M, Pang X, et al. Structural
segregation of gut microbiota between colorectal cancer patients and
healthy volunteers. The ISME journal. 2012;6:320--9.

\hypertarget{ref-lu_mucosal_2016}{}
16. Lu Y, Chen J, Zheng J, Hu G, Wang J, Huang C, et al. Mucosal
adherent bacterial dysbiosis in patients with colorectal adenomas.
Scientific Reports. 2016;6:26337.

\hypertarget{ref-brim_microbiome_2013}{}
17. Brim H, Yooseph S, Zoetendal EG, Lee E, Torralbo M, Laiyemo AO, et
al. Microbiome analysis of stool samples from African Americans with
colon polyps. PloS One. 2013;8:e81352.

\hypertarget{ref-weir_stool_2013}{}
18. Weir TL, Manter DK, Sheflin AM, Barnett BA, Heuberger AL, Ryan EP.
Stool microbiome and metabolome differences between colorectal cancer
patients and healthy adults. PloS One. 2013;8:e70803.

\hypertarget{ref-dejea_microbiota_2014}{}
19. Dejea CM, Wick EC, Hechenbleikner EM, White JR, Mark Welch JL,
Rossetti BJ, et al. Microbiota organization is a distinct feature of
proximal colorectal cancers. Proceedings of the National Academy of
Sciences of the United States of America. 2014;111:18321--6.

\hypertarget{ref-sanapareddy_increased_2012}{}
20. Sanapareddy N, Legge RM, Jovov B, McCoy A, Burcal L, Araujo-Perez F,
et al. Increased rectal microbial richness is associated with the
presence of colorectal adenomas in humans. The ISME journal.
2012;6:1858--68.

\hypertarget{ref-ahn_human_2013}{}
21. Ahn J, Sinha R, Pei Z, Dominianni C, Wu J, Shi J, et al. Human gut
microbiome and risk for colorectal cancer. Journal of the National
Cancer Institute. 2013;105:1907--11.

\hypertarget{ref-burns_virulence_2015}{}
22. Burns MB, Lynch J, Starr TK, Knights D, Blekhman R. Virulence genes
are a signature of the microbiome in the colorectal tumor
microenvironment. Genome Medicine. 2015;7:55.

\hypertarget{ref-zeller_potential_2014}{}
23. Zeller G, Tap J, Voigt AY, Sunagawa S, Kultima JR, Costea PI, et al.
Potential of fecal microbiota for early-stage detection of colorectal
cancer. Molecular Systems Biology. 2014;10:766.

\hypertarget{ref-baxter_microbiota-based_2016}{}
24. Baxter NT, Ruffin MT, Rogers MAM, Schloss PD. Microbiota-based model
improves the sensitivity of fecal immunochemical test for detecting
colonic lesions. Genome Medicine. 2016;8:37.

\hypertarget{ref-geng_diversified_2013}{}
25. Geng J, Fan H, Tang X, Zhai H, Zhang Z. Diversified pattern of the
human colorectal cancer microbiome. Gut Pathogens. 2013;5:2.

\hypertarget{ref-keku_gastrointestinal_2015}{}
26. Keku TO, Dulal S, Deveaux A, Jovov B, Han X. The gastrointestinal
microbiota and colorectal cancer. American Journal of Physiology -
Gastrointestinal and Liver Physiology {[}Internet{]}. 2015 {[}cited 2017
Oct 30{]};308:G351--63. Available from:
\url{http://ajpgi.physiology.org/lookup/doi/10.1152/ajpgi.00360.2012}

\hypertarget{ref-vogtmann_epidemiologic_2016}{}
27. Vogtmann E, Goedert JJ. Epidemiologic studies of the human
microbiome and cancer. British Journal of Cancer {[}Internet{]}. 2016
{[}cited 2017 Oct 30{]};114:237--42. Available from:
\url{http://www.nature.com/doifinder/10.1038/bjc.2015.465}

\hypertarget{ref-kostic_genomic_2012}{}
28. Kostic AD, Gevers D, Pedamallu CS, Michaud M, Duke F, Earl AM, et
al. Genomic analysis identifies association of Fusobacterium with
colorectal carcinoma. Genome Research. 2012;22:292--8.

\hypertarget{ref-zackular_human_2014}{}
29. Zackular JP, Rogers MAM, Ruffin MT, Schloss PD. The human gut
microbiome as a screening tool for colorectal cancer. Cancer Prevention
Research (Philadelphia, Pa.). 2014;7:1112--21.

\hypertarget{ref-schloss_introducing_2009}{}
30. Schloss PD, Westcott SL, Ryabin T, Hall JR, Hartmann M, Hollister
EB, et al. Introducing mothur: Open-Source, Platform-Independent,
Community-Supported Software for Describing and Comparing Microbial
Communities. Appl.Environ.Microbiol. {[}Internet{]}. 2009 {[}cited 12AD
Jan 1{]};75:7537--41. Available from:
\url{http://aem.asm.org/cgi/content/abstract/75/23/7537}

\hypertarget{ref-rognes_vsearch_2016}{}
31. Rognes T, Flouri T, Nichols B, Quince C, Mahé F. VSEARCH: A
versatile open source tool for metagenomics. PeerJ. 2016;4:e2584.

\hypertarget{ref-westcott_opticlust_2017}{}
32. Westcott SL, Schloss PD. OptiClust, an Improved Method for Assigning
Amplicon-Based Sequence Data to Operational Taxonomic Units. mSphere.
2017;2.

\hypertarget{ref-r_citation_2017}{}
33. R Core Team. R: A language and environment for statistical computing
{[}Internet{]}. Vienna, Austria: R Foundation for Statistical Computing;
2017. Available from: \url{https://www.R-project.org/}

\hypertarget{ref-rcompanion_citation_2017}{}
34. Mangiafico S. Rcompanion: Functions to support extension education
program evaluation {[}Internet{]}. 2017. Available from:
\url{https://CRAN.R-project.org/package=rcompanion}

\hypertarget{ref-car_citation_2011}{}
35. Fox J, Weisberg S. An R companion to applied regression
{[}Internet{]}. Second. Thousand Oaks CA: Sage; 2011. Available from:
\url{http://socserv.socsci.mcmaster.ca/jfox/Books/Companion}

\hypertarget{ref-lme4_citation_2015}{}
36. Bates D, Mächler M, Bolker B, Walker S. Fitting linear mixed-effects
models using lme4. Journal of Statistical Software. 2015;67:1--48.

\hypertarget{ref-epir_citation_2017}{}
37. Telmo Nunes MS with contributions from, Heuer C, Marshall J, Sanchez
J, Thornton R, Reiczigel J, et al. EpiR: Tools for the analysis of
epidemiological data {[}Internet{]}. 2017. Available from:
\url{https://CRAN.R-project.org/package=epiR}

\hypertarget{ref-metafor_citation_2010}{}
38. Viechtbauer W. Conducting meta-analyses in R with the metafor
package. Journal of Statistical Software {[}Internet{]}. 2010;36:1--48.
Available from: \url{http://www.jstatsoft.org/v36/i03/}

\hypertarget{ref-vegan_citation_2017}{}
39. Oksanen J, Blanchet FG, Friendly M, Kindt R, Legendre P, McGlinn D,
et al. Vegan: Community ecology package {[}Internet{]}. 2017. Available
from: \url{https://CRAN.R-project.org/package=vegan}

\hypertarget{ref-caret_citation_2017}{}
40. Jed Wing MKC from, Weston S, Williams A, Keefer C, Engelhardt A,
Cooper T, et al. Caret: Classification and regression training
{[}Internet{]}. 2017. Available from:
\url{https://CRAN.R-project.org/package=caret}

\hypertarget{ref-randomforest_citation_2002}{}
41. Liaw A, Wiener M. Classification and regression by randomForest. R
News {[}Internet{]}. 2002;2:18--22. Available from:
\url{http://CRAN.R-project.org/doc/Rnews/}

\hypertarget{ref-pwr_citation_2017}{}
42. Champely S. Pwr: Basic functions for power analysis {[}Internet{]}.
2017. Available from: \url{https://CRAN.R-project.org/package=pwr}

\hypertarget{ref-statmod_citation_2016}{}
43. Giner G, Smyth GK. Statmod: Probability calculations for the inverse
gaussian distribution. R Journal. 2016;8:339--51.

\hypertarget{ref-ggplot2_citation_2009}{}
44. Wickham H. Ggplot2: Elegant graphics for data analysis
{[}Internet{]}. Springer-Verlag New York; 2009. Available from:
\url{http://ggplot2.org}

\hypertarget{ref-gridextra_citation_2017}{}
45. Auguie B. GridExtra: Miscellaneous functions for ``grid'' graphics
{[}Internet{]}. 2017. Available from:
\url{https://CRAN.R-project.org/package=gridExtra}

\newpage

\textbf{Table 1: Total Individuals in each Study Included in the Stool
Analysis}

\footnotesize

\begin{longtable}[]{@{}cccccc@{}}
\toprule
Study & Data Stored & 16S Region & Control (n) & Adenoma (n) & Carcinoma
(n)\tabularnewline
\midrule
\endhead
Ahn & DBGap & V3-4 & 148 & 0 & 62\tabularnewline
Baxter & SRA & V4 & 172 & 198 & 120\tabularnewline
Brim & SRA & V1-3 & 6 & 6 & 0\tabularnewline
Flemer & Author & V3-4 & 37 & 0 & 43\tabularnewline
Hale & Author & V3-5 & 473 & 214 & 17\tabularnewline
Wang & SRA & V3 & 56 & 0 & 46\tabularnewline
Weir & Author & V4 & 4 & 0 & 7\tabularnewline
Zeller & SRA & V4 & 50 & 37 & 41\tabularnewline
\bottomrule
\end{longtable}

\normalsize
\newpage

\textbf{Table 2: Studies with Tissue Samples Included in the Analysis}

\footnotesize

\begin{longtable}[]{@{}cccccc@{}}
\toprule
Study & Data Stored & 16S Region & Control (n) & Adenoma (n) & Carcinoma
(n)\tabularnewline
\midrule
\endhead
Burns & SRA & V5-6 & 18 & 0 & 16\tabularnewline
Chen & SRA & V1-V3 & 9 & 0 & 9\tabularnewline
Dejea & SRA & V3-5 & 31 & 0 & 32\tabularnewline
Flemer & Author & V3-4 & 103 & 37 & 94\tabularnewline
Geng & SRA & V1-2 & 16 & 0 & 16\tabularnewline
Lu & SRA & V3-4 & 20 & 20 & 0\tabularnewline
Sanapareddy & Author & V1-2 & 38 & 0 & 33\tabularnewline
\bottomrule
\end{longtable}

\normalsize
\newpage

\textbf{Figure 1: \(\alpha\)-Diversity Differences between Control,
Adenoma, and Carcinoma Across Sampling Site.} A) \(\alpha\)-diversity
metric differences by group in stool samples. B) \(\alpha\)-diversity
metric differences by group in unmatched tissue samples. C)
\(\alpha\)-diversity metric differences by group in matched tissue
samples. The dashed line represents a Z-score of 0 or no difference from
the median.

\textbf{Figure 2: Relative Risk for Adenoma or Carcinoma based on
\(\alpha\)-Diversity Metrics in Stool.} A) \(\alpha\)-metric relative
risk for adenoma. B) \(\alpha\)-metric relative risk for carcinoma.
Colors represent the different variable regions used within the
respective study.

\textbf{Figure 3: CRC-Associated Genera Relative Risk for Adenoma and
Carcinoma in Stool and Tissue.} A) Adenoma relative risk in stool. B)
Carcinoma relative risk in stool. C) Adenoma relative risk in tissue. D)
Carcinoma relative risk in tissue. For all panels the relative risk was
also compared to whether one, two, three, or four of the CRC-associated
genera were present.

\textbf{Figure 4: OTU Random Forest Model of Stool Across Studies.} A)
Adenoma random forest model between the full community and
CRC-associated genera OTUs only. B) Carcinoma random forest model
between the full community and CRC-associated genera OTUs only. The
dotted line represents an AUC of 0.5 and the lines represent the range
in which the AUC for the 100 different 80/20 runs fell between.

\textbf{Figure 5: OTU Random Forest Model of Tissue Across Studies.} A)
Adenoma random forest model between the full community and
CRC-associated genera OTUs only. B) Carcinoma random forest model
between the full community and CRC-associated genera OTUs only. The
dotted line represents an AUC of 0.5 and the lines represent the range
in which the AUC for the 100 different 80/20 runs fell between.

\textbf{Figure 6: Power and Effect Size Analysis of Studies Included.}
A) Power based on effect size for studies with adenoma individuals. B)
Power based on effect size for studies with carcinoma individuals. C)
The estimated sample number needed for each arm of each study to detect
aneffect size of 1-30\%. The dotted red lines in A) and B) represent a
power of 0.8.

\newpage

\textbf{Figure S1: Relative Risk for Adenoma or Carcinoma based on
\(\alpha\)-Diversity Metrics in Tissue.} A) \(\alpha\)-metric relative
risk for adenoma. B) \(\alpha\)-metric relative risk for carcinoma.
Colors represent the different variable regions used within the
respective study.

\textbf{Figure S2: Random Forest Genus Model AUC for each Stool Study.}
A) AUC of adenoma models using all genera or CRC-associated genera only.
B) AUC of carcinoma models using all genera or CRC-associated genera
only. The black line represents the median within each group.

\textbf{Figure S3: Random Forest Genus Model AUC for each Tissue Study.}
A) AUC of adenoma models using all genera or only CRC-associated genera
divided between matched and unmatched tissue. B) AUC of carcinoma models
using all genera or CRC-associated genera only. The black line
represents the median within each group divided between matched and
unmatched tissue.

\textbf{Figure S4: Random Forest Prediction Success Using Genera for
each Stool Study.} A) AUC for prediction in adenoma using all genera or
CRC associated genera only. B) AUC for prediction in carcinoma using all
genera or CRC-associated genera only. The dotted line represents an AUC
of 0.5. The x-axis is the data set in which the model was initially
trained on.

\textbf{Figure S5: Random Forest Prediction Success of Carcinoma Using
Genera for each Tissue Study.} A) AUC for prediction in unmatched tissue
for all genera or CRC-associated genera only. B) AUC for prediction in
matched tissue using all genera or CRC-associated genera only. The
dotted line represents an AUC of 0.5. The x-axis is the data set in
which the model was initially trained on.

\textbf{Figure S6: Random Forest Prediction Success of Adenoma Using
Genera for each Tissue Study.}

\newpage


\end{document}
